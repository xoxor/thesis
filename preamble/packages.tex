\usepackage{fix-cm}
%Permits to copy eg x ⪰ y ⇔ v(x) ≥ v(y) from PDF to unicode data, and to search. From pdfTeX users manual. See https://tex.stackexchange.com/posts/comments/1203887.
	\input glyphtounicode
	\pdfgentounicode=1
%Latin Modern has more glyphs than Computer Modern, such as diacritical characters. fntguide commands to load the font before fontenc, to prevent default loading of cmr.
	\usepackage{lmodern}
%Encode resulting accented characters correctly in resulting PDF, permits copy from PDF.
	\usepackage[T1]{fontenc}
%UTF8 seems to be the default in recent TeX installations, but not all, see https://tex.stackexchange.com/a/370280.
	\usepackage[utf8]{inputenc}
	\usepackage{siunitx}
%Provides \newunicodechar for easy definition of supplementary UTF8 characters such as → or ≤ for use in source code.
	\usepackage{newunicodechar}
%Text Companion fonts, much used together with CM-like fonts. Provides \texteuro and commands for text mode characters such as \textminus, \textrightarrow, \textlbrackdbl.
	\usepackage{textcomp}
%St Mary’s Road symbol font, used for ⟦ = \llbracket.
	\usepackage{stmaryrd}
	\SetSymbolFont{stmry}{bold}{U}{stmry}{m}{n}
\usepackage[super]{nth}
\usepackage{centernot}
\usepackage{xpatch}
\usepackage{emptypage}
\usepackage[acronym,toc,automake,shortcuts]{glossaries}

\usepackage{caption}
\usepackage{subcaption}
\usepackage{booktabs}
\usepackage{graphicx}
\usepackage{times}
\usepackage{soul}
%\usepackage[hidelinks,hypertexnames=false]{hyperref}
\usepackage{xr}

\usepackage{fancyhdr}
%Headers style
\pagestyle{fancy}
\fancyhead{}
\fancyhead[LE]{\scshape{Part~\thepart}, \leftmark}
%\fancyhead[LE]{\leftmark}% LE -> Left part on Even pages
\fancyhead[RO]{\rightmark}% RO -> Right part on Odd pages
%% remove the number at Conclusion

\makeatletter
\renewcommand\chaptermark[1]{
	\markboth{\textsc{%
			\ifnum\c@secnumdepth>\m@ne\if@mainmatter
			\@chapapp\ \thechapter. \ \fi\fi #1}}{}%
}
\makeatother

\appto\backmatter{
	\pagestyle{fancy}
	\fancyhead[LE,RO]{\leftmark}
} 


%chapter
\makeatletter
\def\thickhrulefill{\leavevmode \leaders \hrule height 1ex \hfill \kern \z@}
\def\@makechapterhead#1{%
	\vspace*{-30\p@}%
	{\parindent \z@ \raggedright \reset@font
		\scshape \@chapapp{} \thechapter
		\par\nobreak
		\interlinepenalty\@M
		\Huge  #1\par\nobreak
		%\vspace*{1\p@}%
		\hrulefill
		\par\nobreak
		\vskip 50\p@
}}
\def\@makeschapterhead#1{%
	\vspace*{-50\p@}%
	{\parindent \z@ \raggedright \reset@font
		\scshape \vphantom{\@chapapp{} \thechapter}
		\par\nobreak
		\interlinepenalty\@M
		\Huge  #1 \par\nobreak
		%\vspace*{1\p@}%
		\hrulefill
		\par\nobreak
		\vskip 30\p@
}}

%part
\makeatletter
\def\@part[#1]#2{%
	\ifnum \c@secnumdepth >-2\relax
	\refstepcounter{part}%
	\addcontentsline{toc}{part}{\thepart\hspace{1em}#1}%
	\else
	\addcontentsline{toc}{part}{#1}%
	\fi
	\markboth{}{}%
	\vspace*{-50\p@}%
	{\parindent \z@ \centering \reset@font
		\LARGE \scshape{Part \thepart}
		\par\nobreak
		\scshape \vphantom{\partname}
		\par\nobreak
		\interlinepenalty\@M
		\Huge  #1 \par\nobreak
		%\Huge \upshape \bfseries #1 \par\nobreak
		%\vspace*{1\p@}%
		\hrulefill
		\par\nobreak
		\vskip 30\p@
	}
		\kern-2pt
	\cleardoublepage
}
\makeatother



\usepackage{lipsum}


%ntheorem doc says: “empheq provides an enhanced vertical placement of the endmarks”; must be loaded before ntheorem. Loads the mathtools package, which loads and fixes some bugs in amsmath and provides \DeclarePairedDelimiter. amsmath is considered a basic, mandatory package nowadays (Grätzer, More Math Into LaTeX).
	\usepackage[ntheorem]{empheq}
%Package frenchb asks to load natbib before babel-french. Package hyperref asks to load natbib before hyperref.
	\usepackage{natbib}

%Turns the doi provided by some bibliography styles into URLs. However, uses old-style dx.doi url (see 3.8 DOI system Proxy Server technical details, “Users may resolve DOI names that are structured to use the DOI system Proxy Server (https://doi.org (current, preferred) or earlier syntax http://dx.doi.org).”, https://www.doi.org/doi_handbook/3_Resolution.html). The patch solves this.
	\usepackage{doi}
	\makeatletter
	\patchcmd{\@doi}{http://dx.doi.org}{https://doi.org}{}{}
	\makeatother
%Makes sure upper case greek letters are italic as well.
	\usepackage{fixmath}
%Provides \mathbb; obsoletes latexsym (see http://tug.ctan.org/macros/latex/base/latexsym.dtx). Relatedly, \usepackage{eucal} to change the mathcal font and \usepackage[mathscr]{eucal} (apparently equivalent to \usepackage[mathscr]{euscript}) to supplement \mathcal with \mathscr. This last option is not very useful as both fonts are similar, and the intent of the authors of eucal was to provide a replacement to mathcal (see doc euscript). Also provides \mathfrak for supplementary letters.
	\usepackage{amsfonts}
%Provides a beautiful (IMHO) \mathscr and really different than \mathcal, for supplementary uppercase letters. But there is no bold version. Alternative: mathrsfs (more slanted), but when used with tikzposter, it warns about size substitution, see https://tex.stackexchange.com/q/495167.
	\usepackage[scr]{rsfso}
%Multiple means to produce bold math: \mathbf, \boldmath (defined to be \mathversion{bold}, see fntguide), \pmb, \boldsymbol (all legacy, from LaTeX base and AMS), \bm (the most recommended one), \mathbold from package fixmath (I don’t see its advantage over \boldsymbol).
%“The \boldsymbol command is obtained preferably by using the bm package, which provides a newer, more powerful version than the one provided by the amsmath package. Generally speaking, it is ill-advised to apply \boldsymbol to more than one symbol at a time.” — AMS Short math guide. “If no bold font appears to be available for a particular symbol, \bm will use ‘poor man’s bold’” — bm. It is “best to load the package after any packages that define new symbol fonts” – bm. bm defines \boldsymbol as synonym to \bm. \boldmath accesses the correct font if it exists; it is used by \bm when appropriate. See https://tex.stackexchange.com/a/10643 and https://github.com/latex3/latex2e/issues/71 for some difficulties with \bm.
	\usepackage{bm}
%Also loaded by tikz.
	\usepackage{xcolor}
%Vizualization, on top of TikZ
	\usepackage{pgfplots}
	\pgfplotsset{compat=1.14}
	\usepackage{graphicx}
	\graphicspath{{graphics/}}
	\usepackage{tikz}
%	\usepackage[intoc]{acro}

\usepackage{hyperref}
\hypersetup{plainpages=false}

%Provides \print­length{length}, useful for debugging.
	%\usepackage{printlen}
	%\uselengthunit{mm}

\usepackage[capitalise]{cleveref}
%Do not use the displaymath environment: use equation. Do not use the eqnarray or eqnarray* environments: use align(*). This improves spacing. (See l2tabu or amsldoc.)
\let\proof\relax
\let\endproof\relax
\usepackage{amsthm}