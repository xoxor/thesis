%basic
\NewDocumentCommand{\R}{}{ℝ}
\NewDocumentCommand{\N}{}{ℕ}
%\mathscr is rounder than \mathcal.
\NewDocumentCommand{\powerset}{m}{\mathscr{P}(#1)}
%Powerset without zero.
\NewDocumentCommand{\powersetz}{m}{\mathscr{P}^*(#1)}
%https://tex.stackexchange.com/a/45732, works within both \set and \set*, same spacing than \mid (https://tex.stackexchange.com/a/52905).
\NewDocumentCommand{\suchthat}{}{\;\ifnum\currentgrouptype=16 \middle\fi|\;}
%Integer interval.
\NewDocumentCommand{\intvl}{m}{\llbracket#1\rrbracket}
%Allows for \abs and \abs*, which resizes the delimiters.
\DeclarePairedDelimiter\abs{\lvert}{\rvert}
\DeclarePairedDelimiter\card{\lvert}{\rvert}
%Perhaps should use U+2016 ‖ DOUBLE VERTICAL LINE here?
\DeclarePairedDelimiter\norm{\lVert}{\rVert}
%Better than using the package braket because braket introduces possibly undesirable space. Then: \begin{equation}\set*{x \in \R^2 \suchthat \norm{x}<5}\end{equation}.
\DeclarePairedDelimiter\set{\{}{\}}
\DeclarePairedDelimiter\ceil{\lceil}{\rceil}
\DeclarePairedDelimiter\floor{\lfloor}{\rfloor}

%Decision Theory (MCDA and SC)
\NewDocumentCommand{\allalts}{}{A}
\NewDocumentCommand{\allcrits}{}{\mathscr{C}}
\NewDocumentCommand{\alts}{}{A}
\NewDocumentCommand{\dm}{}{i}
\NewDocumentCommand{\allF}{}{\mathscr{F}}
\NewDocumentCommand{\allvoters}{}{\mathscr{N}}
\NewDocumentCommand{\voters}{}{N}
\NewDocumentCommand{\allprofs}{}{\boldsymbol{\mathcal{R}}}
\NewDocumentCommand{\prof}{}{P}
\NewDocumentCommand{\ibar}{}{\overline{i}}
\NewDocumentCommand{\lprof}{}{\lambda_P}
\NewDocumentCommand{\lprofi}{O{x}}{\lambda_P(#1)_i}
\NewDocumentCommand{\lprofibar}{O{x}}{\lambda_P(#1)_{\overline{i}}}
\NewDocumentCommand{\ineq}{}{(\sigma \circ \lambda_P)}

\NewDocumentCommand{\linors}{}{\mathcal{L}(\allalts)}
%Thanks to https://tex.stackexchange.com/q/154549
	%\makeatletter
	%\def\@myRgood@#1#2{\mathrel{R^X_{#2}}}
	%\def\myRgood{\@ifnextchar_{\@myRgood@}{\mathrel{R^X}}}
	%\makeatother
\NewDocumentCommand{\prefi}{O{i}}{\succ_{#1}}
\NewDocumentCommand{\paretopt}{}{\text{PO}}
\NewDocumentCommand{\SPPd}{}{\Sigma^\text{PPd}}
\NewDocumentCommand{\SAll}{}{\Sigma^\text{All}}
\NewDocumentCommand{\SThreshold}{}{\Sigma_\text{threshold}}
\NewDocumentCommand{\vpr}{}{\boldsymbol{v}}

\NewDocumentCommand{\musigma}{O{\sigma}O{P}}{\min_{A}({#1}\circ\lambda_{{#2}})}
\NewDocumentCommand{\mustar}{O{\sigma}O{P}}{\min_{\paretopt({#2})} ({#1} \circ \lambda_{#2})}
\NewDocumentCommand{\minineq}{O{\allalts}}{\min_{#1}(\sigma \circ \lambda)}
\NewDocumentCommand{\FBP}{}{\text{FB}(P)}
\NewDocumentCommand{\POP}{}{\text{PO}(P)}

\NewDocumentCommand{\alllosses}{}{\intvl{0, m-1}^N}

\NewDocumentCommand{\Ptop}{}{\bar{P}}
\NewDocumentCommand{\sigmatop}{}{\bar{\sigma}}
\NewDocumentCommand{\smad}{}{\sigma_\text{mad}}

\NewDocumentCommand{\fltwo}{}{\floor{\bar{l_2}}}
\NewDocumentCommand{\bltwo}{}{\bar{l_2}}

\newtheorem{conjecture}{Conjecture}
\newtheorem{example}{Example}
\newenvironment{abstract}{\rightskip1in\itshape}{}

\newtheorem{theorem}{Theorem}
\newtheorem{claim}{Claim}
\newtheorem{prop}{Proposition}
\newtheorem{corollary}{Corollary}
\newtheorem{definition}{Definition}
%\newtheorem{proof}{Proof}
\newtheorem{proposition}{Proposition}
\newtheorem{remark}{Remark}
%\newtheorem{sketch}[proof]{Proof Sketch}

%% generic theorem 

\newtheorem*{nonamethmplain}{\nonamethmname}
\theoremstyle{definition}
\newtheorem*{nonamethmdefinition}{\nonamethmname}
\theoremstyle{remark}
\newtheorem*{nonamethmremark}{\nonamethmname}
\newcommand{\nonamethmname}{}

\NewDocumentEnvironment{genthm}{O{plain}m}
{\vspace{0.2em}\renewcommand{\nonamethmname}{#2}\begin{nonamethm#1}}
	{{\footnotesize\qed} \end{nonamethm#1}}
\NewDocumentEnvironment{genthm*}{O{plain}mo}
{\renewcommand{\nonamethmname}{#2}%
	\IfNoValueTF{#3}
	{\begin{nonamethm#1}\relax}%
		{\begin{nonamethm#1}[#3]}%
			\mbox{}}
		{\end{nonamethm#1}}

	
\newtheoremstyle{indented}{3pt}{3pt}{\addtolength{\leftskip}{2.5em}}{}{\bfseries}{.}{\newline}{\thmname{#3}}
\theoremstyle{indented}
\newtheorem*{indented}{\nonamethmname}


%--------minimax-----
\DeclareMathOperator{\Regret}{Regret}
\DeclareMathOperator{\SCORE}{Score}
\DeclareMathOperator{\PMR}{PMR}
\DeclareMathOperator{\MaxR}{MR}
\DeclareMathOperator{\MMR}{MMR}
\DeclareMathOperator{\leximax}{leximax}
\DeclareMathOperator*{\argmax}{argmax}
\DeclareMathOperator*{\argmin}{argmin}

%Thanks to https://tex.stackexchange.com/q/154549
\makeatletter
\newcommand{\newrelation}[2]{% #1 = control sequence, #2 = replacement text
	\@ifdefinable{#1}{%
		\def#1{%
			\@ifnextchar_{\csname\string#1\endcsname}{\mathrel{#2}}%
		}%
		\@namedef{\string#1}##1##2{\mathrel{#2_{##2}}}%
	}%
}
\makeatother

\newrelation{\prefinc}{\!\parallel\!}%partial pref, complement (incomparable)
\newrelation{\pinc}{\bowtie^\text{p}}
%\newrelation{\prefinc}{Q^\text{p}}%partial pref, complement (incomparable)

\newcommand{\profile}{\bm{v}}%(complete) profile
\newcommand{\pprofile}{{\bm{p}}}%partial profile
\newcommand{\w}{\bm{w}}
\newcommand{\W}{\mathcal{W}}
\newcommand{\Co}{\mathcal{C}}
\newcommand{\pw}{W}%our knowledge about the weights
\newcommand{\strat}[1]{\emph{#1}}
\newcommand{\pref}{\succ}%real, connected pref, strict
\newcommand{\prefeq}{\succeq}%real, connected pref, strict
\newcommand{\prefr}{{\succ}^\text{r}}%real, connected pref, strict
\newcommand{\pprefeq}{\succeq^\text{p}}%partial pref
\newcommand{\ppref}{\succ^\text{p}}%partial pref
\newcommand{\pprefinv}{\prec^\text{p}}%partial pref
\newcommand{\nppref}{\nsucc^\text{p}}%negated partial pref


%------MJ----
\NewDocumentCommand{\dbar}{}{\overline{\delta}}
\NewDocumentCommand{\reddbar}{}{{\color{red}\overline{\delta}}}
\NewDocumentCommand{\second}{}{\prime\prime}
\NewDocumentCommand{\Pbar}{}{\overline{P}}
\NewDocumentCommand{\fmaj}{}{f_{maj}}
\NewDocumentCommand{\Fmaj}{}{F_{maj}}
\NewDocumentCommand{\fmajbar}{}{\overline{f_{maj}}}
\NewDocumentCommand{\Fmajbar}{}{\overline{F_{maj}}}
\NewDocumentCommand{\med}{}{\text{med}}
\NewDocumentCommand\restr{mm}{#1\raisebox{-.5ex}{$|$}_{#2}}