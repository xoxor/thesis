\listfiles
\documentclass[a4paper, 11pt]{book}

\usepackage{psl-cover}
\usepackage[headheight=14.5pt]{geometry}
\usepackage{imakeidx}
\makeindex[title = List of Definitions, intoc, options=-s index.ist]

\usepackage{fix-cm}
%Permits to copy eg x ⪰ y ⇔ v(x) ≥ v(y) from PDF to unicode data, and to search. From pdfTeX users manual. See https://tex.stackexchange.com/posts/comments/1203887.
	\input glyphtounicode
	\pdfgentounicode=1
%Latin Modern has more glyphs than Computer Modern, such as diacritical characters. fntguide commands to load the font before fontenc, to prevent default loading of cmr.
	\usepackage{lmodern}
%Encode resulting accented characters correctly in resulting PDF, permits copy from PDF.
	\usepackage[T1]{fontenc}
%UTF8 seems to be the default in recent TeX installations, but not all, see https://tex.stackexchange.com/a/370280.
	\usepackage[utf8]{inputenc}
	\usepackage{siunitx}
%Provides \newunicodechar for easy definition of supplementary UTF8 characters such as → or ≤ for use in source code.
	\usepackage{newunicodechar}
%Text Companion fonts, much used together with CM-like fonts. Provides \texteuro and commands for text mode characters such as \textminus, \textrightarrow, \textlbrackdbl.
	\usepackage{textcomp}
%St Mary’s Road symbol font, used for ⟦ = \llbracket.
	\usepackage{stmaryrd}
	\SetSymbolFont{stmry}{bold}{U}{stmry}{m}{n}
\usepackage[super]{nth}
\usepackage{centernot}
\usepackage{xpatch}
\usepackage{emptypage}
\usepackage[acronym,toc,automake,shortcuts]{glossaries}

\usepackage{caption}
\usepackage{subcaption}
\usepackage{booktabs}
\usepackage{graphicx}
\usepackage{times}
\usepackage{soul}
%\usepackage[hidelinks,hypertexnames=false]{hyperref}
\usepackage{xr}
\usepackage{dirtytalk}
\usepackage{fancyhdr}
\usepackage{paracol}
%Headers style
\pagestyle{fancy}
\fancyhead{}
\fancyhead[LE]{\scshape{Part~\thepart}, \leftmark}
%\fancyhead[LE]{\leftmark}% LE -> Left part on Even pages
\fancyhead[RO]{\rightmark}% RO -> Right part on Odd pages
%% remove the number at Conclusion

\makeatletter
\renewcommand\chaptermark[1]{
	\markboth{\textsc{%
			\ifnum\c@secnumdepth>\m@ne\if@mainmatter
			\@chapapp\ \thechapter. \ \fi\fi #1}}{}%
}
\makeatother

\appto\backmatter{
	\pagestyle{fancy}
	\fancyhead[LE,RO]{\leftmark}
} 


%chapter
\makeatletter
\def\thickhrulefill{\leavevmode \leaders \hrule height 1ex \hfill \kern \z@}
\def\@makechapterhead#1{%
	\vspace*{-30\p@}%
	{\parindent \z@ \raggedright \reset@font
		\scshape \@chapapp{} \thechapter
		\par\nobreak
		\interlinepenalty\@M
		\Huge  #1\par\nobreak
		%\vspace*{1\p@}%
		\hrulefill
		\par\nobreak
		\vskip 50\p@
}}
\def\@makeschapterhead#1{%
	\vspace*{-50\p@}%
	{\parindent \z@ \raggedright \reset@font
		\scshape \vphantom{\@chapapp{} \thechapter}
		\par\nobreak
		\interlinepenalty\@M
		\Huge  #1 \par\nobreak
		%\vspace*{1\p@}%
		\hrulefill
		\par\nobreak
		\vskip 30\p@
}}

%part
\makeatletter
\def\@part[#1]#2{%
	\ifnum \c@secnumdepth >-2\relax
	\refstepcounter{part}%
	\addcontentsline{toc}{part}{\thepart\hspace{1em}#1}%
	\else
	\addcontentsline{toc}{part}{#1}%
	\fi
	\markboth{}{}%
	\vspace*{-50\p@}%
	{\parindent \z@ \centering \reset@font
		\LARGE \scshape{Part \thepart}
		\par\nobreak
		\scshape \vphantom{\partname}
		\par\nobreak
		\interlinepenalty\@M
		\Huge  #1 \par\nobreak
		%\Huge \upshape \bfseries #1 \par\nobreak
		%\vspace*{1\p@}%
		\hrulefill
		\par\nobreak
		\vskip 30\p@
	}
		\kern-2pt
	\cleardoublepage
}
\makeatother



\usepackage{lipsum}


%ntheorem doc says: “empheq provides an enhanced vertical placement of the endmarks”; must be loaded before ntheorem. Loads the mathtools package, which loads and fixes some bugs in amsmath and provides \DeclarePairedDelimiter. amsmath is considered a basic, mandatory package nowadays (Grätzer, More Math Into LaTeX).
	\usepackage[ntheorem]{empheq}
%Package frenchb asks to load natbib before babel-french. Package hyperref asks to load natbib before hyperref.
	\usepackage{natbib}

%Turns the doi provided by some bibliography styles into URLs. However, uses old-style dx.doi url (see 3.8 DOI system Proxy Server technical details, “Users may resolve DOI names that are structured to use the DOI system Proxy Server (https://doi.org (current, preferred) or earlier syntax http://dx.doi.org).”, https://www.doi.org/doi_handbook/3_Resolution.html). The patch solves this.
	\usepackage{doi}
	\makeatletter
	\patchcmd{\@doi}{http://dx.doi.org}{https://doi.org}{}{}
	\makeatother
%Makes sure upper case greek letters are italic as well.
	\usepackage{fixmath}
%Provides \mathbb; obsoletes latexsym (see http://tug.ctan.org/macros/latex/base/latexsym.dtx). Relatedly, \usepackage{eucal} to change the mathcal font and \usepackage[mathscr]{eucal} (apparently equivalent to \usepackage[mathscr]{euscript}) to supplement \mathcal with \mathscr. This last option is not very useful as both fonts are similar, and the intent of the authors of eucal was to provide a replacement to mathcal (see doc euscript). Also provides \mathfrak for supplementary letters.
	\usepackage{amsfonts}
%Provides a beautiful (IMHO) \mathscr and really different than \mathcal, for supplementary uppercase letters. But there is no bold version. Alternative: mathrsfs (more slanted), but when used with tikzposter, it warns about size substitution, see https://tex.stackexchange.com/q/495167.
	\usepackage[scr]{rsfso}
%Multiple means to produce bold math: \mathbf, \boldmath (defined to be \mathversion{bold}, see fntguide), \pmb, \boldsymbol (all legacy, from LaTeX base and AMS), \bm (the most recommended one), \mathbold from package fixmath (I don’t see its advantage over \boldsymbol).
%“The \boldsymbol command is obtained preferably by using the bm package, which provides a newer, more powerful version than the one provided by the amsmath package. Generally speaking, it is ill-advised to apply \boldsymbol to more than one symbol at a time.” — AMS Short math guide. “If no bold font appears to be available for a particular symbol, \bm will use ‘poor man’s bold’” — bm. It is “best to load the package after any packages that define new symbol fonts” – bm. bm defines \boldsymbol as synonym to \bm. \boldmath accesses the correct font if it exists; it is used by \bm when appropriate. See https://tex.stackexchange.com/a/10643 and https://github.com/latex3/latex2e/issues/71 for some difficulties with \bm.
	\usepackage{bm}
%Also loaded by tikz.
	\usepackage{xcolor}
%Vizualization, on top of TikZ
	\usepackage{pgfplots}
	\pgfplotsset{compat=1.14}
	\usepackage{graphicx}
	\graphicspath{{graphics/}}
	\usepackage{tikz}
%	\usepackage[intoc]{acro}

\usepackage{hyperref}
\hypersetup{plainpages=false}

%Provides \print­length{length}, useful for debugging.
	%\usepackage{printlen}
	%\uselengthunit{mm}

\usepackage[capitalise]{cleveref}
%Do not use the displaymath environment: use equation. Do not use the eqnarray or eqnarray* environments: use align(*). This improves spacing. (See l2tabu or amsldoc.)
\let\proof\relax
\let\endproof\relax
\usepackage{amsthm}
%basic
\NewDocumentCommand{\R}{}{ℝ}
\NewDocumentCommand{\N}{}{ℕ}
%\mathscr is rounder than \mathcal.
\NewDocumentCommand{\powerset}{m}{\mathscr{P}(#1)}
%Powerset without zero.
\NewDocumentCommand{\powersetz}{m}{\mathscr{P}^*(#1)}
%https://tex.stackexchange.com/a/45732, works within both \set and \set*, same spacing than \mid (https://tex.stackexchange.com/a/52905).
\NewDocumentCommand{\suchthat}{}{\;\ifnum\currentgrouptype=16 \middle\fi|\;}
%Integer interval.
\NewDocumentCommand{\intvl}{m}{\llbracket#1\rrbracket}
%Allows for \abs and \abs*, which resizes the delimiters.
\DeclarePairedDelimiter\abs{\lvert}{\rvert}
\DeclarePairedDelimiter\card{\lvert}{\rvert}
%Perhaps should use U+2016 ‖ DOUBLE VERTICAL LINE here?
\DeclarePairedDelimiter\norm{\lVert}{\rVert}
%Better than using the package braket because braket introduces possibly undesirable space. Then: \begin{equation}\set*{x \in \R^2 \suchthat \norm{x}<5}\end{equation}.
\DeclarePairedDelimiter\set{\{}{\}}
\DeclarePairedDelimiter\ceil{\lceil}{\rceil}
\DeclarePairedDelimiter\floor{\lfloor}{\rfloor}

%Decision Theory (MCDA and SC)
\NewDocumentCommand{\allalts}{}{A}
\NewDocumentCommand{\allcrits}{}{\mathscr{C}}
\NewDocumentCommand{\alts}{}{A}
\NewDocumentCommand{\dm}{}{i}
\NewDocumentCommand{\allF}{}{\mathscr{F}}
\NewDocumentCommand{\allvoters}{}{\mathscr{N}}
\NewDocumentCommand{\voters}{}{N}
\NewDocumentCommand{\allprofs}{}{\boldsymbol{\mathcal{R}}}
\NewDocumentCommand{\prof}{}{P}
\NewDocumentCommand{\ibar}{}{\overline{i}}
\NewDocumentCommand{\lprof}{}{\lambda_P}
\NewDocumentCommand{\lprofi}{O{x}}{\lambda_P(#1)_i}
\NewDocumentCommand{\lprofibar}{O{x}}{\lambda_P(#1)_{\overline{i}}}
\NewDocumentCommand{\ineq}{}{(\sigma \circ \lambda_P)}

\NewDocumentCommand{\linors}{}{\mathcal{L}(\allalts)}
%Thanks to https://tex.stackexchange.com/q/154549
	%\makeatletter
	%\def\@myRgood@#1#2{\mathrel{R^X_{#2}}}
	%\def\myRgood{\@ifnextchar_{\@myRgood@}{\mathrel{R^X}}}
	%\makeatother
\NewDocumentCommand{\prefi}{O{i}}{\succ_{#1}}
\NewDocumentCommand{\paretopt}{}{\text{PO}}
\NewDocumentCommand{\SPPd}{}{\Sigma^\text{PPd}}
\NewDocumentCommand{\SAll}{}{\Sigma^\text{All}}
\NewDocumentCommand{\SThreshold}{}{\Sigma_\text{threshold}}
\NewDocumentCommand{\vpr}{}{\boldsymbol{v}}

\NewDocumentCommand{\musigma}{O{\sigma}O{P}}{\min_{A}({#1}\circ\lambda_{{#2}})}
\NewDocumentCommand{\mustar}{O{\sigma}O{P}}{\min_{\paretopt({#2})} ({#1} \circ \lambda_{#2})}
\NewDocumentCommand{\minineq}{O{\allalts}}{\min_{#1}(\sigma \circ \lambda)}
\NewDocumentCommand{\FBP}{}{\text{FB}(P)}
\NewDocumentCommand{\POP}{}{\text{PO}(P)}

\NewDocumentCommand{\alllosses}{}{\intvl{0, m-1}^N}

\NewDocumentCommand{\Ptop}{}{\bar{P}}
\NewDocumentCommand{\sigmatop}{}{\bar{\sigma}}
\NewDocumentCommand{\smad}{}{\sigma_\text{mad}}

\NewDocumentCommand{\fltwo}{}{\floor{\bar{l_2}}}
\NewDocumentCommand{\bltwo}{}{\bar{l_2}}

\newtheorem{conjecture}{Conjecture}
\newtheorem{example}{Example}
\newenvironment{abstract}{\rightskip1in\itshape}{}

\newtheorem{theorem}{Theorem}
\newtheorem{claim}{Claim}
\newtheorem{prop}{Proposition}
\newtheorem{corollary}{Corollary}
\newtheorem{definition}{Definition}
%\newtheorem{proof}{Proof}
\newtheorem{proposition}{Proposition}
\newtheorem{remark}{Remark}
%\newtheorem{sketch}[proof]{Proof Sketch}

%% generic theorem 

\newtheorem*{nonamethmplain}{\nonamethmname}
\theoremstyle{definition}
\newtheorem*{nonamethmdefinition}{\nonamethmname}
\theoremstyle{remark}
\newtheorem*{nonamethmremark}{\nonamethmname}
\newcommand{\nonamethmname}{}

\NewDocumentEnvironment{genthm}{O{plain}m}
{\vspace{0.2em}\renewcommand{\nonamethmname}{#2}\begin{nonamethm#1}}
	{{\footnotesize\qed} \end{nonamethm#1}}
\NewDocumentEnvironment{genthm*}{O{plain}mo}
{\renewcommand{\nonamethmname}{#2}%
	\IfNoValueTF{#3}
	{\begin{nonamethm#1}\relax}%
		{\begin{nonamethm#1}[#3]}%
			\mbox{}}
		{\end{nonamethm#1}}

	
\newtheoremstyle{indented}{3pt}{3pt}{\addtolength{\leftskip}{2.5em}}{}{\bfseries}{.}{\newline}{\thmname{#3}}
\theoremstyle{indented}
\newtheorem*{indented}{\nonamethmname}


%--------minimax-----
\DeclareMathOperator{\Regret}{Regret}
\DeclareMathOperator{\SCORE}{Score}
\DeclareMathOperator{\PMR}{PMR}
\DeclareMathOperator{\MaxR}{MR}
\DeclareMathOperator{\MMR}{MMR}
\DeclareMathOperator{\leximax}{leximax}
\DeclareMathOperator*{\argmax}{argmax}
\DeclareMathOperator*{\argmin}{argmin}

%Thanks to https://tex.stackexchange.com/q/154549
\makeatletter
\newcommand{\newrelation}[2]{% #1 = control sequence, #2 = replacement text
	\@ifdefinable{#1}{%
		\def#1{%
			\@ifnextchar_{\csname\string#1\endcsname}{\mathrel{#2}}%
		}%
		\@namedef{\string#1}##1##2{\mathrel{#2_{##2}}}%
	}%
}
\makeatother

\newrelation{\prefinc}{\!\parallel\!}%partial pref, complement (incomparable)
\newrelation{\pinc}{\bowtie^\text{p}}
%\newrelation{\prefinc}{Q^\text{p}}%partial pref, complement (incomparable)

\newcommand{\profile}{\bm{v}}%(complete) profile
\newcommand{\pprofile}{{\bm{p}}}%partial profile
\newcommand{\w}{\bm{w}}
\newcommand{\W}{\mathcal{W}}
\newcommand{\Co}{\mathcal{C}}
\newcommand{\pw}{W}%our knowledge about the weights
\newcommand{\strat}[1]{\emph{#1}}
\newcommand{\pref}{\succ}%real, connected pref, strict
\newcommand{\prefeq}{\succeq}%real, connected pref, strict
\newcommand{\prefr}{{\succ}^\text{r}}%real, connected pref, strict
\newcommand{\pprefeq}{\succeq^\text{p}}%partial pref
\newcommand{\ppref}{\succ^\text{p}}%partial pref
\newcommand{\pprefinv}{\prec^\text{p}}%partial pref
\newcommand{\nppref}{\nsucc^\text{p}}%negated partial pref


\newcommand{\oquot}{``}
\newcommand{\cquot}{''}

%Requires package xcolor.
\NewDocumentCommand{\commentOC}{m}{\textcolor{blue}{\small$\big[$OC: #1$\big]$}}
\NewDocumentCommand{\commentBN}{m}{\textcolor{magenta}{\small$\big[$BN: #1$\big]$}}
\NewDocumentCommand{\commentRS}{m}{\textcolor{red}{\small$\big[$RS: #1$\big]$}}

\bibliographystyle{abbrvnat}
\NewDocumentCommand{\possessivecite}{m}{\citeauthor{#1}’s \citeyearpar{#1}}

%https://tex.stackexchange.com/a/467188, https://tex.stackexchange.com/a/36088 - uncomment if one of those symbols is used.
%\DeclareFontFamily{U} {MnSymbolD}{}
%\DeclareFontShape{U}{MnSymbolD}{m}{n}{
%  <-6> MnSymbolD5
%  <6-7> MnSymbolD6
%  <7-8> MnSymbolD7
%  <8-9> MnSymbolD8
%  <9-10> MnSymbolD9
%  <10-12> MnSymbolD10
%  <12-> MnSymbolD12}{}
%\DeclareFontShape{U}{MnSymbolD}{b}{n}{
%  <-6> MnSymbolD-Bold5
%  <6-7> MnSymbolD-Bold6
%  <7-8> MnSymbolD-Bold7
%  <8-9> MnSymbolD-Bold8
%  <9-10> MnSymbolD-Bold9
%  <10-12> MnSymbolD-Bold10
%  <12-> MnSymbolD-Bold12}{}
%\DeclareSymbolFont{MnSyD} {U} {MnSymbolD}{m}{n}
%\DeclareMathSymbol{\ntriplesim}{\mathrel}{MnSyD}{126}
%\DeclareMathSymbol{\nlessgtr}{\mathrel}{MnSyD}{192}
%\DeclareMathSymbol{\ngtrless}{\mathrel}{MnSyD}{193}
%\DeclareMathSymbol{\nlesseqgtr}{\mathrel}{MnSyD}{194}
%\DeclareMathSymbol{\ngtreqless}{\mathrel}{MnSyD}{195}
%\DeclareMathSymbol{\nlesseqgtrslant}{\mathrel}{MnSyD}{198}
%\DeclareMathSymbol{\ngtreqlessslant}{\mathrel}{MnSyD}{199}
%\DeclareMathSymbol{\npreccurlyeq}{\mathrel}{MnSyD}{228}
%\DeclareMathSymbol{\nsucccurlyeq}{\mathrel}{MnSyD}{229}
%\DeclareFontFamily{U} {MnSymbolA}{}
%\DeclareFontShape{U}{MnSymbolA}{m}{n}{
%  <-6> MnSymbolA5
%  <6-7> MnSymbolA6
%  <7-8> MnSymbolA7
%  <8-9> MnSymbolA8
%  <9-10> MnSymbolA9
%  <10-12> MnSymbolA10
%  <12-> MnSymbolA12}{}
%\DeclareFontShape{U}{MnSymbolA}{b}{n}{
%  <-6> MnSymbolA-Bold5
%  <6-7> MnSymbolA-Bold6
%  <7-8> MnSymbolA-Bold7
%  <8-9> MnSymbolA-Bold8
%  <9-10> MnSymbolA-Bold9
%  <10-12> MnSymbolA-Bold10
%  <12-> MnSymbolA-Bold12}{}
%\DeclareSymbolFont{MnSyA} {U} {MnSymbolA}{m}{n}
%%Rightwards wave arrow: ↝. Alternative: \rightsquigarrow from amssymb, but it’s uglier
%\DeclareMathSymbol{\rightlsquigarrow}{\mathrel}{MnSyA}{160}

%03B3 Greek Small Letter Gamma
\newunicodechar{γ}{\gamma}
%03B4 Greek Small Letter Delta
\newunicodechar{δ}{\delta}
%2115 Double-Struck Capital N
\newunicodechar{ℕ}{\mathbb{N}}
%211D Double-Struck Capital R
\newunicodechar{ℝ}{\mathbb{R}}
%21CF Rightwards Double Arrow with Stroke
\newunicodechar{⇏}{\nRightarrow}
%21D2 Rightwards Double Arrow
\newunicodechar{⇒}{\ensuremath{\Rightarrow}}
%21D4 Left Right Double Arrow
\newunicodechar{⇔}{\Leftrightarrow}
%21DD Rightwards Squiggle Arrow
\newunicodechar{⇝}{\rightsquigarrow}
%2212 Minus Sign
\newunicodechar{−}{\ifmmode{-}\else\textminus\fi}
%2227 Logical And
\newunicodechar{∧}{\land}
%2228 Logical Or
\newunicodechar{∨}{\lor}
%2229 Intersection
\newunicodechar{∩}{\cap}
%222A Union
\newunicodechar{∪}{\cup}
%2260 Not Equal To (handy also as text in informal writing)
\newunicodechar{≠}{\ensuremath{\neq}}
%2264 Less-Than or Equal To
\newunicodechar{≤}{\leq}
%2265 Greater-Than or Equal To
\newunicodechar{≥}{\geq}
%2270 Neither Less-Than nor Equal To
\newunicodechar{≰}{\nleq}
%2271 Neither Greater-Than nor Equal To
\newunicodechar{≱}{\ngeq}
%2272 Less-Than or Equivalent To
\newunicodechar{≲}{\lesssim}
%2273 Greater-Than or Equivalent To
\newunicodechar{≳}{\gtrsim}
%2274 Neither Less-Than nor Equivalent To – also, from MnSymbol: \nprecsim, a more exact match to the Unicode symbol; and \npreccurlyeq, too small
\newunicodechar{≴}{\not\preccurlyeq}
%2275 Neither Greater-Than nor Equivalent To
\newunicodechar{≵}{\not\succcurlyeq}
%2279 Neither Greater-Than nor Less-Than – requires MnSymbol; also \nlessgtr from txfonts/pxfonts, \ngtreqless from MnSymbol (but much higher), \ngtrless from MnSymbol (a more exact match to the Unicode symbol); for incomparability (not matching this Unicode symbol), may also consider \ntriplesim from MnSymbol,\nparallelslant from fourier, \between from mathabx, or ⋈
\newunicodechar{≹}{\ngtreqlessslant}
%227A Precedes
\newunicodechar{≺}{\prec}
%227B Succeeds
\newunicodechar{≻}{\succ}
%227C Precedes or Equal To
\newunicodechar{≼}{\preccurlyeq}
%227D Succeeds or Equal To
\newunicodechar{≽}{\succcurlyeq}
%227E Precedes or Equivalent To
\newunicodechar{≾}{\precsim}
%227F Succeeds or Equivalent To
\newunicodechar{≿}{\succsim}
%2280 Does Not Precede
\newunicodechar{⊀}{\nprec}
%2281 Does Not Succeed
\newunicodechar{⊁}{\nsucc}
%22B2 Normal Subgroup Of – using \vartriangleleft from amsfonts, which goes well with \trianglelefteq, \ntriangleright, and so on, also from amsfonts; another possibility is \lhd from latexsym, which seems visually equivalent to \vartriangleleft from amsfonts; latexsym also has ⊴=\unlhd, but doesn’t have a symbol for ⊴. Other related symbols: \triangleleft from latesym package is too small; fdsymbol provides \triangleleft=\medtriangleleft and \vartriangleleft=\smalltriangleleft; MnSymbol provides \medtriangleleft and \vartriangleleft=\lessclosed=\lhd which are smaller than \vartriangleleft from amsfont; \vartriangleleft from mathabx (p. 67), looks different (wider); also \vartriangleleft from boisik (p. 69) looks still different; \vartriangleleft=\lhd from stix are smaller. Oddly enough, \triangleright appears as the LMMathItalic12-Regular font whereas \rhd appears as LASY10 and \vartriangleright appears as MSAM10.
\newunicodechar{⊲}{\vartriangleleft}
%22B3 Contains as Normal Subgroup (also: 25B7 White right-pointing triangle or 25B9 White right-pointing small triangle)
\newunicodechar{⊳}{\vartriangleright}
%22B4 Normal Subgroup of or Equal To
\newunicodechar{⊴}{\trianglelefteq}
%22B5 Contains as Normal Subgroup or Equal To
\newunicodechar{⊵}{\trianglerighteq}
%22C8 Bowtie
\newunicodechar{⋈}{\bowtie}
%22EA Not Normal Subgroup Of
\newunicodechar{⋪}{\ntriangleleft}
%22EB Does Not Contain As Normal Subgroup
\newunicodechar{⋫}{\ntriangleright}
%22EC Not Normal Subgroup of or Equal To
\newunicodechar{⋬}{\ntrianglelefteq}
%22ED Does Not Contain as Normal Subgroup or Equal
\newunicodechar{⋭}{\ntrianglerighteq}
%25A1 White Square
\newunicodechar{□}{\Box}
%27E6 Mathematical Left White Square Bracket – requires stmaryrd (alternative: \text{\textlbrackdbl}, but ugly if used in an italicized text such as a theorem)
\newunicodechar{⟦}{\llbracket}
%27E7 Mathematical Right White Square Bracket
\newunicodechar{⟧}{\rrbracket}
%27FC Long Rightwards Arrow from Bar
\newunicodechar{⟼}{\longmapsto}
%2AB0 Succeeds Above Single-Line Equals Sign
\newunicodechar{⪰}{\succeq}
%301A Left White Square Bracket
\newunicodechar{〚}{\textlbrackdbl}
%301B Right White Square Bracket
\newunicodechar{〛}{\textrbrackdbl}
%→ is defined by default as \textrightarrow, which is invalid in math mode. Same thing for the three other commands. Using \DeclareUnicodeCharacter instead of \newunicodechar because the latter warns about the previous definition.
%→ Rightwards Arrow
\DeclareUnicodeCharacter{2192}{\ifmmode\rightarrow\else\textrightarrow\fi}
%¬ Not Sign
\DeclareUnicodeCharacter{00AC}{\ifmmode\lnot\else\textlnot\fi}
%… Horizontal Ellipsis
\DeclareUnicodeCharacter{2026}{\ifmmode\dots\else\textellipsis\fi}
%× Multiplication Sign
\DeclareUnicodeCharacter{00D7}{\ifmmode\times\else\texttimes\fi}
%Permits to really obtain a straight quote when typing a straight quote; potentially dangerous, see https://tex.stackexchange.com/a/521999
%\catcode`\'=\active
%\DeclareUnicodeCharacter{0027}{\ifmmode{^{\prime}}\else\textquotesingle\fi}

%\newcommand{\cmark}{\ding{51}}%
%\newcommand{\xmark}{\ding{53}}
\newglossaryentry{scr}
{
    name=Social Choice Rule,
    description={Social Choice Rule is is a mapping $f:\linors^{N}\rightarrow 2^{A} \setminus \{\emptyset \}$}
}

%\DeclareAcronym{SCR}{short=SCR, long={Social Choice Rule}}

\newacronym{SCR}{SCR}{Social Choice Rule}
\title{Elicitation and explanation for voting rules}

\author{Beatrice NAPOLITANO}

\institute{l'Université Paris-Dauphine}
\doctoralschool{École Doctorale SDOSE}{543}
\specialty{Informatique}
\date{jj mois aaaa}

%% cotutelle
% \entitle{Thesis Subject in English}
% \otherinstitute{CEA Saclay}
% \logootherinstitute{logo-institute}

%\jurymember{7}{Prénom NOM}{Titre, établissement}{Président}
\jurymember{1}{Olivier CAILLOUX}{MdC, Université Paris-Dauphine}{Examinateur}
\jurymember{2}{Ayça EBRU GIRITLIGIL}{MdC, İstanbul Bilgi University}{Examinatrice}
\jurymember{3}{Vincent MOUSSEAU}{Professeur, CentraleSupélec Université Paris-Saclay\vspace{0.3cm}}{Rapporteur}
\jurymember{4}{İpek ÖZKAL SANVER}{Professeure, İstanbul Bilgi University}{Rapporteuse}
\jurymember{5}{Remzi SANVER}{Directeur de recherche, CNRS Université Paris-Dauphine\vspace{0.3cm}}{Directeur de thèse}
\jurymember{6}{Paolo VIAPPIANI}{Chargé de Recherche, Sorbonne Université\vspace{0.3cm}}{Examinateur}
% \jurymember{9}{Prénom NOM}{Titre, établissement}{Invité}
% \jurymember{10}{Prénom NOM}{Titre, établissement}{Invité}

%moins de 1000 caractères
%This is 1415
\frabstract{
  L'objectif de la thèse est le développement de procédures pouvant aider un comité (ou une société) à choisir une règle de vote appropriée. Cela implique une analyse axiomatique des règles de vote, l'explication de ces axiomes en termes non-expert peut comprendre et privilégier les méthodes d'élicitation. Les règles de vote sont des moyens formels d'agréger les préférences d'un groupe d'électeurs dans une décision collective. Quelle règle à utiliser est une question difficile et cela dépend de la conception de la justice de la société ou du comité concerné. Aucune règle ne peut être considérée comme la meilleure indépendamment du contexte. Une règle de vote peut être caractérisée par un ensemble d'axiomes, et l'analyse axiomatique est une approche bien acceptée pour étudier les propriétés satisfaites par les règles de vote. Mais le choix des axiomes à satisfaire ne devrait pas être laissé aux scientifiques. Ainsi, des méthodes pour présenter des axiomes et leurs conséquences à des utilisateurs non experts doivent être mises en œuvre. De plus, nous ne pouvons pas simplement demander au comité quels axiomes il aime, car il est souvent impossible pour les utilisateurs non experts de voir les conséquences de l'acceptation d'un ensemble d'axiomes. Par conséquent, il existe des techniques d'élicitation qui peuvent capturer des préférences subjectives informelles dans des modèles formels.
}

%moins de 1000 caractères
%This is 1208
\enabstract{
  The goal of the Ph.D. project is the development of procedures able to help a committee (or a society) choose a suitable voting rule. This involves axiomatic analysis of voting rules, explanation of those axioms in terms a non-expert can understand and preference elicitation methods. Voting rules are formal means to aggregate preferences of a group of voters into a collective decision. But which rule to use is a difficult question and it depends on the conception of justice of the concerned society or committee. No rule can be considered the best independent of the context. A voting rule can be characterized by a set of axioms, and the axiomatic analysis is a well-accepted approach to studying the properties satisfied by voting rules. But the choice of which axioms should be satisfied should not be left to scientists. Thus, methods to present axioms and their consequences to non-expert users must be implemented. Moreover, we cannot simply ask the committee which axioms it likes, because it is often impossible for non-expert users to see the consequences of accepting a set of axioms. Therefore, elicitation techniques exist that can capture informal subjective preferences into formal models.
}

\frkeywords{Intelligence Artificielle, Computational Social Choice, Théorie de la décision, Élicitation de préférence, Analyse axiomatique, Modélisation des préférences.}
\enkeywords{Artificial Intelligence, Computational Social Choice, Decision Theory, Preference Elicitation, Axiomatic Analysis, Preference Modeling.}


\makeglossaries

\begin{document}

\maketitle{}

\frontmatter

\begingroup
\hypersetup{hidelinks}

\tableofcontents

\listoffigures
\addcontentsline{toc}{chapter}{List of Figures}

\listoftables
\addcontentsline{toc}{chapter}{List of Tables}

\endgroup

\printglossary[toctitle=List of Abbreviations,title=List of Abbreviations,type=\acronymtype]

\mainmatter

\part{Introduction} 

	\chapter{Introduction}
		\section{Research Questions}
		\section{Contributions}
		\section{Organization of the Thesis}
	
	\chapter{Preliminaries}
		%\index{Social Choice!SCR}
		\section{Notation}
	
	\chapter{Literature Review}

\part{Compromise}
	\chapter{Introduction}
	\chapter{Related Work}
	\chapter{Compromising as an equal loss principle}
		%!TeX root= ../thesis.tex

\begin{abstract}
A social choice rule aggregates the preferences of a group of individuals over a set of alternatives into a collective choice. The literature admits several social choice rules whose recommendations are supposed to reflect a compromise among individuals. We observe that all these compromise rules can be better described as \emph{procedural compromises}, i.e., they impose over individuals a willingness to compromise but they do not ensure an outcome where everyone has effectively compromised. We revisit the concept of a compromise in a collective choice environment with at least three individuals having strict preferences over a finite set of alternatives. Referring to a large class of spread measures, we view the concept of compromise from an \emph{equal loss} perspective, favoring an outcome where every voter concedes as equally as possible. As such, being a compromise may fail Pareto efficiency, which we ensure by asking voters to concede as equally as possible among the Pareto efficient alternatives. We show that Condorcet consistent rules, scoring rules (except antiplurality) and Brams-Kilgour compromises (except fallback bargaining) all fail to ascertain an outcome which is a compromise. A slight restriction on acceptable spread measures suffices to extend the negative result to antiplurality and fallback bargaining.
This failure also prevails for social choice problems with two individuals: all well-known two-person social choice rules of the literature, namely, fallback bargaining, Pareto and veto rules, short listing and veto rank, fail to pick ex-post compromises. We conclude that there is a need to propose and study rules that satisfy this equal loss, or outcome oriented, notion of a compromise.
\end{abstract}

\section{Introduction}
\label{sec:introduction}
In a typical social choice problem, several individuals express their preferences over a set of alternatives and one shall be picked as the collective outcome. Although the literature admits several \acp{SCR} with different properties, there is a common understanding that collective choices must reflect “compromises”. One of the first to explicitly refer to a \ac{SCR} as a compromise is \citet{Sertel1986} introducing the \emph{majoritarian compromise}. This \ac{SCR}, further analyzed by \citet{Sertel1999}, is a rediscovery of a method suggested by James W.\ Bucklin at the beginning of the \nth{20} century \citep{Erdelyi2015}. Starting from everyone’s ideal alternative, it falls back to the voters’ second, third and more generally $k$-\emph{th} best, until one of the alternatives considered appears among the first $k$ best for a majority. \citet{Brams2001} generalize this concept and introduce a class of \acp{SCR} called $q$\emph{-approval fall-back bargaining}, where $q$ is the required quantity of support that can vary from a single voter up to unanimity. Different choices of $q$ lead to different \acp{SCR}. Considering $n$ voters, the choice of $q=1$ corresponds to the plurality rule, $q=\ceil{\frac{n}{2}}$ is closely related, but not identical, to the majoritarian compromise and $q=n$ represents a bargaining procedure called \emph{fall-back bargaining}, which has been further analyzed by \citet{Kibris2007} and \citet{Congar2012}. We will refer to these rules as \ac{BK} compromises with threshold q.

As \citet{OezkalSanver2004} discuss, the concept of compromising is mostly understood as the trade off between the number of voters supporting an alternative (i.e., the quantity of support) and the distance of that alternative from the supporters’ ideal alternative (i.e., the quality of support). This trade off, which is explicit for $q$-approval fall-back bargaining, is also the basis for several other \acp{SCR} such as the \emph{median voting rule} proposed by \citet{Bassett1999} and further analyzed by \citet{Gehrlein2003} or the \emph{Condorcet practical method} described by \citet{Nurmi1999}. \Citet{Merlin2019} identify and analyze a large class of \emph{compromise rules} that explore this trade off.

One can argue that a collective choice \emph{per se} implies a compromise. After all, except extreme cases such as dictatorships,
a \acp{SCR} operates on the principle that all voters could fall back from their ideal position. Whether all voters effectively do fall back and whether this fall is “equal among voters” is the subject of our analysis. In what follows, we will present examples where they do not, which could be considered counter to the spirit of compromising. \footnote{This objection to the compromise nomenclature was raised by Jean-François Laslier during a CNRS workshop on compromising hosted by Istanbul Bilgi University in 2018.}

Consider the following example.
\begin{example}
	\label{ex:ex1}
	Let $N$ be a set of $n ≥ 3$ voters and $A$ a set of alternatives. $\linors$ represents the set of linear orders over $A$. Consider the following preference profile $P\in \linors^{N}$,
	\begin{center}
		$
		\begin{array}{cccc}
			\mathbf{1} \quad &c&b&a\\
			\mathbf{n-1} \quad &a&b&c\\
		\end{array}\quad ,
		$
	\end{center}
	which represents one individual who prefers $c$ to $b$, $b$ to $a$, hence $c$ to $a$; and $n-1$ individuals who prefer $a$ to $b$, $b$ to $c$, hence $a$ to $c$. At $P$, all \acs{BK} compromises, except when $q=n$ (i.e. fall-back bargaining that would select $b$) and $q=1$ (which would select both $a$ and $c$), will ignore the single voter and will pick $a$ as the collective outcome.
\end{example}

As a matter of fact, almost every \ac{SCR} will ignore this “marginal minority” and choose $a$ in this situation. While this choice is defensible on the grounds of qualified majoritarianism, it is questionable whether $a$ can be qualified as a compromise. Observe that $b$ receives unanimous support when each voter falls back one step from his ideal point. The question becomes more compelling when $a$ remains the collective choice even if the ignored group is much larger.

\begin{example}
	\label{ex:ex2}
	Consider the following preference profile with $n=100$:
	\begin{center}
		$
		\begin{array}{cccc}
			\mathbf{49} \quad &c&b&a\\
			\mathbf{51} \quad &a&b&c\\
		\end{array} \quad.
		$
	\end{center}
	When $q\in \intvl{1,\frac{n}{2}-1}$, all \acs{BK} compromises pick $\{a,c\}$, and, when $q=\frac{n}{2}$ and $q=\frac{n}{2}+1$, all \acs{BK} compromises pick $\{a\}$. Again, it does not appear as a compromise as almost half of voters reach their best alternative while the remaining half have to be contented with their worst one.
\end{example}

Observe that all these \acp{SCR} impose to voters a willingness to compromise, but do not effectively ensure a compromise as the collective choice. In fact, the term “compromise” in this literature refers to procedural compromises that differ from outcome oriented compromises, a conceptual distinction that seems to be overlooked in the literature.
This can also be viewed as a distinction between ex-ante and ex-post compromises, where the profile is the source of uncertainty.

To define an ex-post compromise, we adapt a concept of equal losses that considers allocation of continuous utilities. This principle is used for bargaining \citep{Chun1988, Chun1991} and bankruptcy problems \citep{Herrero2001}. 
We introduce two definitions of compromise. In both of them, we pick a spread measure that determines how equally a given vector of numbers is distributed and we propose to make a collective choice where voters give up from their ideal points “as equally as possible”. The difference between the two is that one, called \emph{egalitarian compromise}, insists on equality at the expense of Pareto efficiency while the other, called \emph{Paretian compromise}, is constrained to pick among the Pareto efficient alternatives. We show that Pareto efficient \acp{SCR} cannot ensure egalitarian compromises under any spread measure. We prove that several well-known \acp{SCR} such as Condorcet extensions, scoring rules, $q$-approval fall-back bargaining, all fail to be Paretian compromises under any spread measure. We provide examples for which being a Paretian compromise would necessitate to pick an alternative that is, although Pareto optimal, ranked very low by all voters. Such alternative would never be picked by any of the above-mentioned \acp{SCR}. 

We conclude that the equal-loss principle appears adequate for collective choice problems with at least three individuals, when egalitarianism, in the sense of conceding equally, is a major concern. Imagine a situation where the head of a laboratory needs to decide which project to fund and she asks for the preferences of the laboratory members. The workplace harmony is extremely important and, in order to avoid conflicts, the winning project must be equally supported by all members. 

Consider now a situation with only two voters. As the vast literature on the ultimatum game \citep{Werner2014} suggests, mutual consent is hard to obtain when one individual sees injustice at the levels of mutual losses. The equal-loss principle seems to be crucial in this new scenario.

Collective choice models with two individuals can be interpreted as bargaining or arbitration problems. While the bargaining interpretation necessitates an explicitly defined disagreement outcome \citep{Kibris2007}, the arbitration interpretation \citep{Sprumont1993} remains within the classical collective choice environment with no explicit disagreement outcome. In this paper, we consider the latter interpretation. 

Arbitration rules are thoroughly discussed by \citet{Barbera2020}. As prominent examples, we have fallback bargaining proposed by \citet{Brams2001}, the veto-rank and short listing procedures analysed by \citet{Clippel2014} and the Pareto-and-veto rules analysed by \citet{Laslier2020}. These models consider discrete alternatives which are not contained by the classical \citet{Nash1950} bargaining environment with convex utilities. We make the same assumption. However, as \citet{Mariotti1998} and \citet{Nagahisa2002} illustrate, the two worlds can be interconnected, as we do for the equal-loss principle of \citet{Chun1988} and \citet{Chun1991}. The arbitration environment presents an instance where the equal-loss principle could matter and it is rather surprising to discover that most interesting \acp{SCR} used as arbitration solutions fail to be Paretian compromises. 

The rest of the paper is organized as follows. \Cref{sec:notation} presents the basic notions and notation. \Cref{sec:compromise} introduces egalitarian compromises and Paretian compromises, two concepts that turn out to be logically incompatible. \Cref{sec:more2voters} shows that with at least three individuals, many \acp{SCR} fail to pick a compromise. \Cref{sec:2voters} considers the two-individual case, showing that most \acp{SCR} of the literature fail to pick compromises. \Cref{sec:conclusion} makes some concluding remarks. 

\section{Basic notions and notation}
\label{sec:notation}
Consider a finite set $N$ of individuals with $\#N=n\geq 2$ and a finite set $A$ of alternatives with $\#A=m\geq 3$. We write $\linors$ for the set of linear orders over $A$.
A generic element $\prefi$ of $\linors$ stands for a preference of $i\in N$. This implies that, given any $x ≠ y\in A$, precisely one of $x \prefi y$ and $y\prefi x$ holds while $x \prefi x$ holds for no $x\in A.$ Moreover, $x\prefi y$ and $y\prefi z$ implies $x\prefi z$ $\forall x,y,z\in A$.

A \emph{profile} $P: N → \linors$ associates with each individual $i \in N$ a preference $P(i) = {\prefi}$. A \emph{\acl{SCR}} (\acs{SCR}) is a mapping $f:\linors^{N} \rightarrow 2^{A} \setminus \{\emptyset \}$. 

We write $r_{\prefi}(x)=\#\{y\in A \suchthat y \prefi x\}+1$ for the \emph{rank} of $x\in A$ at ${\prefi} \in \linors$. We denote by $\lambda_{\prefi}(x)=r_{\prefi}(x)-1$ the loss in terms of ranks for $i\in N$ with preference $\prefi$, when $x$ is elected instead of the best alternative
for $i$. The mapping $\lambda_P: A → \alllosses$ assigns to each $x\in A$ the loss vector $\lambda_{P}(x)=(\lambda_{\prefi}(x))_{i\in N}$ induced by the election of $x$. The double brackets denote intervals in the integers.

We are interested in measuring the spread of loss vectors. To this end, we define a \emph{spread measure} $\sigma: \alllosses → \R_{+}$ as a function that associates a spread value to every possible loss
vector. We write $\Sigma$ for the set of spread measures $\sigma$ that satisfy, for every $l\in\alllosses$, $\sigma(l)=0 ⇔ l_{i}=l_{j}$ $\forall i,j\in N$. Thus, the spread of $l$ gets its lowest value $0$ in case of perfect equality and only in this case. Note that this condition incorporates the minimal requirement to identify a spread measure and leads to the largest set of spread measures one could define. As discussed in \Cref{sec:RestrictionOnSigma}, this flexibility has the advantage of making our results more general.

Given any distinct $x,y\in A$, we say that $x$ \emph{Pareto dominates} $y$ at $P \in\linors^{N}$ (or equivalently $y$ is \emph{Pareto dominated} by $x $ at $P$) iff $x\prefi y,\forall i\in N$. We denote by
$\paretopt(P)= \set{x \in A \suchthat \forall y \in A\setminus\{x\}, \exists i \in N \suchthat x \pref_i y}$ the set of \emph{Pareto optimal} alternatives at $P$.
A \ac{SCR} $f$ is \emph{Paretian} iff $f(P)\subseteq\paretopt(P)$ $\forall P\in\linors^{N}$.

\section{Egalitarian versus Paretian compromises}
\label{sec:compromise}
\subsection{Egalitarian compromises}
\label{sec:EgCompromise}
We let $\argmin_{X}(\sigma \circ \lambda_P) = \set{x \in X \suchthat \allowbreak{}\forall y \in X: \sigma(\lambda_P(x)) ≤ \sigma(\lambda_P(y))}$ denote the minimal elements of $X \subseteq A$ according to $(\sigma \circ \lambda_{P})$. In other words, $\argmin_{X}(\sigma\circ\lambda_{P})$ denotes the alternatives in X whose loss vectors are the most equally distributed according to the spread measure $\sigma$.

In what follows, we define some classes of \acp{SCR} that we are interested in analyzing. 


\begin{definition} A \ac{SCR} $f$ is an \emph{Egalitarian Compromise} (EC) iff \[\exists \sigma \in \Sigma \suchthat \forall P \in \linors^N \text{ we have }f(P) \subseteq \musigma.\]
\end{definition}

\begin{definition} A \ac{SCR} $f$ is \emph{Egalitarian Compromise Compatible} (ECC) iff \[\exists \sigma \in \Sigma \suchthat \forall P \in \linors^N \text{ we have } f(P) \cap \musigma \neq \emptyset.\]
\end{definition}

Under a \ac{SCR} that is EC (resp., ECC), \emph{all} (resp., \emph{some}) winners are among the alternatives with most equally distributed losses. Clearly, EC is a subclass of ECC. Perhaps less obviously, being ECC (or EC) is incompatible with being Paretian. This will be deduced from the following proposition, which will also be useful to prove other theorems.% \cref{th:incompatibility}.

\begin{proposition} \label{prop:muSigmaLast}
	For $n ≥ 2, m ≥ 3$, there exists a profile $P \in \linors^N$ and an alternative $a_m$ such that $\forall i \in N$: $r_{\prefi}(a_m)=m$, and such that $\forall \sigma \in \Sigma: \musigma = \set{a_m}$; hence, $\musigma \cap \paretopt(P) = \emptyset$.
\end{proposition}
\begin{proof}
	Consider the following profile $P$:
	\begin{center}
		$
		\begin{array}{cccccc}
			\mathbf{1} \quad &a_1&a_2&\dots&a_{m-1}&a_m\\
			\mathbf{n-1} \quad &a_{\pi_{(1)}}&a_{\pi_{(2)}}&\dots&a_{\pi_{(m-1)}}&a_m\\
		\end{array}
		$ \quad,
	\end{center}
	where $\pi$ is the following permutation over $\intvl{1, m-1}$:
	\[
	\pi(i) = 
	\begin{cases}
		i+1 & \text{if } i \in \intvl{1, m-2} \\
		1 & \text{if } i = m-1
	\end{cases} \quad .
	\]
	In $P$, $a_m$ is the only alternative such that $r_{\prefi}(a_m)$ is independent of $i$; hence, $\sigma(\lambda_P(a)) > 0$, $\forall a \in A\setminus \{a_m\}$, $\forall \sigma \in \Sigma$. Thus, the set $\musigma$ consists of the sole element $a_m$, and, because $a_m$ is Pareto dominated, $\musigma \cap \paretopt(P) = \emptyset$.
\end{proof}

Our main result for \cref{sec:EgCompromise} follows easily.
\begin{theorem} \label{th:nonParetian}
	For $n\geq 2, \ m\geq3$, no Paretian \ac{SCR} is ECC.
\end{theorem}
\begin{proof}
	Proving this amounts to show that $\forall \sigma \in \Sigma, \exists P \in \linors^N \suchthat \paretopt(P) \cap \musigma = \emptyset$. Suffices to use \cref{prop:muSigmaLast}, which asserts that there exists a profile $P$ such that $\forall \sigma \in \Sigma: \musigma \cap \paretopt(P) = \emptyset$.
\end{proof}

\subsection{Paretian compromises}
Having seen the tension for a \ac{SCR} being Paretian and ECC, we investigate the consequences of inverting the order of priorities by insisting that at least some of the winning alternatives are Pareto optimal, and considering the most equally distributed loss vectors among those.

We consider two classes of \acp{SCR}. 
Observe that $\mustar$ denotes the set of Pareto optimal alternatives whose loss vectors are the most equally distributed according to the spread measure $\sigma$.

\begin{definition} A \ac{SCR} $f$ is a \emph{Paretian Compromise} PC iff \[\exists \sigma \in \Sigma \suchthat \forall P \in \linors^N \text{ we have } f(P) \subseteq \mustar.\]
\end{definition}

\begin{definition} A \ac{SCR} $f$ is \emph{Paretian Compromise Compatible} PCC iff \[\exists \sigma \in \Sigma \suchthat \forall P \in \linors^N \text{ we have } f(P) \cap \mustar \neq \emptyset.\]
\end{definition}

Again, it is clear that PC is a subclass of PCC. It will also probably come with no surprise that for a \ac{SCR}, being PC is incompatible with being ECC, as being PC requires to be Paretian, which permits to use \cref{th:nonParetian}. On the other hand, it is less immediate that being EC is incompatible with
being PCC, because being PCC does not require to be Paretian. This is however true.

\begin{theorem} \label{th:incompatibility} 
	For $n ≥ 2, m ≥ 3$, no \ac{SCR} is both EC and PCC.
\end{theorem}
\begin{proof}	
	Considering the profile $P$ of \cref{prop:muSigmaLast}, with $a_m$ the alternative mentioned there, any EC $f$ and any $\sigma \in \Sigma$, suffices to prove that $f(P) \cap {\mustar[\sigma][P]} = \emptyset$.
	
	First, from \cref{prop:muSigmaLast} we have that
	$\set{a_m} \cap \paretopt(P) = \emptyset$, hence $\set{a_m} \cap$ \break $ {\mustar[\sigma][P]} = \emptyset$. 
	
	Second, because $f$ is EC, for some $\sigmatop$, $f(P) \subseteq {\musigma[\sigmatop][P]}$. Using \cref{prop:muSigmaLast} again, we see that ${\musigma[\sigmatop][P]} = \set{a_m}$, hence $f(P) = \set{a_m}$.
	
	That $f(P) \cap {\mustar[\sigma][P]} = \emptyset$ follows from these two facts.
\end{proof}

It is interesting to note that the incompatibility is not complete, however.

\begin{remark}
	For $n ≥ 2$, $m ≥ 3$, there exist \acp{SCR} that are both ECC and PCC, such as the \ac{SCR} that selects the whole set of alternatives at every profile. However, this \ac{SCR} fails to be Paretian, as is any \ac{SCR} that is ECC.
\end{remark}


\section{Which \acp{SCR} are compromises?}
\label{sec:more2voters}
In this section we assume $n\geq 3$ and leave the analysis of $n=2$ to the
next section.

\subsection{Condorcet consistent rules}

An alternative $x\in A$ is a \emph{Condorcet winner} at $P\in \linors^N$ iff for all $y\in A \setminus \set{x} $, $\#\set{i \in N \suchthat x \prefi y} >\#\set{i \in N \suchthat y \prefi x}$. So each profile admits
either no or a unique Condorcet winner. An \ac{SCR} $f$ is \emph{Condorcet
	consistent} iff $f(P)=$ $\left\{ x\right\} $ at each $P\in \linors^N$ that
admits $x$ as the (unique) Condorcet winner.

\begin{theorem} \label{th:condorcet}
	Let $n\geq 3$ and $m\geq 3$. A Condorcet consistent \ac{SCR} $f$ is neither ECC nor PCC.
\end{theorem}
\begin{proof}
	Consider the following profile $P$, where the dots represent the sequence $a_4$ to $a_m$:
	\begin{center}
		$
		\begin{array}{cccccc}
			\mathbf{n-1} \quad &a_1&a_2&a_3&\dots\\
			\mathbf{1} \quad &a_3&a_2&\dots&a_1\\
		\end{array}
		$ \quad.
	\end{center}
	
	Consider any Condorcet consistent \ac{SCR} $f$. Thus, $f(P)=\{a_1\}$. However, $\musigma=\mustar=\{a_2\}$ $\forall \sigma \in \Sigma$, so there exists a profile $P$ such that both $f(P)\cap \musigma$ and $f(P)\cap \mustar$ are empty.
\end{proof}

Note that Condorcet consistent rules need not be Paretian so the fact that they all fail ECC does not follow from \cref{th:nonParetian}. 

\subsection{Scoring rules}
\label{sec:scoringrules}
A \emph{score vector} is an $m-$tuple $w=(w_{1},\dots,$ $w_{m})\in \intvl{0, 1}^{m}$ with $w_{1}=1$, $w_{m}=0$ and $w_{i}\geq w_{i+1}$ $\forall i\in \intvl{1, m-1}$. Given a score vector $w$, we write $s^{w}(x,P)=\sum_{i\in N}w_{r_{\prefi}(x)}$ for the score of $x\in A$ at $P \in \linors^N$. Each vector $w$ identifies a \emph{scoring rule} $f^w_n$ defined as $f^w_n(P)=\left\{ x\in A:s^{w}(x,P)\geq s^{w}(y,P) \ \forall y\in A\right\}$ for every $P \in \linors^N$.

We first show that no scoring rule is ECC, for any value of $n$ and $m$ at least 3.

\begin{theorem}\label{th:srECC}
	Let $n\geq 3$ and $m\geq 3.$ No score vector $w$ induces a scoring rule $f^w_n$ that is ECC.
\end{theorem}
\begin{proof}
	Take any score vector $w$. Consider the profile $P$ of \cref{prop:muSigmaLast}. Observe that $\musigma=\{a_m\} \ \forall \sigma \in \Sigma $. However, as $w_{1}>w_{m}$, we have $s^{w}(a_{1},P)>s^{w}(a_{m},P)$ which implies $a_{m}\notin f^{w}(P)$.
\end{proof}

We call antiplurality score vector the vector $w$ such that $w_{i} = 1, \forall i \in \intvl{1, m-1}$, and $w_{m}=0$.

\begin{theorem}
	\label{th:AntSatsPCC}
	Let $m\geq 3$ and let $w$ be the antiplurality score vector. The \ac{SCR} $f_{n}^{w}$ satisfies PCC for all $n\geq 3$.
\end{theorem}
\begin{proof}
	Define $\bar\sigma \in \Sigma$ as, $\forall l \in \intvl{0,m-1}^N$: $\bar\sigma(l) = 1$ iff $\exists i, j \in N \suchthat l_i ≠ l_j$; $\bar\sigma(l) = 0$ otherwise.
	We show the non-emptiness of $f^w_n(P) \cap \mustar[\bar\sigma]$ for any profile $P$.
	
	Let $k = \min_{\paretopt(P)} \set{(\bar\sigma \circ \lambda_P)(x)}$ be the minimal value attained by $\bar\sigma \circ \lambda_P$ over $\paretopt(P)$. By construction of $\bar\sigma$, $k$ equals either $0$ or $1$.
	
	For $k = 1$, take any $x \in f^w_n(P) \cap \paretopt(P)$. This intersection is non-empty because whenever the antiplurality rule picks a Pareto dominated alternative $z$, it also picks all alternatives which Pareto dominate $z$.
	By definition of $\bar\sigma$, $\bar\sigma(x) ≤ 1$, hence, $x \in \mustar[\bar\sigma]$.
	%	We then have that $x \in \mustar[\bar\sigma]$ as by definition of $\bar\sigma$, $\bar\sigma(x) ≤ 1$.
	
	For $k = 0$, take any $x \in \mustar[\bar\sigma]$. As $\bar\sigma (\lambda _{P}(x))=0$, we have, $\forall i, j \in N$: $\lambda_i^P(x) = \lambda_j^P(x)$, hence, $\forall i, j \in N$: $r_{\succ_i}(x) = r_{\succ_j}(x)$. 
	The case $r_{\succ_i}(x) = m, \forall i \in N$ is ruled out by $x \in \paretopt(P)$. Hence, $r_{\succ_i}(x) ≤ m - 1, \forall i \in N$, hence, $x \in f^w_n(P)$.
\end{proof}

It is worth noting that the antiplurality rule $f_{n}^{w}$ is not Paretian, hence fails PC  for all $n\geq 3$. This can be seen by picking a unanimous profile $P \in \linors^{N}$ with $a_{1}\prefi a_{2}\prefi \dots \prefi a_{m}$ $\forall i\in N$, where $\mustar=\left\{ a_{1}\right\} \forall \sigma \in \Sigma $ while $f_{n}^{w}(P)=A \setminus \left\{ a_{m}\right\}$.

\begin{theorem}
	\label{th:srPCC}
	Let $m\geq 3.$ Take any score vector $w$ which is not the antiplurality score vector. For some $n ≥ 3$, the \ac{SCR} $f_{n}^{w}$ fails PCC.
\end{theorem}

\begin{proof}
	Take any $m\geq 3$ and any score vector $w$ that is not the antiplurality score vector; therefore, $w_{m-1}<1$. Pick any $n$ such that $n ≥ m - 1$ and $n > \frac{1}{1 - w_{m - 1}}$. Consider a profile $P \in \linors^N$ conforming to
	
	\begin{center}
		$
		\begin{array}{cccccc}
			i = 1 \quad & a_2 & … & a_m & a_1\\
			2 ≤ i ≤ m - 2 \quad & a_1 & … & a_m & a_i\\
			m - 1 ≤ i ≤ n \quad & a_1 & … & a_m & a_{m-1}\\
		\end{array}
		$\quad,
	\end{center}
	where all alternatives except $a_m$ appear at least once in the last rank.
	Thus, for every $\sigma \in \Sigma$, we have 
	$\sigma (\lambda _{P}(x))>0$ $\forall x\in A \setminus \left\{ a_{m}\right\}$
	while
	$\sigma (\lambda_{P}(a_{m}))=0$. 
	Moreover, $a_{m}\in \paretopt(P)$. Thus, $\mustar=\left\{ a_{m}\right\} $ $\forall \sigma \in \Sigma $. On the other hand, $s^{w}(a_{1}; P)=n-1$, $s^{w}(a_{m}; P)=n\cdot w_{m-1}$ and
	as $n > \frac{1}{1 - w_{m - 1}}$ (or, equivalently, $n - 1 > n w_{m - 1}$), we have $s^{w}(a_{1}; P)>s^{w}(a_{m};$ $P)$,
	establishing $a_{m}\notin f^{w}(P)$, thus $f^{w}(P)\cap \mustar=\emptyset $ $\forall \sigma \in \Sigma $.
\end{proof}

\subsection{\acs{BK} compromises}
\label{sec:BKn3}
Given any $k\in \intvl{1, m}$, we write $n_{k}(x,P)=\#\{i\in
N\mid r_{\prefi}(x)\leq k\}$ for the \emph{$k$-support} that $x$ gets at $P$, that is, the number of individuals for whom the rank of alternative $x\in A$ is lower than or equal to $k$ in the profile $P \in \linors^N$.
Note that $n_{k}(x,P)\in \intvl{1, n}$ is non-decreasing on $k$ and $n_{m}(x,P)=n.$ For each $q\in \intvl{1,n}$, we define $\rho_{q}(x,P)=\min \{k\in \intvl{1,m} \suchthat n_{k}(x,P)\geq q\}$ as the minimal rank $k$ at which the $k$-support that $x$ gets at $P$ is at least $q$. We
write $\rho _{q}(P) = \min_{x \in A} \set{\rho_{q}(x, P)}$ for the minimal rank $k$ at which the $k$-support that some alternative gets at $P$ is at least $q$. A \emph{\ac{BK} compromise with threshold }$q$ is the
\ac{SCR} $f_{q}$ defined for each $P\in \linors^N$ as $f_{q}(P)=\{x\in A \suchthat n_{\rho _{q}(P)}(x,P)\geq q\}$.
We can also define a tie breaking version of the \acs{BK} compromise where among the winners only the alternatives with the greatest support are selected: $f'_{q}(P)=\{x\in A \suchthat n_{\rho _{q}(P)}(x,P)\geq n_{\rho _{q}(P)}(y,P), \forall y\in A\}$

We first consider the \acs{BK} compromise with threshold $q=n$, $f_n$, which corresponds to the rule also known as \textit{fallback bargaining} \citep{Brams2001}.

\begin{theorem}
	\label{th:FBsatsPC}
	Let $n\geq 3$ and $m\geq 3.$ The \acs{BK} compromise $f_{n}$ satisfies PC.
\end{theorem}

\begin{proof}
	Define $\bar{\sigma } \in \Sigma$ as, $\forall l \in \intvl{0,m-1}^N$: $\bar\sigma(l) = 1$ iff $\exists i, j \in N \suchthat l_i ≠ l_j$; $\bar\sigma(l) = 0$ otherwise.
	Considering any $x \in f_n(P)$, let us show that $x \in \mustar[\bar{\sigma}]$. Because $x \in f_n(P)$, $x \in \paretopt(P)$, and therefore, suffices to show that $\forall y \in \paretopt(P)$, $\bar{\sigma}(\lambda_P(y)) ≥ \bar{\sigma}(\lambda_P(x))$. Given the choice of $\bar{\sigma}$, picking any $y \in \paretopt(P)$ with $y≠x$, suffices to show that $\bar{\sigma}(\lambda_P(y)) = 1$, equivalently, that $\exists i, j \in N \suchthat r_{\prefi}(y) ≠ r_{\pref_j}(y)$. 
	Because $x \in f_n(P)$, $\rho_n(P) = \rho_n(x, P) = \max_{N} r_{\prefi}(x)$.
	It follows from $\rho_n(P) = \min_{z \in A} \set{\rho_n(z, P)}$ that $\rho_n(y, P) ≥ \rho_n(x, P)$, thus, $\exists i \in N \suchthat r_{\prefi}(y) ≥ \rho_n(P)$. 
	Also, $y \in \paretopt(P)$ implies that $\exists j \in N \suchthat r_{\pref_j}(y) < r_{\pref_j}(x)$, thus $\exists j \in N \suchthat r_{\pref_j}(y) < \rho_n(P)$. 
	Therefore, $r_{\prefi}(y) ≠ r_{\pref_j}(y)$.
\end{proof}

\begin{theorem}
	\label{th:FBfailsECC}
	Let $n\geq 3$ and $m\geq 3.$ The BK compromise $f_{n}$ fails ECC. 
\end{theorem}
\begin{proof}
	As $f_{n}$ is Paretian, the proof comes straightforward from \cref{th:nonParetian}.
\end{proof}

\begin{theorem}
	\label{th:BKthreshold}
	Let $n\geq 3$ and $m\geq 3.$ A BK compromise $f_{q}$ with threshold $q \in \intvl{1, n-1}$ is neither ECC nor PCC.
\end{theorem}
\begin{proof}
	%Take any $n\geq 3$ and $m\geq 3.$ Let $A=\left\{ a_{1},\text{ }a_{2,}...
	%\text{ }a_{m}\right\} $. Pick some $q\in \left\{ 1,...,n\right\} $ and
	%consider the BK compromise $f_{q}$. 
	Consider the following profile $P$ (also used in the proof of \cref{th:condorcet}), where the dots represent the sequence $a_4$ to $a_m$:
	\begin{center}
		$
		\begin{array}{cccccc}
			\mathbf{n-1} \quad &a_1&a_2&a_3&\dots\\
			\mathbf{1} \quad &a_3&a_2&\dots&a_1\\
		\end{array}
		$\quad.
	\end{center}
	When $q=1$ we have that $f_{1}(P)=\{a_1,a_3\}$ and when $q \in \intvl{2, n-1}$ we have that $f_{q}(P)=\{a_1\}$. Because $\sigma(\lambda_P(a_2)) = 0$ and $\sigma(\lambda_P(a_1)) > 0$ and $\sigma(\lambda_P(a_3)) > 0$, neither $\musigma$ nor $\mustar$ contain $a_1$ nor $a_3$ for any $\sigma \in \Sigma$. 

	Note that for $q \in \intvl{1, n-1}$ we have that $f'_{q}(P)=\{a_1\}$ since the tie breaking version of the \acs{BK} compromise selects the alternatives with the greatest support. This version of the rule was used in the proof for \Cref{th:BKthreshold} published by \citet{Cailloux2022}.
	%Remzi’s proof
	%Take any $n\geq 3$ and $m\geq 3.$ Let $A=\left\{ a_{1},\text{ }a_{2,}...%
	%\text{ }a_{m}\right\} $. Pick some $q\in \left\{ 1,...,n\right\} $ and
	%consider the BK compromise $f_{q}$. Consider the profile $P\in \linors^{N}$ such that 
	%$a_{1}\succ _{i}a_{2}\succ _{i}...\succ _{i}a_{m}$ $\forall i\in N\diagdown
	%\left\{ n\right\} $ and $a_{\pi (1)}\succ _{n}a_{\pi (2)}\succ _{n}...\succ
	%_{n}a_{\pi (m)}$ where $\pi $ is a bijection on $\left\{ 1,\text{ }2,...,%
	%\text{ }m\right\} $ with $\pi (1)=3$, $\pi (2)=2,\pi (3)=1$, $\pi (i)=i+1$ $%
	%\forall i\in \left\{ 4,...,\text{ }m-1\right\} $ and $\pi (m)=4 $, we have $%
	%f_{q}(P)=\left\{ a_{1}\right\} $ while $\mu _{\sigma }(P)=\mu _{\sigma
	%}^{\ast }(P)=\left\{ a_{2}\right\} $ $\forall \sigma \in \Sigma $.
\end{proof}

\subsection{Restrictions on sigma}
\label{sec:RestrictionOnSigma}
The perfect equality recognition condition we adopt for spread measures, i.e., that the spread gets its lowest value $0$ in case of perfect equality and only in this case, is very basic. Unless this condition is violated, $\Sigma$ is the largest set of spread measures we could conceive. On the other hand, it is possible to let $\Sigma$ shrink by imposing additional conditions over spread measures. Nevertheless, as the satisfaction of PC, PCC, EC, or ECC requires the existence of a spread measure, all of our negative results, namely, those expressed by Theorems \ref{th:nonParetian}, \ref{th:incompatibility}, \ref{th:condorcet}, \ref{th:srECC}, \ref{th:srPCC}, \ref{th:FBfailsECC} and \ref{th:BKthreshold} prevail when $\Sigma$ is restricted. In a similar vein, the positive results in Theorems \ref{th:AntSatsPCC} and \ref{th:FBsatsPC} risk to be lost with additional conditions over spread measures.
Indeed, this section shows that a mild restriction removes the positive results concerning the only two rules that we found to be compatible with any of our compromise concepts.

\begin{definition}
	\label{def:conditionC}
	Given any $m\geq4$ and $n\geq \max\{4,m-1\}$, we say that a spread measure $\sigma$ satisfies condition $C_{m,n}$ iff we have $\sigma(m-3, m-1, m-2, \dots, m-2) < \sigma(m-2, m-3, \dots, 1, 0, \dots, 0)$.
\end{definition}

As both vectors are $n$ dimensional, the term $m-2$ repeats $n-2$ times in the first vector and the term $0$ repeats $n-m+2$ times in the second vector.

This condition imposes a very reasonable requirement on spread measures for large values of $m$ and $n$. Asking for $\sigma(1,3,2,2)$ to be smaller than $\sigma(2,1,0,0)$ is demanding while asking for $\sigma(5,7,6,6,6,6,6)$ to be smaller than $\sigma(6,5,4,3,2,1,0)$ reflects a mild assumption. In any case, as we state below, several well-known spread measures of the literature (see \citet{Allison1978} for a comprehensive account) satisfy \cref{def:conditionC}. Letting $\bar{l} = \sum_{i=1}^{n} l_i / n$ denote the arithmetic mean of the values of $l = (l_1, …, l_n)$, we consider the following measures:

\begin{itemize}
	\item the mean absolute difference $\sigma_{mad}(l)= \frac{1}{n^2} \sum_{i=1}^{n}\sum_{j=1}^{n}|l_i-l_j|$;
	\item the average absolute deviation $\sigma_{ad}(l)= \frac{\sum_{i=1}^{n}|l_i-\bar{l}|}{n}$;
	\item the standard deviation $\sigma_{sd}(l)= \sqrt{\frac{\sum_{i=1}^{n}(l_i-\bar{l})^2}{n}}$;
	\item the Gini coefficient $\sigma_{G}(l)= \frac{\sum_{i=1}^{n}\sum_{j=1}^{n}|l_i-l_j|}{2 \cdot n \cdot \sum_{i=1}^{n} l_i}$.
\end{itemize} 

\begin{remark}
	\label{prop:spreadMeas}
	We checked experimentally that $\sigma_{mad}$, $\sigma_{ad}$, $\sigma_{sd}$ and $\sigma_{G}$ all satisfy condition $C_{m,n}$, for $m \in \intvl{4,1000}$ and $n \in \intvl{b, 1000}$ where $b = \max\{4,m-1\}$.
\end{remark}

%A \emph{spread measure} $\sigma: \alllosses → \R_{+}$ satisfies condition gamma iff  $\sigma (m-3,$ $m-1,m-2,...,$ $m-2)$ <$\sigma(m-2,$ $m-1,...1,$0, $\ 0)$.
%			(\lambda_{P}(y))$ that associates a spread value to every possible loss
%vector. We write On the other hand, $\lambda
%			^{P}(x)=(m-3,$ $m-1,m-2,...,$ $m-2)$ and $\lambda_{P}(y)=(m-2,$ $m-1,,...1,$
%			$0,$ $\ 0)$.

We write $\Sigma^{C_{m,n}} \subseteq \Sigma$ for the set of spread measures that satisfy condition $C_{m,n}$. 
\begin{theorem} 
	\label{th:3votRestriction}
	For all $m\geq 4$, $n\geq \max\{4,m-1\}$, under $\Sigma^{C_{m,n}}$,
	\begin{itemize}
		\item [1)] $f_n^{w}$ fails PCC when $w$ is the antiplurality score vector;
		\item [2)] the \acs{BK} compromise  $f_n$ fails PCC.
	\end{itemize}
\end{theorem}
\begin{proof}
	Given any $\sigma \in \Sigma^{C_{m,n}}$, let us show that there exists a profile $P$ such that $f_n^{w} \cap \mustar = \emptyset$ and $f_n \cap \mustar = \emptyset$. To that aim, consider some $x,y\in A$ and some $P\in \linors^{N}$ with $r_{\prefi[1]}(x)=m-2$, $r_{\prefi[2]}(x)=m,$ $r_{\prefi}(x)=m-1$ $\forall i\in N \setminus \left\{ 1, 2\right\}$, and $r_{\prefi}(y)=m-i$ $\forall i\in \intvl{1,m-1}$, $r_{\prefi[j]}(y)=1$ $\forall j\in \intvl{m,n}$. Moreover, for each $z\in A \setminus \left\{ x,y\right\} $, let $r_{\prefi[i]}(z)=m$ for some $i\in N$. 
	
	Note that $f_n^{w}(P) = f_{n}(P) = \set{y}$. On the other hand, $\lambda_{P}(x)=(m-3, m-1,m-2,\dots,m-2)$ and $\lambda_{P}(y)=(m-2, m-3,\dots,1,0, \dots, 0)$. As $\sigma(\lambda_{P}(x)) < \sigma(\lambda_{P}(y))$ (because $\sigma \in \Sigma^{C_{m,n}}$), we see that $y\notin \mustar$.
\end{proof}

\section{Two-voters case}
\label{sec:2voters}
In addition to \emph{fallback bargaining (FB)} \citep{Brams2001} (defined in \cref{sec:BKn3}), we consider three prominent solutions of the literature.

\emph{Pareto-and-Veto rules (PV)} \citep{Moulin1983, Abreu1991, Laslier2020} distribute a veto power of $v_1$ and $v_2$ alternatives to voters 1 and 2, respectively, with $v_1+v_2=m-1$. So, every voter $i=1,2$ (simultaneously) vetoes his worst $v_i$ alternatives. The \ac{SCR} picks all non-vetoed and Pareto optimal alternatives.

The \emph{Veto-Rank mechanism (VR)} is commonly used in the selection of arbitrators \citep{Clippel2014}. Given a list of $m$ (odd) alternatives (that are candidates to be arbitrators), each of the two voters (that are the two parties that must agree on an arbitrator) simultaneously vetoes his worst $\frac{m-1}{2}$ alternatives. The selected alternatives are the ones with the highest Borda score among the non-vetoed alternatives.

Again within the context of selecting arbitrators, \citet{Clippel2014} propose and analyze \emph{Shortlisting (SL)} where one of the two parties starts by vetoing her worst $\frac{m-1}{2}$ alternatives ($m$ being odd), and then the second party chooses her best alternative out of the remaining ones. As the outcome of the procedure depends on the party that starts, symmetry among players is ensured by defining the solution as the union of the two outcomes where one and the other party starts.


\begin{definition}
	Given any $m \geq 7$, a spread measure $\sigma \in \Sigma$ satisfies condition $D_m$ iff 
	$\sigma(\ceil{\frac{m}{2}}, \ceil{\frac{m}{2}} - 2) < \sigma(0, \ceil{\frac{m}{2}} - 1)$ and 
	$\sigma(\ceil{\frac{m}{2}} - 2, \ceil{\frac{m}{2}}) < \sigma(\ceil{\frac{m}{2}} - 1, 0)$.
\end{definition}

For $m=7$ the condition requires $\sigma(4, 2) < \sigma(0, 3)$ and $\sigma(2, 4) < \sigma(3, 0)$ which is reasonable in our context. When the value of $m$ is larger, the condition appears even more convincing. As $m$ grows, the distance between $0$ and $\ceil{\frac{m}{2}} - 1$ grows, while the distance between $\ceil{\frac{m}{2}}$ and $\ceil{\frac{m}{2}} - 2$ remains constant. Requiring, for example, the spread of $(15, 13)$ to be smaller than the spread of $(0, 14)$ is very reasonable.

We write $\Sigma^{D_{m}} \subseteq \Sigma$ for the set of spread measures that satisfy condition $D_{m}$. 

\begin{theorem} \label{th:2votPCC}
	Let $m \geq 7$. Under $\Sigma^{D_{m}}$, FB and PV fail PCC. Furthermore, when $m$ is odd, VR and SL also fail PCC.
\end{theorem}
\begin{proof}
	Take any $m \geq 7$ and any $\sigma \in \Sigma^{D_m}$. Define $\alpha = \ceil{\frac{m}{2}} - 1$ and $\beta = \ceil{\frac{m}{2}} - 2$. It follows from $\sigma \in \Sigma^{D_{m}}$, that $\sigma(\alpha + 1, \beta) < \sigma(0, \beta + 1)$ and $\sigma(\beta, \alpha + 1) < \sigma(\beta + 1, 0)$.
	For $m$ odd, note that $\alpha + \beta + 2 = m$ and consider the profile $P$ where voter $i_1$ has the preference $x \succ a_1 \succ … \succ a_\alpha \succ y \succ b_1 \succ … \succ b_\beta$ and voter $i_2$ has the preference $b_1 \succ … \succ b_\beta \succ y \succ x \succ a_1 \succ … \succ a_\alpha$. For $m$ even, note that $\alpha + \beta + 3= m$, and define the profile $P$ in the same way, except that a supplementary alternative $z$ is added at the bottom of both rankings.
	
	Note that $\sigma(\lambda_{P}(y)) = \sigma(\alpha + 1, \beta)$ and that $\sigma(\lambda_{P}(x)) = \sigma(0, \beta + 1)$. 
	Therefore, $\sigma(\lambda_{P}(y)) < \sigma(\lambda_{P}(x))$. As $y$ is not Pareto-dominated, an \ac{SCR} that uniquely picks $x$ at $P$ cannot be PCC. In a similar vein, at the profile $P'$ which is obtained by the inversion of the preferences of $i_1$ and $i_2$ at $P$, an \ac{SCR} that is PCC cannot pick $x$ uniquely.	
	
	The proof will be concluded by showing that FB, PV, and (when $m$ is odd) VR and SL all pick only $x$ at $P$ or at $P'$.
	
	We readily see that FB picks only $x$ at $P$ (and at $P'$) since $x$ is the first alternative which reaches the unanimous consent.
	For PV, let $v_{i_1} ≥ v_{i_2}$ (thus $v_{i_1} ≥ \ceil{\frac{m-1}{2}} ≥ \ceil{\frac{m-2}{2}} = \beta + 1$ and $v_{i_2} ≤ \floor{\frac{m - 1}{2}} = \ceil{\frac{m - 2}{2}} = \alpha$), and consider the profile $P$. Observe that the first voter vetoes at least $y$ and every $b_j$ ($1 ≤ j ≤ \beta$) while no voter vetoes $x$. As $x$ Pareto-dominates every $a_j$ ($1 ≤ j ≤ \alpha$), PV picks only $x$ at $P$. When $v_{i_2} ≥ v_{i_1}$, a similar reasoning yields that PV picks only $x$ at $P'$.
	
	Now let $m$ be odd.
	
	For VR, a reasoning similar to the one applied to PV yields $x$ as the unique choice at $P$: each voter vetoes her worst $\frac{m-1}{2}$ alternatives, thus $i_1$ vetoes $y$ and every $b_j$ ($1 ≤ j ≤ \beta$) and $i_2$ vetoes every $a_j$ ($1 ≤ j ≤ \alpha$). The alternative $x$ is the only non-vetoed alternative, so it is selected as the sole winner.
	
	Finally, SL also picks $x$, as it is the unique winner no matter which voter starts the veto phase. If $i_1$ starts, $y$ and every $b_j$ ($1 ≤ j ≤ \beta$) get vetoed, then $i_2$ chooses her best alternative out of the remaining ones which is $x$. If $i_2$ starts, every $a_j$ ($1 ≤ j ≤ \alpha$) get vetoed, then $i_1$ chooses her best alternative which is $x$. 
\end{proof}



\section{Concluding remarks}
\label{sec:conclusion}
We define an ex-post compromise as an outcome where individuals give up as equally as possible from their ideal points. 
With three or more individuals, several well known \acp{SCR} fail to pick ex-post compromises, under any reasonable meaning attributed to “giving up equally”. 
Our findings cover Condorcet extensions and scoring rules but also \acs{BK} compromises, which impose a willingness to compromise without ensuring a compromised outcome.
In particular we find that: no Condorcet procedure is ECC or PCC (\Cref{th:condorcet}), no scoring rule is ECC (\Cref{th:srECC}) or PCC except for the antiplurality rule (\Cref{th:AntSatsPCC,th:srPCC}), \acs{BK} compromises are neither ECC or PCC (\Cref{th:BKthreshold,th:FBfailsECC}) except for fallback bargaining that is PC (\Cref{th:FBsatsPC}).

Our impossibility results are stated for the set of spread measures $\Sigma$, but they prevail for any subset of $\Sigma$. As $\Sigma$ is the largest set of sensible spread measures, they are valid for any specific concept of equity one might pick. On the other hand, our possibility results on fallback bargaining and antiplurality are not propagated to subsets of $\Sigma$. In fact, as soon as a reasonably mild restriction on $\Sigma$ is imposed, both rules are no longer PCC (\Cref{th:3votRestriction}). With two individuals and a similar restriction on $\Sigma$, all well-known two-person SCRs of the literature, namely, fallback bargaining, Pareto and veto rules, short listing and veto rank, fail to pick ex-post compromises (\Cref{th:2votPCC}).

The exclusion of the equal-loss principle by almost all \acp{SCR} of the literature leads to ask whether the principle is uninteresting in a discrete social choice context. This seems to be the case for voting situations where the number of voters exceeds the number of candidates and usually every candidate is ranked last by at least one voter. In these cases, the main concern is about the support of alternatives rather than equality. On the other hand, two-person collective choice problems are typically interpreted as arbitration or bargaining situations where mutual consent is a critical element in reaching a solution. Thus, the equal-loss principle appears to be valid for two-person collective choice problems and our analysis raises the question of designing new discrete arbitration rules compatible with the equal-loss principle. 

We also want to mention that different notions of compromise can be conceived. \citet{Borgers1991} defines as compromises all Pareto-optimal alternatives that are not the top choice of any individual. In this context a compromise does not always exist. This is also a possible approach to adopt and we thank an anonymous reviewer for this remark.

We close by noting, as one anonymous reviewer to whom we are grateful remarked, that the tension between equity and efficiency is not new in economics. In our paper we try to cast this tension in a context where it does not seem to have been considered yet. We certainly hope that this is only the beginning of a discussion that may lead to further progress in the future. In particular, viewing a compromise through the equal loss principle can be especially interesting in richer informational settings with a status-quo point, cardinal individual preferences or a continuum of alternatives.

%\begin{acknowledgements}
%	This paper is a part of the ‘Polarization viewed from a social choice perspective’ (POSOP) research project that is carried on under the RDI program funded by Istanbul Bilgi University. We would like to thank POSOP  for the support. We also thank Jean-François Laslier who provided the inspiration and the basis for this article. Last but not least, we thank the associate editor and three anonymous reviewers for their comments and valuable suggestions.
%\end{acknowledgements}




	
\part{Incomplete Information}
	\chapter{Introduction}
	\chapter{Related Work}
	\chapter[Simultaneous Elicitation of PSR and Agent Preferences]{Simultaneous Elicitation of Scoring Rule and Agent Preferences for Robust Winner Determination}
		\chaptermark{Simultaneous Elicitation of PSR and Agent Preferences}
		%!TeX root= ../thesis.tex

\begin{abstract}

Social choice deals with the problem of determining a consensus choice from the preferences of different agents. In the classical setting, the voting rule is fixed beforehand and full information concerning the preferences of the agents is provided. This assumption of full preference information has recently been questioned by a number of researchers and several methods for eliciting the preferences of the agents have been proposed. In this paper we argue that in many situations one should consider as well the voting rule to be partially specified. Focusing on positional scoring rules, we assume that the chair, while not able to give a precise definition of the rule, is capable of answering simple questions requiring to pick a winner from a concrete profile. In addition, we assume that the agent preferences also have to be elicited. We propose a method for robust approximate winner determination and interactive elicitation based on minimax regret; we develop several strategies for choosing the questions to ask to the chair and the agents in order to converge quickly to a near-optimal alternative. Finally, we analyze these strategies in experiments where the rule and the preferences are simultaneously elicited.

\end{abstract}
%
%
%
\section{Introduction}
Aggregation of preference information is a central task in many computer systems (recommender systems, search engines, etc).
In many situations, such as in group recommender systems, the goal is to find a consensus choice;
social choice theory can provide foundations for such applications.
The traditional approach to social choice assumes that 1) the full preference orderings of the agents and 2) the social choice function are expressed beforehand. These represent two strong hypotheses.
Requiring agents to express full preference orderings can be prohibitively costly (in terms of cognitive and communication cost).
This observation has motivated several works assuming partial preference orders: 
one early work is  by \citet{Conitzer2005} who studied the complexity of communication when using different voting rules;
\citet{Konczak05} studied the computation of possible and necessary winners for various voting rules;
\citet{Xia2008} then showed that, while the identification of a necessary co-winner in scoring rules is polynomial,  the determination of possible co-winners is NP-hard;
additional complexity results were given by \citet{Walsh2007} and \citet{Pini2007}.

Since in many practical situations there would be too many possible winners but no necessary winners, several works addressed the problem of agent preferences elicitation using a variety of approaches (minimax regret, Bayesian methods, etc.) with the goal of converging to a necessary winner \citep{Naamani-Dery2015,Kalech2011,Lu2011,Pini2009,Benabbou2016,Dey2016_2}. Among those, \citet{Walsh2009} and \citet{Conitzer2009} analyzed when to stop the elicitation process.

A second concern is the ability of the chair (the person or organization supervising the voting process) to provide a precise definition of the voting rule, suggesting the relaxation of the second hypothesis. Indeed, it is often difficult for non-experts to formalize a voting rule on the basis of some generic preferences over a desired aggregation method. 
Here we provide two examples of such situations.

Consider, as a first example, a chair that is about to hire a new employee whose performances are evaluated by several experts. The members of the chair may not have a voting rule in mind at the start of the process, and might not wish to agree on a specific voting rule. However, they might be willing to answer a few questions requiring to select who should be the winner out of specific profiles. 

Consider, as a second example, the reviewing process of a conference where the best paper must be elected. The agents express their preferences on the papers they reviewed, but they are not aware of the voting rule the Program Chair will apply when aggregating them. Nonetheless, reviewers are still willing to participate in the process. Also, the PC may not have a specific voting rule in mind, and she will find it hard to provide a precise scoring vector if asked. Maybe she strongly believes that being ranked once in the first position is “much more” valuable than being ranked two times second, but does not know exactly how much more (though she can judge example cases).

In this paper, we focus on positional scoring rules with convex weights, that are a particularly common method used to aggregate rankings. 
We develop methods, based on the notion of minimax regret, for determining a robust ``winner'' under uncertainty of both the voting rule and the agent preferences.
We provide incremental elicitation methods that at each step of the elicitation question either one of the agents or the chair, and we discuss several heuristics to choose questions that quickly reduce the regret.
Answers to questions are encoded as constraints; questions to the agents are comparisons between pairs of alternatives while questions to the chair ask to select a winner out of a synthetic profile.

While some previous works have considered partially specified aggregation methods \citep{Stein1994,Llamazares2013,Viappiani2018}, we do not know of any work considering both sources of uncertainty at the same time. 
Actually, very few works altogether have considered the problem of eliciting a voting rule by asking questions to the chair. We mention the work of \citet{Cailloux2014} that assumes a different representation for the rule.
Additionally, some works address the manipulability of voting rules \citep{Elkind2012,Dey2018,Conitzer2011,Baumeister2019} and strategic behaviors \citep{Endriss2016,Lev2019,Annemieke2012}.

Our approach is evaluated on simulations with synthetic and real datasets where both the voting rule and the agent preferences are initially unknown to the system and incrementally revealed through questioning. We assume the chair to be human, thus able to answer questions about a limited number of alternatives, so we focus on small scale social choice situations. We compare the effectiveness of several questioning strategies based on the current knowledge of the rule and preferences. To summarize our contributions: 1) we provide a novel mechanism for eliciting a voting rule by translating abstract questions about weights to a choice of an alternative given a concrete profile; 2) we show that with our elicitation method it is possible to reach low regret with a reasonable number of questions; 3) we present elicitation strategies that achieve good results within reasonable computation time; 4) we show that for the class of rules considered, asking a few questions to the chair suffice to reach low regret; 5) our experiments suggest that low degree of similarity among preferences (as in impartial culture) is a more challenging setting than less varied profiles.

\section{Social choice with partial information}
\label{sec:background}
We now introduce some basic concepts.
We consider a set $A$ of $m$ alternatives (products, restaurants, public projects, job candidates, etc.) and an infinite set $\N$ of potential agents.

A {\em profile} $(\pref_j, j \in N)$ considers a finite subset of agents $N \subset \N$ and associates to each agent a preference order ${\pref_j}  \in \linors$, a linear order over the alternatives.
A profile is equivalently represented by $\profile=(v_j, j \in N)$ where $v_j(x) \in \set{1, \ldots, m}$ denotes the rank of alternative $x$ in the preference order $\pref_j$. 
A social choice function $f : \cup_{\emptyset ≠ N \subset \N, N \text{finite}} \linors^N \rightarrow \powersetz{A}$ associates to each profile a set of (tied) winners, where $\powersetz{A}$ is the powerset of $A$ excluding the empty set.
Among the many possible social choice functions, we consider convex {\em positional scoring rules (PSRs)}. A PSR $f^{\w}$ is parameterized by a \emph{scoring vector} $\w$ associating weights $w_r \in [0, 1]$ to positions, with $1 = w_1 ≥ w_2 ≥ … ≥ w_m = 0$.
Let $\alpha^{x}_r$ be the number of times that alternative $x$ was ranked in the $r$-th position.
Given $\profile$ and $\w$, an alternative $x \in A$ obtains the score
\begin{align}
	\label{eq:srule}
	s(x; \profile, \w) = \sum_{j\in N} w_{v_j(x)}
	= \sum_{r=1}^{m} \alpha^{x}_r w_r\ .
\end{align}
The winners $f^{\w}(\profile)$ are the alternatives with highest score.

An important class of PSRs is the one using convex weights \citep{Stein1994,Llamazares2016}, meaning that the difference between the weight of the first position and the weight of the second position is at least as large as the difference between the weights of the second and third positions, etc.
\begin{equation} 
	\label{eq:convexity}
	\forall r \in \{1,\ldots,m-2\}: w_r - w_{r+1} \geq w_{r+1}-w_{r+2}.
\end{equation}
The constraint above is a natural and common assumption, often used when aggregating rankings in sport competitions (such as F1 racing, alpine skiing world cup): losing ranks at the top  is more damaging than losing ranks at the bottom.
Let $\W$ denote the set of such convex weight vectors.

We consider a specific finite set of agents $N^* \subset \N$ and let $\profile^* = (\pref_j^*, j \in N^*)$ and $\w^*$ denote the profile and weight vector, unknown to us, that represent the preferences of the agents in $N^*$ and of the chair. 

At a given time, our knowledge of agent $j$'s preference is encoded by a partial order ${\ppref_j} \subseteq {\pref_j^*}$ over the alternatives, a transitive and asymmetric relation (we assume that preference information is truthful).
An incomplete profile $\pprofile = (\ppref_j, j \in N^*)$ maps each agent to a partial preference.
Let $C(\ppref_j) = \set{{\succ} \in \linors \suchthat {\ppref_j} \subseteq {\succ}}$ denote the set of possible completions of $\ppref_i$ and $C(\pprofile) = \prod_{j \in N} C(\ppref_j)$ the set of complete profiles extending $\pprofile$. Note that $\profile^* \in C(\pprofile)$.

The vector $\w^*$  is also unknown but we assume that the chair is able to specify additional preference information taking the form of linear constraints about $\w^*$. Let $\pw \subseteq \W$ denote the set of weight vectors compatible with the preferences expressed by the chair about the scoring vector.
We will show in \cref{sec:elicit} that the additional preferences we use can be elicited by showing a complete profile of a synthetic election and asking who should be elected in this case.

\section{Robust winner determination}
\label{sec:mmr}
It is desirable in an elicitation protocol such as ours to be able to stop before reaching full knowledge of the agent preferences or of the preferences of the chair about the voting rule. As, often, there are no necessary winners and too many possible winners, it is useful to declare a winner given partial information.
As a decision criterion to determine a winner, we propose to use minimax regret \citep{Savage1954}. 
This decision criterion has been used for robust optimization under data uncertainty \citep{Kouvelis1997} as well as in decision-making with uncertain utility values \citep{Salo2001,Boutilier2006}.
In particular, \citet{Lu2011} have adopted minimax regret for winner determination in social choice where
the preferences of agents are partially known, while the social choice function is known.

We consider the simultaneous presence of incomplete knowledge in agent preferences and in the weights of the PSR.
We use \emph{maximum regret} to quantify the worst-case error, and let the alternatives that minimize this quantity win, giving some robustness in face of ignorance.
Intuitively, the quality of a proposed alternative $a$ is how far $a$ is from the optimal one in the worst case, given the current knowledge.

Given $\pprofile$ and $\pw$ (that represent the current knowledge about agent preferences and the PSR),
the maximum regret is considered by assuming that an adversary can both 1) extend the partial profile $\pprofile$ into a complete profile, and 2) instantiate the weights choosing among any weight vector in $\pw$.
We formalize the notion of minimax regret in multiple steps.
First of all, $\Regret(x, \profile, \w)$
is the “regret” of selecting $x$ as a winner instead of the optimal alternative under $\profile$ and $\w$:
$$\Regret(x, \profile, \w) = \max_{y \in A} s(y; \profile,\w) - s(x; \profile, \w).$$
The {\em pairwise maximum regret} of $x$ relative to $y$ given the partial profile $\pprofile$ and the set of weights $W$ is the worst-case loss of choosing $x$ instead of $y$ under all possible realizations of the full profile {\em and} all possible instantiations of the weights:
$$\PMR(x,y;\pprofile,W) = \max_{\w \in W} \max_{\profile \in C(\pprofile)} s(y; \profile,\w) - s(x; \profile,\w).$$
The maximum regret is the worst-case loss of $x$:
\begin{align}
	\MaxR(x;\pprofile,W) & = \max_{y \in A} \PMR(x,y; \pprofile, W) = \max_{\w \in W} \max_{\profile \in C(\profile)} \Regret(x; \profile, \w).
\end{align}
$\MaxR(x;\pprofile,W)$ is the result of an adversarial selection of the complete profile $\profile \in C(\pprofile)$ and of the scoring vector $\w \in W$ that jointly maximize the loss between $x$ and the true winner under $\profile$ and $\w$.
Finally,  $\MMR(\pprofile,W) = \min_{x \in A} \MaxR(x;\pprofile,W)$ is the value of {\em minimax regret} under $\pprofile$ and $W$, obtained when recommending a {\em minimax optimal} alternative $x^*_{\pprofile, W} \in A^*_{\pprofile, W} = \argmin_{x \in A} \MaxR(x;\pprofile,W)$.
Picking as consensus choice an alternative associated with minimax regret provides a recommendation that gives worst-case guarantees. 
In cases of ties, we can return all minimax alternatives $A^*_{\pprofile, W}$ as winners or pick one of them using some tie-breaking strategy.

Observe that if $\MMR(\pprofile, W)\!=\!0$, then any $x^{*}_{\pprofile,W} \in A^*_{\pprofile, W}$ is a necessary winner: any valid completion of the profile and choice of $w \in W$ gives to $x^{*}_{\pprofile,W}$ the highest score.

We note that our notion of regret gives some cardinal meaning to the scores: instead of just being used to select winners under the corresponding PSR, their differences are considered as representing the regret of the chair.

\paragraph{Computation of minimax regret}
Given a voting rule and a partially specified profile, \citet{Xia2008} determine necessary winners by showing constructions that attempt to maximize the score difference between a proposed winner and a chosen alternative. This reasoning was later adopted by \citet{Lu2011} who used the considerations on the worst-case completions for computing the minimax regret. 

In order to compute pairwise maximum regret, and therefore minimax regret, we decompose the PMR into the contributions associated to each agent by adapting this same reasoning to our setting. The context is however more challenging due to the presence of uncertainty in the weights.

Recall that, in the computation of $s(x; \profile, \w)$, $w_{v_j(x)}$ represents the score that $x$ obtains in the ranking $v_j$ (see \cref{eq:srule}).
Since scoring rules are additively decomposable, we can consider separately the contribution of each agent to the total score. Thus, we can write the actual regret of choosing $x$ instead of $y$ as $s(y; \profile,\w) - s(x; \profile, \w) = \sum_{j \in N} w_{v_j(y)} - w_{v_j(x)}$, and we obtain \[\PMR(x,y; \pprofile, W) =  \max_{\w \in W} \sum_{j \in N} \max_{v_j \in C(\succ_j^p)} [w_{v_j(y)} - w_{v_j(x)}].\]

The following propositions show that the procedure for completing a partial profile,  proposed by \citet{Lu2011} when considering a fixed weight vector, also applies in our setting. We write $a \pprefeq_j b$ iff $a \ppref_j b \lor a = b$ and adopt the canonical notation when considering a relation as a function, writing ${\pprefeq_j}(x)$ for $\set{y \suchthat x \pprefeq_j y}$.


\begin{proposition} \label{claim:completion}
	\begin{sloppypar}
		There exists a completion $\hat{\profile} \in C(\pprofile)$ of the partial profile $\pprofile$ such that ${\PMR(x,y; \pprofile, W) = \max_{\w \in W} [ s(y; \hat{\profile}, \w) - s(x; \hat{\profile}, \w) ]}$ and such that the linear order $\hat{v}_{j}$ of each agent $j$ satisfies:
	\end{sloppypar}
	\vspace{-0.5cm}
	\begin{eqnarray}
		\label{eq:complx}
		a \pref_j x &⇔& \lnot(x \pprefeq_j a);\\
		\label{eq:comply}
		y \pref_j a &⇔& \lnot(a \pprefeq_j y) ∧ \lnot((x \pprefeq_j y) ∧ \lnot(x \pprefeq_j a)).
	\end{eqnarray}
\end{proposition}

\begin{proof}[Proof Sketch]
	Consider our knowledge $\pprefeq_j$ about the preference of the agent $j$. 
	The adversary's goal is to make the score of $y$ as high as possible and the score of $x$ as low as possible. 
	To do this, he should complete $\ppref_j$ to $\pref_j$ by placing above $x$ as many alternatives as possible; that is, all the alternatives except those that are known to be worse than $x$ (those $a$ such that $x \pprefeq_j a$); and similarly, he should put below $y$ all the alternatives he can. Two conditions must be excluded for $a$ to go below $y$. The alternatives such that $a \pprefeq_j y$ can’t be put below $y$.
	Furthermore, the first objective must take priority over the second one: when an alternative should go above $x$ according to the first objective (because $¬(x \pprefeq_j a)$), and $x$ is known to be better than $y$ (thus $x \pprefeq_j y$), then $a$ should be put above $x$ (irrespective of whether $a \pprefeq_j y$), which will move both $x$ and $y$ one rank lower than if $a$ had been put below $y$. 
	This maximizes the adversary’s interests: because the weight vector is convex, the score difference will be lower when both alternatives are ranked lower (Equation \ref{eq:convexity}), and that difference of scores is in favor of $x$ when $x \ppref_j y$, thus to be minimized from the the adversary's point of view.
\end{proof}


\begin{proposition} \label{claim:rankPMR}
	\begin{sloppypar}
	The rank of $x$ in the PMR-maximizing linear orders of agent $j$ is $\hat{v}_{j}(x) = 1+\card{A}-\card{{\pprefeq_j}(x)}$, and the rank of $y$ is $\hat{v}_{j}(y)=1+\card{{\pprefinv_j}(y)}+\card{\beta}$, where $\card{\beta} = \card{A \setminus ({\pprefeq_j}(x) \cup {\pprefinv_j}(y))}$ if $(x \pprefeq_j y)$ and $\card{\beta} = 0$ otherwise.
	\end{sloppypar}
\end{proposition}

\begin{proof}[Proof]
	The rank of $x$ is directly obtained from Eq.(\ref{eq:complx}). The rank of $y$ is obtained by complementing Eq.(\ref{eq:comply}), obtaining $a \prefeq_j y ⇔ (a \pprefeq_j y) ∨ ((x \pprefeq_j y) ∧ ¬(x \pprefeq_j a))$, and, observing that $a \pref_j y ⇔ a ≠ y ∧ a \prefeq_j y$, obtaining that $a \pref_j y$ if and only if
	\begin{equation}
		\label{eq:betteryinter}
		(a \neq y) ∧ [(a \pprefeq_j y) ∨ ((x \pprefeq_j y) ∧ ¬(x \pprefeq_j a))],
	\end{equation} 
	or equivalently, if and only if
	\begin{equation}
		\label{eq:bettery}
		%a \pref_j y ⇔ 
		(a \ppref_j y) ∨ ((x \pprefeq_j y) ∧ ¬(x \pprefeq_j a)).
	\end{equation} 
	Indeed, \eqref{eq:betteryinter} $⇒$ \eqref{eq:bettery}, and \eqref{eq:bettery} $⇒$ \eqref{eq:betteryinter} because $(x \pprefeq_j y) ∧ ¬(x \pprefeq_j a) ⇒ a ≠ y$ (as when $a = y$, $(x \pprefeq_j y)$ and $¬(x \pprefeq_j a)$ are opposite claims). Suffices now to rewrite \cref{eq:bettery} to let the two disjuncts designate disjoint sets:
	\begin{equation}
		\label{eq:betteryfinal}
		a \pref_j y ⇔ 
		(a \ppref_j y) ∨ ((x \pprefeq_j y) ∧ ¬(x \pprefeq_j a) ∧ ¬(a \ppref_j y)).
	\end{equation}
\vspace{-0.5cm}
\end{proof}


Note that in \cref{claim:rankPMR}, in the case $(x \pprefeq_j y)$, $\beta$ is the number of alternatives incomparable with both $x$ and $y$.
\begin{proposition}\label{claim:PMR}
	The $\PMR$ can be written as:
	\begin{align} 
		\PMR(x,y; \pprofile, W)  
		& = \max_{w \in W} \sum_{j \in N} w_{\hat{v}_j(y)} - w_{\hat{v}_j(x)} = \max_{w \in W} \sum_{r=1}^m (\hat{\alpha}_{r}^{y} - \hat{\alpha}_{r}^{x}) w_i,
	\end{align}
	where $\hat{\alpha}_{r}^{y}$ (resp. $\hat{\alpha}_{r}^{x}$)  is the number of times $y$ (resp. $x$) has rank $r$ in the complete profile $\hat{\profile}$ defined in \cref{claim:rankPMR}. 
\end{proposition}

\cref{claim:PMR}  shows that PMR is linear in the weights.
The pairwise max regret $\PMR(x,y;\allowbreak \pprofile,W)$ can thus be obtained by solving the following linear program defined on the variables $w_1, …, w_m$:
\begin{align}
	\max_{\w} \sum_{r=1}^m (\hat{\alpha}_{r}^{y} - \hat{\alpha}_{r}^{x}) w_{r} \quad
	\text{ s.t. } w_1 = 1 ≥ … ≥ w_m = 0, \text{\cref{eq:convexity}} \text{ and } \w \in \pw.
\end{align}
The max regret $\MaxR(x; \pprofile, W)$ is determined by computing the pairwise regret of $x$ with all other alternatives in $A$, and the recommended alternatives are the ones with least max regret. 
Observe that when the $\PMR$ of an alternative $x$ (against some other alternative $y$) exceeds the best MR value found so far, we do not need to further evaluate $x$. 
This idea can be exploited using a minimax-search tree \citep{Braziunas2012}.

\section{Interactive Elicitation} 
\label{sec:elicit}
We propose an incremental elicitation method based on minimax regret.
At each step, the system may ask a question either to one of the agents about her preferences or to the chair about the voting rule. 
The goal is to obtain relevant information to reduce minimax regret as quickly as possible.
The elicitation can be terminated either after a given number of questions, or when the minimax regret is lower than a threshold (or when it drops to zero if we wish optimality).

\paragraph{Question types}
We distinguish between questions asked to the agents and questions asked to the chair.
As {\em questions asked to the agents} we consider comparison queries relating two alternatives.
The effect of a response to a question asked to an agent is the increase in our knowledge about the agent rankings, thus augmenting the partial profile $\pprofile$. 
If agent $j$ answers a comparison query stating that alternative $a$ is preferred to $b$, then the partial order $\ppref_j$ is augmented with $a \ppref_j b$ and by transitive closure.

A bit more discussion is needed about {\em questions asked to the chair}.
Such questions aim at refining our knowledge about the scoring rule; a response gives us a constraint on the weight vector $\w$.
In particular, we want to obtain constraints of the type $w_{r} - w_{r+1} \geq \lambda (w_{r+1} - w_{r+2})$ for $r \in \{1,\ldots,m-2\}$, relating the difference between the importance of ranks $r$ and $r+1$ with the difference between ranks $r+1$ and $r+2$.

\paragraph{Building concrete questions for the chair}
Even if the chair might be considered able to answer directly such abstract questions, we want to ensure that these questions can also, in principle, be asked in a more concrete way: in terms of winners of example profiles. Such questions have clear semantics whose understanding can be assumed to be shared by the chair, contrary to abstract questions about weights. 
Moreover, this way of questioning the chair is independent of the voting rule that is being elicited; whereas questions about weights only make sense when considering PSRs.
Asking who should win in specific profiles has been used in experimental settings investigating the feeling of justice of individuals \citep{Giritligil2005}, but, to the best of our knowledge, the use of such questions to systematically guide an elicitation process about voting rules is novel. 
This is similar to favor, in decision theory, direct choice questions ("please choose either a or b") compared to, say, questioning the decision maker about the shape of her utility function. The former are considered “observable”: acts of choice are translated to preference statements \citep[Ch.\ 1]{colell_microeconomic_1995}. 

Although questioning in terms of profiles and in terms of weights is logically equivalent in our setting, there is no a priori certainty that questioning the chair using different phrasing would yield logically equivalent answers: research in experimental psychology shows that participants’ answers differ widely when changing the phrasing of preference-related questions \citep{Lichtenstein2006}. To get out of such conundrums, we need a language considered “fundamental”. Questions of the form “In this profile, who should win?” arguably provides such a natural language.

Thus, our task is to build a profile, given $\lambda$ and $r ≤ m-2$, in such a way that the set of (tied) winners picked by the chair reveals whether $w_{r} - w_{r+1} \geq \lambda (w_{r+1} - w_{r+2})$.
\begin{proposition}\label{prop:chairQuestions}
	Given a rational $\lambda = p/q > 1$ and a rank $r$ between $1$ and $m - 2$, there exists a profile $P$ such that, for any weight vector $\w \in \W$, $a \in f(P)$ iff $w_{r} - w_{r+1} ≥ \lambda (w_{r+1} - w_{r+2})$ and $b \in f(P)$ iff $w_{r} - w_{r+1} ≤ \lambda (w_{r+1} - w_{r+2})$, where $f$ is the PSR parameterized with $\w$.
\end{proposition}

\begin{proof}[Proof]
	\label{proof:chairQuestions}
	Define a linear order $>_1$ over $A$ as placing $a$ at rank $r$, $b$ at rank $r + 1$, and the remaining alternatives arbitrarily. 
	Define $>_2$ over $A$ as placing $a$ at rank $r + 2$, $b$ at rank $r + 1$, and the remaining alternatives arbitrarily.
	Define an arbitrary linear ordering $>$ over $A \setminus \set{a, b}$. 
	Define a linear order $>_3$ as placing $a$ first, $b$ second, and following the order of $>$ for the remaining positions.
	Finally, define a linear order $>_4$ as placing $b$ first, $a$ second, and following the \emph{inverse} order of $>$ for the remaining positions.
	
	Define $P$ as the profile of $3 (p + q)$ agents containing $q$ times $>_1$, $p$ times $>_2$, and $>_3$ and $>_4$ each $p + q$ times.
	As a result, $a$ obtains the following ranks: $q$ times $r$, $p$ times $r + 2$, $p + q$ times first, and $p + q$ times second. The alternative $b$ obtains the ranks $r + 1$, $2$ and $1$, each $p + q$ times. Consider any alternative $c \in A \setminus \set{a, b}$. Its score is maximal when it comes first in $>_1$, first in $>_2$ and first in $>$, by convexity of the weights. In that case, $c$ is positioned at the ranks $1$, $3$ and $m$, each $p + q$ times. 
	
	Letting $s(x)$ denote the score of $x$ at $P$, we obtain $s(a) = q w_r + p w_{r + 2} + (p + q) w_1 + (p + q) w_2$, thus, $s(a) ≥ (p + q) w_m + (p + q) w_1 + (p + q) w_2$; $s(b) = (p + q) w_{r + 1} + (p + q) w_2 + (p + q) w_1$; and, $\forall c \in A \setminus \set{a, b}$, 
	$s(c) ≤ (p + q) w_1 + (p + q) w_3 + (p + q) w_m$. It follows that $a$ or $b$ maximize $s$ (as $s(a) ≥ s(c)$). We conclude by observing that $a \in f(P) ⇔ s(a) ≥ s(b) ⇔ q w_r + p w_{r + 2} ≥ (p + q) w_{r + 1} ⇔ w_r - w_{r + 1} ≥ (p / q) (w_{r + 1} - w_{r + 2})$, and similarly for $b \in f(P)$.
\end{proof}

\begin{example}
	Suppose we want to ask the following question to the chair: $w_{2} - w_{3} ≥ 2 (w_{3} - w_{4})$. We show the profile in Figure \ref{fig:profileQstComm} to the chair and ask who should win (each column is the preference of one agent).
	Both $a$ and $b$ have scores higher than $c$ and $d$ for all convex weights, thus either $a$ or $b$ will be picked under our hypothesis; and $s(a) ≥ s(b) ⇔ w_2 + 2 w_4 ≥ 3 w_3$.
	Figure \ref{fig:profileQstCommCompact} represents the same profile using a compressed view, the numbers in bold indicating the number of agents having the preference in the corresponding column. As the proof shows, constructed profiles require only four different linear orders.
	\begin{figure}
		\centering
		\caption{Profile representing a question to the chair in extended (a) and compact (b) form.}
		\begin{subfigure}[b]{0.49\textwidth}
			\begin{center}
				$
				\begin{array}{ccccccccc}
					c&d&d&a&a&a&b&b&b\\
					a&c&c&b&b&b&a&a&a\\
					b&b&b&c&c&c&d&d&d\\
					d&a&a&d&d&d&c&c&c\\
				\end{array}
				$
			\end{center}
			\caption{}
			\label{fig:profileQstComm}
		\end{subfigure}
		\hfill
		\begin{subfigure}[b]{0.49\textwidth}
			\begin{center}
				$
				\begin{array}{cccc}
					\mathbf{1}&\mathbf{2}&\mathbf{3}&\mathbf{3} \\
					c&d&a&b\\
					a&c&b&a\\
					b&b&c&d\\
					d&a&d&c\\
				\end{array}
				$
			\end{center}
			\caption{}
			\label{fig:profileQstCommCompact}
		\end{subfigure}
	\end{figure}
\end{example}


\paragraph{Elicitation strategies}
We develop several strategies for simultaneous elicitation of agent preferences and of the PSR.
While it is of course possible to first fully elicit the agent preferences and afterwards elicit weights, we want to investigate approaches that are able to recommend winning alternatives before obtaining complete knowledge of the profile or the rule.
We define here various strategies; a strategy tells us, given the current partial knowledge $(\pprofile, W)$, which question to ask next.

The \strat{Random} strategy is used as a baseline. It first chooses equiprobably whether to question the chair or the agents. In the first case, it draws one rank in $1 ≤ r ≤ m-2$ equiprobably, takes the middle of the interval of values for $\lambda$ that are still possible considering our knowledge so far, and asks whether $w_r - w_{r+1} ≥ \lambda (w_{r+1} - w_{r+2})$.
In the second case, it draws equiprobably among the agents whose preference is not known entirely; it then draws an alternative $a$ among those involved in some incomparabilities in $\ppref_j$ and an alternative $b$ among those incomparable with $a$ in $\ppref_j$.

Let $(x^{*},\bar{y}, \bar{\profile}, \bar{\w})$ be the current solution of the minimax regret, where $x^{*}$ is the minimax optimal alternative and $\bar{y}, \bar{\profile}, \bar{\w}$ the corresponding adversarial choices. 
The \strat{Pessimistic} strategy considers a set of $n + (m-2)$ candidate questions: one per agent, and one per rank (excluding the first and the last one which are known).

The candidate questions to the agents are chosen by extending the idea of \citet{Lu2011}, that privilege learning about the relationship of $x^*$ and $\bar{y}$ to the other alternatives if possible.
Given $j \in N^*$, if $x^*$ and $\bar{y}$ are incomparable in $\ppref_j$, the candidate question concerns the pair $(x^*, \bar{y})$, otherwise, %the candidate question
it concerns the pair $(x^*, z)$ for some $z$ incomparable to $x^*$ (randomly chosen), or if none such $z$ exist, the pair $(\bar{y}, z)$ for some $z$ incomparable to $\bar{y}$, or, if both $x^*$ and $\bar{y}$ are comparable to every alternatives in $\ppref_j$, any incomparable pair is picked at random. 

The candidate questions to the chair are determined as in the Random strategy.

Once having selected $n + m - 2$ candidate questions, the Pessimistic strategy selects the one that leads to minimal regret in the worst case.
Assume that a question $q_1$ has type $t_1$ (being “chair” or “agent”), and leads to the new knowledge states $(\pprofile_1, W_1)$ if answered positively and $(\pprofile'_1, W'_1)$ if the answer is negative. 
Define \[R^{\max}_1 = \max\set{\MMR(\pprofile_1, W_1), \MMR(\pprofile'_1, W'_1)}\]
and \[R^{\min}_1 = \min\set{\MMR(\pprofile_1, W_1), \MMR(\pprofile'_1, W'_1)} \epsilon_{t} + \epsilon'_{t}.\]
The terms $\epsilon_t$ and $\epsilon'_{t}$ are real numbers associated to the type $t$ of question; these parameters are used to fine tune the choice of the question type. 
Define similarly $t_2$, $R^{\max}_2$ and $R^{\min}_2$ for  $q_2$.
Pessimistic considers question $q_1$ to be better  than $q_2$ iff $R^{\max}_1 < R^{\max}_2 \text{ or } [R^{\max}_1 = R^{\max}_2 \text{ and } R^{\min}_1 < R^{\min}_2]$.
In other words if the maximal {\em a posteriori} MMR of two questions are (approximately) equal, then it considers the (penalized) minimal MMR values.

The \strat{Extended pessimistic} strategy uses the same criterion as the pessimistic strategy, but extending it to a bigger set of candidate questions, the same as those considered by the Random strategy. These candidate questions are then evaluated using the same operator as for the Pessimistic strategy.
Extended pessimistic is applicable only to very small problem instances: its complexity is in $O(n^2 m^5)$, because we consider $O(m^2)$ questions for each agent and need for each question to compute MMR twice, whose complexity is $O(nm^3)$.

The \strat{Two phases} strategy is developed in order to investigate the effect of varying the proportion of questions of the two types, when asking all questions to the chair at the beginning or at the end.
It is parameterized by $q_c$,  the number of questions to be asked to the chair.
The \strat{Two phases-ca} variant first asks $q_c$ questions to the chair, then $k - q_c$ questions to the agents, using in both cases Pessimistic to select the specific questions; whereas the \strat{Two phases-ac} variant starts with $k - q_c$ questions to the agents, then questions the chair. 

Finally, the \strat{Elitist} strategy aims at uncovering as quickly as possible the top alternatives of all agents. 
For any agent $j$, it asks to compare an alternative currently undominated in $\ppref_j$ with one that is currently incomparable. Thus, the top alternative for $j$ will be known after having asked exactly $m-1$ questions to $j$.
After having asked $n (m-1)$ questions to the agents, it questions the chair only, using the same approach as Pessimistic.
This strategy can be expected to perform well when the chair assigns a large weight to the first rank, as compared to the other ranks. It is used to further challenge Pessimistic, which is not specifically tailored to such a situation.

\section{Empirical Evaluation} 
\label{sec:experiments}
We  performed several numerical experiments using both real data and randomly generated profiles in order to validate our approach and test the performance of our elicitation strategies.

Given a problem size $(m, n)$, a number of questions $k$ and a strategy to test, we first create an “oracle”, representing the true preferences of the agents (randomly generated or coming from real data) and the weights associated with the chair’s scoring rule (randomly generated).
We start with empty knowledge ($\pprofile = \emptyset, W = \W$) about the preference orderings of the agents and the weights of the chair. We obtain the first question to be asked using the strategy under test. We then use the oracle to answer the question and update the system's knowledge, which is thus used to obtain the next question. This is repeated until $k$ answers have been obtained, computing the resulting MMR values along the way for various values of $k$. We repeat this whole experiment a variable number of times, for a given $(m, n, k)$, and report the average resulting MMR and standard deviation \textit{sd}. The sizes of the considered scenarios are comparable to the ones used by \citet{Cailloux2014}. 

The oracle is built as follows. 
For the real preferences, we used three datasets from \href{https://www.preflib.org/}{PrefLib} \citep{PrefLib}: \textit{T Shirt} (researchers voted on tee shirt designs; $m \!=\! 11, n \!=\! 30$),
\textit{Courses}  (students voted on courses; $m \!=\! 9, n \!=\! 146$; referred to as AGH on PrefLib)
and \emph{Skate} (judges voted on skaters at the Euros Pairs Short Program; $m \!=\! 14, n \!=\! 9$).
For the synthetic datasets, we follow an Impartial Culture (IC) assumption: the linear order of each agent is drawn i.i.d.\ uniformly.
We believe IC to be a challenging situation and expect the number of questions to ask, in order to reach a certain level of regret, to decrease with less varied profiles.
To generate the scoring rule weights, we first draw $m \!-\! 1$ numbers uniformly at random (in the interval $\intvl{0,1}$ representing weight ``differences''), normalize and sort them; a sequence of convex decreasing weights is then obtained by a decumulative sum. The penalty parameters for the Pessimistic and Extended pessimistic strategies are  $\epsilon_{\text{chair}} = 1.1$, $\epsilon'_{\text{chair}} = 10^{-6}$, $\epsilon_{\text{agent}} = 1.0$ and $\epsilon'_{\text{agent}} = 0$.
\sisetup{table-number-alignment = center, table-figures-integer=2, table-figures-decimal=1, table-auto-round}

	\begin{figure}
		\centering
		\caption{Average MMR in problems of size $(5, 10)$ after $k$ questions.}
		\label{fig:smallSize}
		\begin{tikzpicture}[scale=1.2]
			\begin{axis}[
				y=8,
				xlabel=Number of Questions,
				ylabel=Avg. Regret,
				ytick={0,2,4,6,8,10},
				xtick distance=10,
				ytick distance=2,
				xtick pos=left,
				ymajorgrids=true,
				ytick style={draw=none},
				ymin=0,
				ymax=11,
				xmin=0,
				xmax=100,
				yticklabels={0,2,4,6,8,10},
				legend style={font=\scriptsize}]
				\addlegendimage{mark=*,teal,mark size=1.5}
				\addlegendimage{mark=triangle*,orange,mark size=1.5}
				\addlegendimage{mark=square*,blue,mark size=1.5}
				\addlegendimage{mark=diamond*,red,mark size=1.5}
				
				\addplot[thick, mark=*, mark size = {2}, mark indices = {35}, teal] table [x=k, y=Pes.]{data/comparison.dat};
				\addlegendentry{Pes.}
				\addplot[thick, mark=triangle*, mark size = {2}, mark indices = {45}, orange] table [x=k, y=Ex.Pes.]{data/comparison.dat};
				\addlegendentry{Ex.Pes.}
				\addplot[thick, mark=square*, mark size = {2}, mark indices = {50}, blue] table [x=k, y=Eli.]{data/comparison.dat};
				\addlegendentry{Eli.}
				\addplot[thick, mark=diamond*, mark size = {2}, mark indices = {50}, red] table [x=k, y=Rnd.]{data/comparison.dat};
				\addlegendentry{Rnd.}
				
			\end{axis}
		\end{tikzpicture}
	\end{figure}

	
	\begin{table}
		\centering
		\caption{Average MMR in problems of size $(10, 20)$ after $k$ questions assuming geometric weights.}
		\label{tab:geometricWeights}
		\begin{tabular}{S[table-figures-integer=3, table-figures-decimal=0]S[table-number-alignment = right]@{ ± }S[table-number-alignment = left, table-figures-integer=1]S[table-number-alignment = right]@{ ± }S[table-number-alignment = left, table-figures-integer=1]}
			\toprule
			{k} & {Pes.} & {sd} & {Eli.} & {sd} \\
			\midrule
			0	&20.0	&0.0	&20.0	&0.0\\
			50	&15.96	&0.54	&17.26	&0.42\\
			100	&12.48	&0.93	&15.6	&0.43\\
			150	&9.58	&1.37	&13.94	&0.76\\
			200	&7.43	&1.25	&10.95	&1.12\\
			250	&5.26	&1.52	&6.6	&0.79\\
			300	&3.47	&1.32	&6.57	&0.79\\
			\bottomrule
		\end{tabular}
	\end{table}


\paragraph{Comparison of strategies}
Our first experiment concerns small size situations.
\Cref{fig:smallSize} compares some of our strategies in the case $m = 5, n = 10$ (variations around this size yield similar conclusions), where the results are averaged over $200$ runs.
We see that asking random questions does not allow to reach a low regret level even after having asked 100 questions, whereas a low regret level ($\MMR \!=\! 1$) is reached by Pessimistic before having asked 60 questions. This also holds for other problem sizes. For instance, for $m =10$, $n = 20$ and $500$ questions, Random strategy reaches an average regret (over 20 runs) of $9.3$ (±$ 0.7$) and Pessimistic $0.5$ (±$ 0.5$).
We notice that Pessimistic performs slightly better than Extended pessimistic, showing that Pessimistic chooses candidate questions wisely; this is good news since Pessimistic is much faster: it takes on average only $16$s for a complete elicitation session (for $m = 5$, $n = 10$ and $100$ questions), while Extended pessimistic takes $50$s. Although their performance is close, Pessimistic performs systematically better in multiple runs of the experiment.

We also compared the Pessimistic strategy against Elitist in a situation specifically tailored to advantage Elitist. For that experiment specifically, instead of drawing the weights of the oracle randomly, we fix it to a “geometric” weight vector, such that $w_r - w_{r + 1} = 2(w_{r + 1} - w_{r + 2})$, for all $r ≤ m - 2$, so as to dramatically increase the importance of the weights associated to the top ranks. Even in that case, we see in \cref{tab:geometricWeights} that Pessimistic performs better than Elitist.

\paragraph{Evaluation of Pessimistic Strategy}
\label{sec:lowRegret}
Our next set of experiments evaluate the Pessimistic strategy in absolute terms. 
We first wonder how many questions should be asked in order to achieve low regret, fixed at $n / 10$: this is equivalent to the difference of score of an alternative $x$ that results from switching from a profile $P$ to a profile $P'$ where a tenth of the agents rank $x$ last instead of first.
\cref{tab:questions}, first five columns, contains the result: it displays, for each dataset, the number of questions asked to the chair ($q_{c}^{\scriptscriptstyle{\MMR} ≤ n/10}$), and the quartiles of the number of questions asked to the agents ($q_{a}^{\scriptscriptstyle{\MMR} ≤ n/10}$), averaged over $20$ runs. It is interesting to note that about twenty or thirty questions per agent on average suffice to reach a low regret in those instances. We find also noteworthy that the Pessimistic strategy chooses to ask zero questions to the chair but still achieves low regret, in most of those instances.

Another interesting measure is the average number of questions asked to the chair ($q_{c}^{\scriptscriptstyle{\MMR} = 0}$) and to the agents ($q_{a}^{\scriptscriptstyle{\MMR} = 0}$) before reaching zero regret. The results for various sizes are displayed in the last two columns of \cref{tab:questions}. Here, we see that the Pessimistic strategy does choose to question the chair when reaching low enough regret values. The m15n30 dataset did not reach zero regret in $1000$ questions.

\begin{table}
	\centering
	\caption{Questions asked by Pessimistic strategy on several datasets to reach $\frac{n}{10}$ regret, columns 4 and 5, and zero regret, last two columns.}
	\label{tab:questions}
	\scalebox{0.9}{
	\begin{tabular}{p{1.5cm}S[table-number-alignment = center, table-figures-integer=2]S[table-number-alignment = center, table-figures-integer=3]S[table-number-alignment = center, table-figures-integer=2,table-column-width=5em]p{0.5em}@{}S[table-figures-integer=2, table-figures-decimal=1]@{ | }S[table-figures-integer=2, table-figures-decimal=1]@{ | }S[table-figures-integer=2, table-figures-decimal=1]@{ ]} S[table-number-alignment = center, table-figures-integer=2, table-figures-decimal=1,table-column-width=5em]p{0.5em}@{}S[table-figures-integer=2, table-figures-decimal=1]@{ | }S[table-figures-integer=2, table-figures-decimal=1]@{ | }S[table-figures-integer=2, table-figures-decimal=1]@{ ]}}
		\toprule
		{dataset} & m & n &{$q_{c}^{\scriptscriptstyle{\MMR} ≤ n/10}$} & \multicolumn{4}{c}{$q_{a}^{\scriptscriptstyle{\MMR} ≤ n/10}$} & {$q_{c}^{ \scriptscriptstyle{\MMR} = 0}$} & \multicolumn{4}{c}{$q_{a}^{ \scriptscriptstyle{\MMR} = 0}$} \\
		\midrule
		m5n20 & 5&20&0.0&[&4.3 &4.95 & 5.84 &5.25&[& 5.36 & 6.15 & 7.21\\
		m10n20&10&20&0.0&[&13.85 & 16.1& 18.41&31.95&[&19.66 & 21.78 & 24.7\\
		m11n30&11&30&0.0&[&16.55&19.0&22.26&45.15&[&23.07&25.7&28.89\\
		tshirts&11&30&0.0&[&13.08&16.6&19.58& 43.15 &[&28.22&31.98 &35.62\\
		courses&9&146&0.0&[&6.03 &7.0 &7.0&0.0&[&6.81 & 7.0 &7.0\\
		%m9n146&9&146&0&0&[1.94 &8 &9.25&999.3&0.47&0&0&[1.94&8&9.25&999.3&0.47\\
		m14n9&14&9&5.4&[&30.3&33.45&36.65&64.05&[&37.55&40.5&44.3\\
		skate&14&9&0.0&[&11.35&11.6&12.3&0.0&[&11.5&11.8&12.75 \\
		m15n30&15&30&0.0&[&24.95&29.5&33.68 \\
			
		\bottomrule
	\end{tabular}
}
\end{table}

\Cref{fig:linearity} shows the decrease in MMR according to the number of questions asked for various problem sizes. In particular, this shows important differences between some real datasets and the problems generated using IC.
In the \textit{Skate} problem, the value $\MMR\!=\!1$ is reached after less than $100$ questions, while the IC case of the same size ($m = 14, n = 9$) requires more than $200$ questions to reach that value. This reasoning also applies to the \textit{Courses} dataset but not to the \textit{T Shirt} dataset. This can be explained by the high degree of similarity in the preference rankings of the \textit{Skate} and the \textit{Courses} problems, which helps reducing the regret faster. For example, in \textit{Skate} the top-2 alternatives are the same for all agents, and 8 out of 9 agents rank the same alternative at position 3. By contrast, in \textit{T Shirt}, the alternatives are evenly distributed in the preference rankings. 

\begin{figure}
	\caption{Average MMR (normalized by $n$) after $k$ questions with Pessimistic strategy for different datasets.}
	\label{fig:linearity}
	\centering
	\begin{tikzpicture}[scale=1.2]
		\pgfplotsset{
			every axis legend/.append style={
				at={(0.5,1.1)},
				anchor=south
			},
		}
		\begin{axis}[
			y=80,
			legend columns=3,
			xlabel=Number of Questions,
			ylabel=MMR/n,
			ytick={0,0.5,1},
			xtick distance=100,
			xtick pos=left,
			ymajorgrids=true,
			ytick style={draw=none},
			ymin=0,
			ymax=1,
			xmin=0,
			xmax=1000,
			yticklabels={0,0.5,1},
			legend style={font=\footnotesize}]
			
			\addlegendimage{mark=halfsquare right*,brown,mark size=2}
			\addlegendimage{mark=diamond*,red,mark size=2}
			\addlegendimage{mark=pentagon*,cyan,mark size=2}
			\addlegendimage{mark=halfcircle*,violet,mark size=2}
			\addlegendimage{mark=*,pink,mark size=2}
			\addlegendimage{mark=triangle*,green,mark size=2}
			\addlegendimage{mark=halfsquare left*,blue,mark size=2}
			\addlegendimage{mark=square*,teal,mark size=2}
			\addlegendimage{mark=halfsquare*,magenta,mark size=2}
			
			
			\addplot[thick, mark=halfsquare right*, mark size = {2}, mark indices = {120}, brown] table [x=k, y=5.20]{data/linearity.dat};
			\addlegendentry{m=5, n=20}
			\addplot[thick, mark=diamond*, mark size = {2}, mark indices = {150}, red] table [x=k, y=10.20]{data/linearity.dat};
			\addlegendentry{m=10, n=20}
			\addplot[thick, mark=pentagon*, mark size = {2}, mark indices = {240}, cyan] table [x=k, y=11.30]{data/linearity.dat};
			\addlegendentry{m=11, n=30}
			\addplot[thick, mark=halfcircle*, mark size = {2}, mark indices = {400}, violet] table [x=k, y=tshirts]{data/linearity.dat};
			\addlegendentry{tshirts m11n30}
			\addplot[thick, mark=*, mark size = {2}, mark indices = {400}, pink] table [x=k, y=courses]{data/linearity.dat};
			\addlegendentry{courses m9n146}
			\addplot[thick, mark=triangle*, mark size = {2}, mark indices = {400}, green] table [x=k, y=9.146]{data/linearity.dat};
			\addlegendentry{m=9, n=146}
			\addplot[thick, mark=halfsquare left*, mark size = {2}, mark indices = {200}, blue] table [x=k, y=14.9]{data/linearity.dat};
			\addlegendentry{m=14, n=9}
			\addplot[thick, mark=square*, mark size = {2}, mark indices = {60}, teal] table [x=k, y=skate]{data/linearity.dat};
			\addlegendentry{skate m14n9}
			\addplot[thick, mark=halfsquare*, mark size = {2}, mark indices = {400}, magenta] table [x=k, y=15.30]{data/linearity.dat};
			\addlegendentry{m=15, n=30}
		\end{axis}
	\end{tikzpicture}
\end{figure}

	
	\begin{table}
		\centering
		\captionsetup{type=table}
		\caption{Average MMR in problems of size $(10, 20)$ after $500$ questions, among which $q_c$ to the chair.}
		\label{tab:twoP500}
		\begin{tabular}{S[table-figures-integer=3, table-figures-decimal=0]S[table-number-alignment = right]@{ ± }S[table-number-alignment = left, table-figures-integer=1]S[table-number-alignment = right]@{ ± }S[table-number-alignment = left, table-figures-integer=1]}
			\toprule
			{$q_c$} & {2 ph.\ ca} & {sd} & {2 ph.\ ac} & {sd} \\
			\midrule		
	
			0	&	0.59	&	0.59	&	0.59	&	0.59	\\
			15	&	0.5		&	0.49	&	0.55	&	0.58	\\
			30	&	0.32	&	0.45	&	0.36	&	0.46	\\
			50	&	0.08	&	0.2 	&	0.14	&	0.27	\\
			100	&	0.39	&	0.57	&	0.23	&	0.46	\\
			200	&	2.13	&	1.59	&	2.38	&	1.23	\\
			300	&	5.8 	&	1.65	&	6.65	&	1.46	\\
			400	&	11.28	&	1.09	&	11.94	&	1.22	\\
			500	&	20.0	&	0.0	&	20.0	&	0.0	\\
			
			\bottomrule
		\end{tabular}
	\end{table}


\paragraph{Comparison with Two Phases}

The experiments so far let the strategy free to question either the chair or an agent at each step. One may wonder what is lost in terms of regret by asking different proportions of questions to the chair and the agents. Such restrictions may be useful because of (partial) unavailability of the chair, or because the estimated cognitive costs may differ sensibly. 

\Cref{tab:twoP500} shows the MMR value reached in problems of size $m = 10, n = 20$ after $500$ questions, using the Two phases strategy, in the “ca” (chair then agents) and in the “ac” (agents then chair) variants. These numbers are to be compared with the MMR value reached after 500 questions with the Pessimistic strategy (displayed in \cref{fig:linearity}), which is $0.7$; the Pessimistic strategy asks on average $13$ (± $13$) questions to the chair in this setting. 
The line $q_c = 0$, where no question is asked to the chair, suggest that it is possible to obtain a good-quality recommendation while knowing only that the voting rule is a scoring rule with convex weights, which is our basic hypothesis. However, we observe that asking no questions to the chair does not permit to reach $\MMR\!=\!0$. The strategy, indeed, obtains full knowledge of the profile after an average of 500 questions to the agents but never reaches zero.

\section{Conclusions}  
\label{sec:conclusions}
In this paper we have considered a social choice setting with partial information about the agent preferences and voting rule.
We have proposed the use of minimax regret both as a means of robust winner determination and as a guide to the process of simultaneous elicitation of preferences and voting rule.
Our experimental results suggest that regret-based elicitation is effective and allows to quickly reduce worst-case regret significantly. They also show that, in our setting, good quality (low regret) recommendations can be achieved short of having full knowledge of weights or profile.

As part of our contribution, we provide an open-source library that can be found at \url{https://github.com/oliviercailloux/minimax}, to reproduce our experiments and perform many more.


Some directions for future works include developing new elicitation strategies, considering alternative heuristics; extending the elicitation to voting rules beyond scoring rules; eliciting preferences while restraining to concrete and easy questions.


	\chapter{Majority Judgment}
		\begin{abstract}
	\ac{MJ} is a voting system where voters assign grades to candidates using an ordinal scale. The winner is the candidate with the highest majority-grade \textemdash which is the median of the grades received. This method has attracted increasing attention of french associations and political parties which have started to use \ac{MJ} for internal decisions or local elections. In particular LaPrimaire.org is a french association that uses \ac{MJ} to choose its candidate for the french presidential election. The vote is conducted in two rounds: in the first one the voters judge five candidates randomly picked; the five candidates with the highest medians pass at the second round as finalists and the voters are asked to judge them. Is the random selection of candidates a good elicitation technique? In this paper we explore the consequences of profile incompleteness and we question the elicitation of voters preferences.
\end{abstract}

\section{Introduction}
\label{sec:intro}
\acrlong{MJ} (\ac{MJ}) is a voting method proposed by \citet{Balinski2007,Balinski2011} to elect one out of $m$ candidates based on the judgments of $n$ voters. The latter express their preferences by assigning to each candidate one of the following adjectives: Excellent, Very good, Good, Average, Mediocre, Inadequate, To be rejected. Those adjectives represent a common language whose semantic is assumed to be a shared knowledge among the voters carrying thus an absolute meaning. For each candidate the median of the grades she received is computed, this is called \textit{majority-grade}. The candidate with the highest majority-grade is elected. Ties are broken by considering the majority-grade of first order: one vote associated with the majority-grade of each tied candidates is removed and their medians are recomputed. The candidate with the highest new median is elected. If there is still a tie the process is repeated until a unique winner is found. The authors describe an additional tie breaking procedure that uses the \textit{majority-gauge}, but \citet{Felsenthal2008} show that it does not always yield the same result as the iterative mechanism.

\subsection{Related work}
The idea of using the median in voting is not new, the first use can be traced back to Galton's 'middlemost' \citep{Galton1907a,Galton1907b}. More recently \citet{Bassett1999} proposed the median as a substitute for Borda's mean, advocating for its statistical robustness \textemdash which measures the sensitivity to departures from the hypothesized model.

Numerous observers described the median grade as the highest level at which a candidate obtains the support of the majority of the voters. In other words, starting for the highest grade $h$ we check if the majority of the voters assigned at least $h$ to some candidate $c$. If this is not the case, we descend in the grading scale until such level $\hat{h}$ is found where a candidate $\hat{c}$ satisfies half population. The grade $\hat{h}$ is then the median of $\hat{c}$ grades, and, since it is the first level we stopped at, it corresponds to the best possible median. This method was proposed by James W. Bucklin in the early twentieth century \citep{Hoag1926} and it was rediscovered several times in literature for example under the names of \textit{Majoritarian Compromise} \citep{Sertel1986,Sertel1999} and \textit{Fallback Bargaining} \citep{Brams2001}. Note, also, that when the number of grades is equal to two (approve, disapprove) then \ac{MJ} is reduced to Approval Voting.

\paragraph{Pros}
\begin{itemize}
	\item It maximizes voter-expressiveness;
	\item it satisfies anonymity, neutrality, unanimity, monotonicity and independence from irrelevant alternatives;
	\item it is immune to candidate cloning;
	\item truth is a dominant strategy.
\end{itemize}

\paragraph{Cons}
\begin{itemize}
	\item The monotonicity axiom holds only for a fixed population \citep{Felsenthal2008,Laslier2018}. This violates the participation criterion leading to the no show paradox \citep{Fishburn1983}: a voter can obtain a more desirable outcome if they do not participate in the election than the one they obtain by participating and voting sincerely. *\\
	* \citet{Balinski2011}(pp. 285-290) state that "in practice, the no-show paradox is simply not important."
	\item \ac{MJ} is manipulable if voters can obtain information on other voters preferences by exaggerating their grades.* \\
	* Although \citet{Bassett1999} proved that the high breakdown property of the median \textemdash sensitivity to outlying observations \textemdash makes difficult for a minority to manipulate the ranking. However, \citet{Gehrlein2003} studied that the probability of being subject to manipulation is just slightly smaller than other methods like Borda and Copeland.
	\item \ac{MJ} does not respect the majority principle.
	\begin{example}\citet{Laslier2018}\\ Consider the following profile formed by three voters $i_1, i_2, i_3$ and two candidates $x, y$. Suppose the voters assign to candidates a grade between 0 and 20.
		\begin{center}
			$
			\begin{array}{ccc}
				& x & y \\
				\mathbf{i_1} \quad &20&11\\
				\mathbf{i_2} \quad &9&0\\
				\mathbf{i_3} \quad &9&10\\
			\end{array}\quad .
			$
		\end{center}
		The median of $x$ is $9$ and the median of $y$ is $10$, thus $y$ is elected although only one voter slightly prefers $y$ to $x$. This example can be reproduced with \ac{MJ} grades and with a larger number of voters:
		\begin{center}
			$
			\begin{array}{ccc}
				& x & y \\
				\mathbf{50} \quad &\text{Excellent}&\text{Good}\\
				\mathbf{50} \quad &\text{Mediocre}&\text{To be rejected}\\
				\mathbf{1} \quad &\text{Mediocre}&\text{Average}\\
			\end{array}\quad .
			$
		\end{center}
		\label{ex:laslier}
	\end{example}
\end{itemize}
Is it worth mentioning that Balinski replied to these critics in an article written in french published on the Revue économique \citep{Balinski2019}. In particular he dismissed criticism based on toy examples that he said have little relevance in real cases. The Example \ref{ex:laslier} proposed by \citet{Laslier2018} is also studied by \citet{Balinski2011} (pp. 281), where the authors stress the fact that is very unlikely that the last voter would associate a different grade to 9 and 10. In a large electorate the distinction would be too fine to make a significance difference between 20 and 19 or 10 and 9, so they affirm that the example would more realistically translate to: 
\begin{center}
	$
	\begin{array}{ccc}
		& x & y \\
		\mathbf{50} \quad &\text{Excellent}&\text{Good}\\
		\mathbf{50} \quad &\text{Mediocre}&\text{To be rejected}\\
		\mathbf{1} \quad &\text{Average}&\text{Average}\\
	\end{array}\quad .
	$
\end{center}
And $x$ would be preferred.

\section{Where is it used?}
\ac{MJ} has being adopted by a progressively larger number of french political parties including: Le Parti Pirate, Génération(s), LaPrimaire.org, France Insoumise and La République en Marche.
%https://www.lopinion.fr/edition/politique/en-marche-teste-elections-jugement-majoritaire-mode-scrutin-tres-201884
"Mieux Voter" \citep{MV} is a french association that promotes the use of \ac{MJ} as voting method whenever a collective choice has to be selected: public administration, associations, companies. On their website it is possible to find all the citizens lists \textendash party lists that are not affiliated to any national political party \textemdash that used \ac{MJ} to rank their candidates during the local elections of 2020. In two cases, Bordeaux et Annecy, the candidate selected using \ac{MJ} was then elected as a mayor. 


\subsection{Case LaPrimaire.org}
LaPrimaire.org \citep{LaPrimaire} is a french political initiative whose goal is to select an independent candidate for the french presidential election using \ac{MJ} as voting rule. All french citizens over 18 with rights to vote can participate as candidates or voters. The association Democratech implemented the platform for the first time in 2016 in view of the 2017 presidential elections. The number of voters who participated in the election was $10676$ during the first round (with $53383$ votes) and $32685$ during the second round (with $163425$ votes). Between May and October 2021 the process will be repeated to select the candidate who will run for the 2022 presidential elections \citep{LaPrimaire2022}.

The procedure consists of several steps whose duration is defined by a calendar. In the first phase, any eligible candidate can submit her nomination to the platform and the voters can support one or multiple nominations. The candidacies that receive at least 500 supports pass to the next phase and represent the candidates for the first round of the election. In the first round each voter is asked to express her judgment, using \ac{MJ}, on five random candidates. At the end of this phase the five candidates with the highest medians are considered the finalists who qualify for the second round. In the second round each voter is asked to express her judgment, using \ac{MJ}, on all the five finalists. The candidate with the best median at the end of this phase is selected as representative for the presidential election.

It is important to mention that the participation of this candidate to the actual election is not granted. In fact, by the french law a candidate must collect at least 500 signatures of elected officials in order to participate to the presidential election. The candidate selected by the voters of LaPrimaire.org in 2016 collected only 135 signatures and did not participate in the 2017 presidential elections.  

\paragraph{Questions}
\begin{itemize}
	\item Does expressing judgment on randomly selected candidates influence the result? (If we change the questions does the result change?)
	\item Does the number of questions influence the result? (If we change the number of questions does the result change?)
	\item If yes, do these effects are mitigated by a second round?
	\item Which is the right number of questions? (Best trade-off between communication cost and optimal result.)
	\item Can we select the next question with minimax regret instead of randomly selecting a candidate?
	\item Can we say anything about the "fairness" of proposing the candidates to judge? Suppose I have strong opinions about only two candidates: one I extremely like and one I extremely dislike. There is a chance I will not be asked about those two candidates, in this case I cannot say much about the other candidates and I am also frustrated because I did not get to express my opinions.
\end{itemize}


\section{Incomplete Profile}
Consider $n$ voters and $m$ candidates and assume that a voter $i \in N$ judges only a fraction of the $m$ candidates. What is the resulting voting rule? What are its properties? Can a voter manipulate the result by judging only some candidates?


\part{Conclusion}
	\chapter{Conclusion}
		\section{Contribution}
		\section{Future works}




\backmatter

\bibliography{biblio}
\addcontentsline{toc}{chapter}{Bibliography}

%\begingroup
%\hypersetup{hidelinks}
%\printindex
%\endgroup
 
%\printglossary

\chapter{Résumé long en français}

\end{document}
