\setlength\epigraphwidth{.7\textwidth}
\setlength\epigraphrule{0pt}
\renewcommand{\textflush}{flushepinormal}
%\renewcommand{\sourceflush}{flushright}
\epigraph{\itshape Democracy is difficult, flawed and unstable. It involves barely distinguishable political parties taking part in lengthy, over-complicated and expensive decision-making processes. Trying to engage so many people with political issues seems to lead only to complexity and disagreement. So why bother?}{--- Michela Murgia, \textit{Istruzioni per diventare fascisti}}

This chapter aims at presenting the theory behind collective decision-making processes: Social Choice Theory. Starting from the history of this field, we study its goals, challenges and evolution. In particular, we focus our attention on some of the problems of this theory and describe the questions that will be explored during the rest of the dissertation.

Collective decision-making is a process in which individual preferences are aggregated to form a single group choice. This broad definition allows to consider instances that seem very different from each other. Some examples include \textit{fair allocation problems} \textemdash which address the problem of fairly dividing some resources among individuals who have different interests in such resources\textemdash \textit{matching problems} \textemdash which deal with the problem of matching individuals from two separate groups considering their preferences, e.g. students to schools, tenants to houses etc. \textemdash and \textit{judgment aggregation problems} \textemdash whose solutions try to group the beliefs of different individuals into a judgment that reflects the society as a single entity. However, despite its many applications, the main example of a collective decision-making problem is voting. 
\commentOC{I do not understand this sentence: why despite, what is “its”?}\commentBN{Is this better?: However, despite the many applications involving a process of collective decision-making, the main example is voting.}
Given a set of individuals who express their preferences over a given set of candidates, how do we select the \textit{best} candidate for the group? This is a very ancient dilemma that many brilliant minds faced during centuries and, as \citet{McLean1990} wrote, "the theory of voting is known to have been discovered three times and lost twice."

The first records of collective choice problems can be traced back to ancient Greece (ca. 500 BC), where the word democracy itself has it origins from the greek dēmokratiā: dēmos \textit{people} and kratos \textit{rule}. \commentOC{where it has it origins from…?}\commentBN{ancient Greece ?}
Many scholars, including Aristotle, have described in details the functioning of Athenian democratic institutions, however voting processes have never been formally analysed. This appears to be due to the fact that Athenian voters where asked to express their preferences over only two alternatives: the choices were often in the form of \textit{yes/no}, \textit{banish/not banish}. Today we know that most of the problems in social choice arise when at least three alternatives are considered. Exactly this reason brought Pliny the Younger, a magistrate of Ancient Rome, to write a letter to his jurist friend Titius Aristo (105 AD). In the letter, that can be found translated in \citet[Chapter 2]{McLeanUrken1995}, Pliny asks Titius advice about a problem emerged during a vote in the Senate. A consul had been found dead and his servants were suspected of his death, the Senate had to decide whether to free them, to banish them or to put them to death. For the first time they were facing a problem with more than two choices and, with it, its related issues. Pliny explains that the first option was acclaimed by the largest number of senators but not by the majority of them. Thus, he asked the Senate to consider the alternatives as three separate ones on the principle of huge disparity in punishments, and tried to arrange a series of binary votes among the three verdicts. Pliny was aware of his own attempt to manipulate the elections in favor of his top choice, freeing the slaves, but so was one of his adversary, supporter of the death penalty. The latter knew that the most lenient sentence would have won in binary votes against banishment and death taken separately, so he withdraw his proposal that could not have defeated any of the others. The supporters of death penalty then moved their support to banishment that easily won having now a strong majority. Note how by changing the way we aggregate the same set of preferences we can have completely different outcomes.

Pliny's concern shows how some desiderata of electoral protocols are not only restricted to modern society. The desire for a transparent system that was difficult to manipulate was present also in medieval elections. However, after Pliny, the democratic process did not find much room in the public discussion. The only two exceptions of abstract works on voting rules in the Middle Ages were the ones of Ramon Llull (ca. 1232-1316) and Nicolaus Cusanus (1401-1464). The lack of theoretic analysis led to overly complicated electoral process in order to make manipulation so difficult to be in practical impossible \citep{Uckelman2010}. For example, in Venice the procedure to elect the Doge (the head of the state) involved nine stages. 


%Social choice theory as a scientific discipline with sound mathematical foun-
%dations came into existence with the publication of the Ph.D. thesis of \citet{Arrow1951}, who introduced the axiomatic method into the study of ag-
%gregation methods and whose seminal Impossibility Theorem shows that any such
%method that satisfies a list of seemingly basic fairness requirements must in fact
%amount to a dictatorial rule.

TODO:
\begin{itemize}
	\item Venice ballot \cite{Wolfson1899}, Lull and Cusanus \cite{McLean1990} \citet{Uckelman2010}
	\item ca 1700 Condorcet, Borda, and Laplace; their work was entirely ignored until 1952 . ca 1800 Dodgson
	\item Arrow \cite{Arrow1951}: start of formal social choice. Duncan Black \cite{Black1958}, \citet{Arrow2002,Arrow2011}
	\item evolution to computational social choice \cite{Comsoc2016}
\end{itemize}

REF:
History \citet{McLeanUrken1995,McLean1990,Urken2004} 
Basis \citet{Gaertner2006, Taylor2005, Nitzan2009}






