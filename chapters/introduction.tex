%!TeX root=../thesis.tex
\setlength\epigraphwidth{.7\textwidth}
\setlength\epigraphrule{0pt}
\renewcommand{\textflush}{flushepinormal}
%\renewcommand{\sourceflush}{flushright}
\epigraph{\itshape Democracy is difficult, flawed and unstable. It involves barely distinguishable political parties taking part in lengthy, over-complicated and expensive decision-making processes. Trying to engage so many people with political issues seems to lead only to complexity and disagreement. So why bother?}{--- Michela Murgia, \textit{Istruzioni per diventare fascisti}\\ \scriptsize{(Trans. \textit{How to be a Fascist: A Manual} by Alex Valente)}}


Collective decision-making is a process in which individual preferences are aggregated to form a single group choice. This broad definition allows many instances, that seem very different from each other, to fit into this category.
Some examples include \textit{fair allocation problems}\textemdash which address the problem of fairly dividing some resources among individuals who have different interests in such resources, e.g. houses allocation, creation of a working schedule etc.\textemdash 
\textit{matching problems}\textemdash which deal with the problem of matching individuals from two separate groups considering their preferences, e.g. students to schools, tenants to houses etc.\textemdash 
and \textit{judgment aggregation problems}\textemdash which try to group the beliefs of different individuals into a judgment that reflects the society as a single entity.
In this thesis we will focus on another example of collective decision-making: \textit{voting}. Given a set of individuals who express their preferences over a given set of candidates, how do we select the \textit{best} candidate for the group? This is a very ancient dilemma that has been faced multiple times during centuries and that we are going to tackle throughout this manuscript.
As we shall see, the answer depends on many factors, such as the information we have at our disposal \textemdash do we know the preferences of all voters with respect to all candidates? \textemdash and the properties we want the outcome to satisfy \textemdash what do we mean by the best candidate? Who decides what is best?

An important methodology in our work is to define desirable properties we want a rule to satisfy. This allows us, on one hand, to divide aggregation rules into classes of methods satisfying the same properties, and, on the other hand, to do the inverse process: starting from an already defined method to help the decision-maker understand its properties.
The lack of information and the assistance to decision-makers are two of the themes explored in the course of this work.

This introductory chapter aims to introduce the contributions and organization of this dissertation, but also, and maybe more importantly, to introduce its motives.
We have mentioned the analysis of aggregation processes through the definition of desiderata, but this axiomatic approach is very recent and started only after the publication of the Ph.D. thesis of Kenneth \citet{Arrow1951}.
The theory behind collective decision-making processes is, however, much older and, as \citet{McLean1990} wrote, \say{the theory of voting is known to have been discovered three times and lost twice}. 

To understand what we are studying today, we think it is essential to take a look at the past.
Starting from the first traces of decision-making processes in history, we consider early voting mechanisms that focused mainly on particular cases.
From the analysis of specific instances, we move on to the start of \textit{Social Choice Theory} as a formal scientific discipline.
We then focus on how it has managed to evolve by incorporating computational aspects typical of the computer science field.
Finally, we discuss how this led us to investigate the problems that will be described in the following chapters and we outline our contributions.

\section{Voting rules from practice to theory}
The first records of collective choice problems can be traced back to ancient Greece (ca. 500 BC), and the word democracy itself originated from the greek dēmokratiā: dēmos \textit{people} and kratos \textit{rule}.
Many scholars, including Aristotle, have described in details the functioning of Athenian democratic institutions, however they never formally analyzed voting processes. This appears to be partly due to the fact that Athenian voters where asked to express their preferences over only two alternatives: the choices were often in the form of \textit{yes/no}, \textit{banish/not banish} \citep{McLeanUrken1995}. Today we know that most of the problems in social choice arise when at least three alternatives are considered.

\paragraph{From ancient Rome to the Middle Ages: the precursors of social choice.}
A problem concerning the use of three alternative is exactly the reason that brought Pliny the Younger, a magistrate of Ancient Rome, to write a letter to his jurist friend Titius Aristo (105 AD). In the letter, that can be found translated in \citet[Chapter 2]{McLeanUrken1995}, Pliny asks Titius advice about a problem that emerged during a vote in the Senate. 
A consul had been found dead and his servants were suspected of his death, the Senate had to decide whether to free them, to banish them or to put them to death. For the first time they were facing a problem with more than two choices and, with it, its related issues.
Pliny explains that the first option was acclaimed by the largest number of senators but not by the majority of them. Thus, he asked the Senate to consider the alternatives as three separate ones on the principle of huge disparity in punishments, and tried to arrange a series of binary votes among the three verdicts.
Pliny was aware of his own attempt to manipulate the elections in favor of his top choice, freeing the slaves, but so was one of his adversary, supporter of the death penalty.
The latter knew that the most lenient sentence would have won in binary votes against banishment and death taken separately, so he withdraw his proposal that could not have defeated any of the others.
The supporters of death penalty then moved their support to banishment that easily won, having now a strong majority. Note how by changing the way we aggregate the same set of preferences we can have completely different outcomes.

Pliny's concern shows how some desiderata of electoral protocols are not only restricted to modern society. The desire for a transparent system that was difficult to manipulate was present also in medieval elections. However, after Pliny, the democratic process did not find much room in the public discussion. The only two exceptions of abstract works on voting rules in the Middle Ages were the ones of Ramon Llull (ca. 1232-1316) and Nicolaus Cusanus (1401-1464). The lack of theoretic analysis led to the creation of overly complicated electoral processes in order to make manipulations so difficult as to be impossible in practice \citep{Uckelman2010}. For example, Venice's 1268 ordinance, which described the procedure to elect the Doge (the head of the state), involved nine stages from the elections of the first committee to the election of the Doge himself. The procedure, discussed in detail by \citet{Lines1986, Coggins1998, Mowbray2007} can be summarized by this folk poem \citep[p.79]{Doglioni1666}:
\vspace{1em}
\begin{paracol}{2}
	\begin{leftcolumn}
		\centering
		\textit{\say{Trenta elegge il conseglio. \\ 
			Di quei nove hanno il meglio; \\
			Questi eleggon quaranta; \\
			Ma chi di lor si vanta \\
			Son dodici che fanno \\
			Venticinque: ma stanno \\
			Di questi solo nove \\
			Che fan con le lor prove \\
			Quarantacinque a ponto \\
			De’ quali undici in conto, \\
			Eleggon quarantuno, \\
			Che chiusi tutti in uno, \\
			Con venticinque almeno \\
			Voti, fanno il sereno \\
			Principe che corregge \\
			Statuti, ordini e legge.}}
	\end{leftcolumn}
	
	\begin{rightcolumn}
		\centering
			Thirty elect the board. \\ 
			Of those, nine prevail. \\
			These elect forty.\\
			But those of them who boast \\
			are twelve, that elect \\
			twenty-five: of these \\
			only nine remain; \\
			that make with their proofs \\
			forty-five in total. \\
			Of those, eleven \\
			elect forty-one, \\
			who closed all together, \\
			with at least twenty-five \\
			votes, make the serene \\
			Prince who corrects \\
			articles, orders and laws. \\
	\end{rightcolumn}
\end{paracol}
\vspace{1em}

\citet{Wolfson1899} and \citet{Keller2014} describe more electoral processes in Italian communes. 

At the end of the thirteenth century Ramon Llull designed and discussed two voting methods in his works: \textit{Artifitium electionis personarum} [ca. 1274-83] (The method of the election of persons), \textit{En qual manera Natana fo eleta a abadessa} [ca. 1283] (In which way that Natana was elected abbess which corresponds to the 24th chapter of his novel \textit{Blaquerna}), and \textit{De arte eleccionis} [1299] (On the method of election). \cite{Hagele2001} present the original texts together with their English translation. Llull was a Catalan missionary and philosopher, and he was particularly interested in ecclesiastic elections. In the first two works we cited, Llull describes a precise schema of comparisons between each pair of candidates. The candidate winning the largest number of pairwise comparison was selected as a winner. The last method, described in \textit{De arte eleccionis}, is also based on pairwise comparisons but with a system of successive eliminations. The candidate winning the last comparison was selected as a winner. While the first two methods can be viewed as equivalent to either a Condorcet's method or a Borda count depending on the interpretation of Llull's text, the third method is a Condorcet's procedure \citep{McLean1990}. As we will see in the course of this manuscript, a candidate is the Condorcet winner if she beats every other candidate in pairwise comparisons. A voting rule is a Condorcet procedure if it elects the Condorcet winner whenever it exists.

Unfortunately, his works have not received much attention, except from Nicolas Cusanus, a German theologian and philosopher who studied Llull's \textit{De arte eleccionis}. He rejected Llull's work and proposed his own voting procedure in a chapter of his work \textit{De concordantia catholica} [ca. 1431-34] (Catholic concordance), which can be found in \citet[Chapter 4]{McLeanUrken1995}. Cusanus' main concern was to reduce manipulation and strategic voting and he recommended a Borda count method with secrete voting. We must say that there are no evidence that Cusanus was aware of the other two methods proposed by Llull, but only of the Condorcet-like one. With this in mind, we can see that although both Llull and Cusanus wanted to reduce manipulation, the methods they proposed turned out to be quite different.
This is probably due to the different contexts they were considering. Llull had in mind an election where members of an abbey had to select their own leader: voters were people who knew and trusted each other and would have lived together even after the vote. Cusanus, on the other hand, wrote \textit{De concordantia catholica} while attending the Council of Basel: voters were people that would have met only once to elect the new head of the Catholic Church and had their own interests and strategies. This underlines a very important point about voting methods: context matters. If we were to define a voting procedure, the first thing to consider is \say{to elect what?}.
\paragraph{Enlightenment and the revival of choice.}
To find new discussions on voting procedures we must skip the rest of the Middle Ages and move on to the Enlightenment. In particular, we must travel to the France of the late eighteenth century. This should come as no surprise since by that time, in France, sciences had been extremely well organized for at least a century already through the establishment of the \say{Académie royale des sciences} by Louis XIV in 1666 \citep{Demeulenaere1995}. Starting from 1753, membership to the Academy was decided through elections using plurality rule and in 1770 Jean-Charles de Borda argued that this method could lead to undesirable results. In particular, he showed how the candidate elected under plurality could lose in case of pairwise comparisons and he proposed, instead, a rank order count \textemdash what we now call Borda count which is also equivalent to the Cusanus method. Although Borda presented these observations to the Academy in 1770, elections continued to be conducted using plurality rule and his work was published many years later \citep{Borda1784}.
One of the reasons for this delay has been attributed to one of his political opponents: Nicolas de Condorcet \citep{Urken2004} \textemdash also known as Marquis of Condorcet. In fact, the latter had meanwhile become secretary of the Academy, presumably also thanks to the voting method in place, and was responsible for scheduling publications. %Borda was an ardent supporter of absolute monarchy
The year after the publication of Borda's method, the Marquis of Condorcet also proposed a voting procedure, the one we now recognize as equivalent to LLull's method \citep{Condorcet1785}.
After the revolution, the Academy was reformed into the so-called \say{Institut de France} and Napoleon appointed a commission to study the voting mechanisms to be used within it. The commission chose Borda's method but after a while Napoleon took the unilateral decision to switch back to the use of plurality.

Once again, all these works were ignored until the second half of the 20th century. One of the few noteworthy works is that of Charles Dogson, better known as Lewis Carroll. Although not aware of the studies of the Marquis of Condorcet, in 1876 he proposed a voting procedure that solves the paradox arising in the Condorcet method when cycles occur.
In his proposition the winner is the only candidate that wins every pairwise comparison (Condorcet winner). If such a candidate does not exist then for each candidate we count the number of changes in the voter preferences required for they to be the Condorcet winner. The elected candidate is the one associated to the lowest number of changes.
\paragraph{The axiomatization of social choice theory.}
This concludes a brief introduction to the start of voting theory which has so far been concerned with analyzing procedures, and their related problems, on the basis of examples. From this point on, and especially since the publication of the Ph.D. thesis of Kenneth \citet{Arrow1951}, social choice theory emerges as a scientific discipline. Arrow, in fact, was the first to introduce axiomatic analysis of aggregation methods and to characterize classes of rules on the basis of the set of properties they satisfy. In his thesis, he also defines a small set of axioms that seem naturally desirable and then he proves that the only voting rule able to satisfy them all is the dictatorship. This opened the door to the research that followed over the next sixty years.

In the meantime, however, the aforementioned works were still unknown to the general public, until a few years later, when Duncan \citet{Black1958} published the book that soon became the basis of Social Choice Theory: \textit{The Theory of Committees and Elections}. From this point onward, new axioms are recurrently proposed and new electoral rules analyzed. In \Cref{ch:preliminaries}, we will describe the most relevant ones to this work.

For reasons of space, and complexity of the topic, many authors who have worked on electoral procedures \textemdash e.g. Pufendorf, Daunou, LaPlace, Lhulier, Nanson, Galton, Scitovsky, Bergson, Samuelson etc.\textemdash have not been mentioned in this chapter. More details on the history of Social Choice Theory can be found in \citet{Black1958,McLeanUrken1995,McLean1990,Urken2004}. Some other excellent sources for approaching the study of this discipline are: \citet{Arrow2002,Arrow2011,Gaertner2006, Fishburn2015, Taylor2005, Nitzan2009}.
\paragraph{The origin of computational social choice.}
Starting from the first traces of decision-making processes we have almost reached the present days. But we are still missing something to complete the picture. All rules mentioned so far never considered the computational cost of executing these procedures.
Since Arrow, an aggregation method has been viewed as more or less reasonable on the basis of the axioms, the properties, it satisfies. However, the same rule may be proven impractical in its use \citep{Bartholdi1989,Rothe2003,Hemaspaandra2005,Davenport2004}.
Furthermore, most of the traditional rules require voters to express an order of preference over all alternatives. When the set of candidates is very large this comes at a huge cost, both from a cognitive and communicative point of view.
If we were to decide where to have dinner, how would we expect all voters to order dozens, or even hundreds, of restaurants? What if some choices are not comparable \citep{Pini2007}? 
Even assuming that voters are prepared to make such an effort, what is the cost of communicating all this information \citep{Lang2004,Conitzer2005}?
It is from all these questions (and many others such as the cost of manipulability for example) that many computer scientists have become interested in problems previously restricted to social theory.
The first time the term \textit{Computational Social Choice} was used to refer to this specific area of research was in 2006, during the first edition of the COMSOC workshop (Amsterdam).
\textit{\say{The Handbook of Computational Social Choice}} brings together the main notions of all major research sub-areas related to this new field \citep{Comsoc2016}.

\section{Research questions and contributions}
Studying the history of aggregation methods, we realized that starting from a wide spectrum of procedures we can identify classes whose elements satisfy common properties. Given a method we can analyze the axioms it satisfies.
The power of this is that we could also go the other way round: given some preferences about the properties that the rule must respect, we can help the decision-maker to define their desired aggregation method. 
One of the points we will focus on is precisely this: 
\begin{itemize}
	\item how can we help the decision-maker formally define an electoral rule on the basis of generic preferences about its properties?
\end{itemize}
In \Cref{ch:minimax} we will study a situation in which a committee of non-experts has to decide how to aggregate the preferences of voters. 
Assume that the committee assigns a score to each position in the preference order of the voters. So, for example, the first ranked gets 10 points, the second 5 etc.
Imagine now that the committee wishes the score of the best choice to be \say{much higher} than the one of the third best choice.
How do we translate this \say{much higher} into a voting rule? What exactly does it mean?
In this thesis we will focus on a particular class of rules, \textit{Positional Scoring Rules} \textemdash which we will describe in detail in \Cref{ch:preliminaries} \textemdash and develop a method for asking questions to the committee in terms of picking winners in an example profile. 
So, we transform a complex question involving differences of weights between consecutive positions into questions more easily answerable by a human committee such as "who should win in this profile"? From the answer to this question we deduce the answer to the original question.

Note how an important concept that we encountered when discussing Llull and Cusanus returns here. Both wanted to solve the same problem by proposing two different methods. This is because they were considering the problem within different contexts.
Context is extremely important when defining a voting rule; there is no best rule that we can choose to apply in every case. That is why it makes sense to ask questions to the committee and really understand what it means in their context for an alternative to be ranked first rather than third. The answers could vary a lot depending on what kind of preferences are being aggregated and what is being elected. This idea will come back later when we will discuss \Cref{ch:compromise}.

Another important problem \textemdash which we mentioned introducing the field of Computational Social Choice\textemdash when deciding on a voting rule is its cost. This is a point that we also find in our first research question. 
In this case it is represented by the cognitive cost for non-experts to formalize a voting rule on the basis of some generic preferences. But, as we know, the cost is also related to voters when they have to order a large set of preferences, and computational in transmitting all this information.
This awareness leads us to study strategies of voters preferences elicitation:
\begin{itemize}
	\item how can we acquire the most relevant information at the lowest cost?
\end{itemize}
We study this question for different voting rules (Positional Scoring Rules and Majority Judgment) in \Cref{ch:minimax,ch:MJ}. In particular, in \Cref{ch:minimax}, we combine this research question with the first one and we try to understand whether, in a context of zero information, interleaving the elicitation of voter preferences and the elicitation of the voting rule yields better result than a more linear approach. 
In \Cref{ch:MJ} we will study concrete scenarios taken from real voting situations, in which voter preferences are obtained by asking a given number of random questions to voters. Among other things, we will investigate how to identify the minimum number of questions to be asked so that the probability of eliciting the winner, using the Majority Judgment rule, is high.

Coming back to the idea \say{different contexts, different definitions} we began to wonder how different situations could justify the use of different voting rules that seek to achieve the same goal.
In the literature, various notions of compromise have been proposed, and with them various voting rules that more or less attempt to implement these definitions.
When we talk about compromise what we mean depends very much on the context, when we want to decide where to dine we are probably fine with voters trying to find a middle ground even if the outcome does not turn out to be a compromise. This is the case with some voting rules that we will discuss later on, such as Fallback Bargaining, where it may happen that the outcome is the best choice for some voters but highly disliked by many others.
However, when we deal with contexts where egalitarianism \textemdash which we will interpret as everyone conceding equally\textemdash is a major concern, this notion of compromise is not acceptable anymore.
On these bases, we develop the grounds for our concept of compromise:
\begin{itemize}
	\item how can we define the notion of compromise where each voter concedes as equally as possible?
\end{itemize}
In \Cref{ch:compromise} we address this question and analyze which of the existing rules, if any, satisfy this new definition. We will see that different definitions make sense in different contexts and that one is not necessarily better than the other.

\section{Organization of the Thesis}

In what follows we will describe how this manuscript is organized. 

\Cref{ch:preliminaries} collects the important notions that are the basis of the themes discussed in this thesis and that will return recurrently in subsequent chapters. We will describe the difference between voting with different types of ballots: ranked ballots, which we will use in \Cref{ch:minimax,ch:compromise}, and rated ballots, which we will consider in \Cref{ch:MJ}. We will describe the voting rules we will use in our contributions and some of the axioms that characterize them.

In \Cref{ch:literature}, using the same approach used in this introduction, we will dig into the past to position our work. In particular, in \Cref{sec:litCMP}, starting from the meaning of compromise we will trace its use throughout history. We will analyze the definitions that have been given of compromising rules in social choice theory and point out how some of them do not represent certain ideas of compromise. We will show how our approach fits within this literature. 
Noting that the Majority Judgment rule is seen as a form of compromise in the literature, we will delve into this rule in \Cref{sec:litMJ}. We will study its introduction, its uses and also its criticisms.
As the last topic of this chapter, we cover in \Cref{sec:litMNMX} another important aspect of this thesis: Preference Elicitation. In fact, if so far it has been assumed to know the preferences of voters and the voting procedure, this cannot be taken for granted. That of preference elicitation is a well studied problem and we will describe the different approaches by which it has been tackled in the literature. In addition, we will mention how it has been addressed in different fields, particularly in the decision aiding and machine learning literature, and the similarities with our approach.

\Cref{part:contributions} includes our contributions. Specifically, \Cref{ch:minimax} is an article published in an international conference \citep{Napolitano2021} and \Cref{ch:compromise} is an article published in a journal \citep{Cailloux2022}. \Cref{ch:uml} accompanies \Cref{ch:minimax} by describing the code provided in support of the contribution. 
Finally, \Cref{ch:MJ} is an original contribution, not yet published, whose goal is to study the consequences of the elicitation process in a voting situation that uses Majority Judgment.

To conclude, \Cref{ch:conclusion} provides a summary of the contribution of this thesis and some perspectives for future directions.
