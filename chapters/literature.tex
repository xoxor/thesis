%!TeX root= ../thesis.tex

One way of approaching a concept is to reflect on the etymology of the word itself. Compromise, from the Latin \textit{compromissus}, past participle of \textit{compromittere}: com \textit{together} and promittere \textit{to promise}.
The notion of compromise can indeed be found in the ancient Rome, where two disputing parties who wished to submit their contention to arbitration, would appoint a third party (an arbitrator) and make the mutual promise to abide by the latter's unquestionable judgment. If either party broke this promise, it would have had to pay a penalty.
In a fragment of a letter written by Proculus, which was reported by \citet[p.529]{Zimmermann1996}, it is clear that the two disputants agreed to give the arbitrator unlimited powers in his decision. No appeal was possible against the final award, which was binding no matter how unjust and unequal it was.

This idea of compromise as a mere arbitration process seems very different from the notion we have of it today. Let us consider the definitions of the two verbs in a modern dictionary: to arbitrate "to settle a dispute between two people or groups after hearing the arguments and opinions of both" \citep{Arbitration}; to compromise "to come to agreement by mutual concession" \citep{Compromise}.
The mutual concession is the element that immediately makes us think of a compromise and it is precisely the missing factor in the description of the Roman compromise and modern arbitration.

In all fairness, however, Roman society is not the first one to use compromise; we find references to the use of arbitration and the selection of arbitrators also in Aristotle’s Constitution of the Athenians [53.2–4]. %considered to be written between 328 BC and 322 BC
%Official arbiters were appointed from among men fifty-nine years of age 
Moreover, Aristotle is the first to associate the concept of fairness with that of arbitration:

\textit{\say{And it is fair to want to go into arbitration rather than to court; for the arbitrator sees what is fair, but the jury looks to the law, and for this reason arbitrators have been invented, that fairness may prevail.}}\citet{OnRhetoric}[1.13.19]

\textit{\say{Fairness, for example, seems to be just; but fairness is justice that goes beyond the written law.}}\citet{OnRhetoric}[1.13.13]



--------------

compromise as a noun vs compromise as a verb (outcome or process)

compromise, a general technique for eliciting mutual consent  COMPROMISE AND LITIGATION MARTIN SHAPIRO \cite{Shapiro1979}

King Solomon’s story
Machiavelli N (1532) (1961) The Prince

\cite{Luban1985}
In a simple and well-known example, Rich and Poor are offered $1000$ to divide as they will, provided that they agree on a division. Rich offers Poor $100$, threatening to walk if Poor doesn't accept; since Rich can afford to make good on the threat, and Poor needs the money, Poor accepts. On the first, minimal, criterion listed above, this is a successful negotiation because it produced an agreement. Indeed, it may be successful on the "equal satisfaction" criterion as well, if $100$ gives Poor the same level of satisfaction that $900$ gives Rich. On others, it is not. If simple equality is an important social value, only a $500/500$ split would be satisfactory; if public policy is strongly redistributive, or if we are (or should be) committed to the Biblical and Marxist principle "to each according to his need," a $50/950$ split may be a successful outcome. Depending on the notion of success at work, different empirical conditions would likely have to be met for negotiation to succeed. These differences, moreover, concern only the outcome of the negotiation. When we look at the process as well, we may be even less inclined to consider the $900/1000$ split satisfactory: after all, it resulted from sheer economic bullying on the part of Rich. Most of us, indeed, are likely
to consider the $900/1000$ outcome to this negotiation outrageous. Nevertheless, we often find social scientific work presupposing the least ambitious criterion of success: A negotiation is successful when it produces an agreement (even if the agreement is lopsided)


Nash describes his solution as "the amount of satisfaction each individual should expect to get from the situation" (p. 155), and thus as a
"fair bargain" (p. I58).
For criticisms of the Nash solution, see Amartya K. Sen, Collective Choice and Social
Welfare (San Francisco, CA: Holden Day, I970), pp. 121-22;

----
\cite{Day1989}
An uncompromising person is one who is not disposed to make any concessions, so (it is erroneously inferred) a compromising person is one who is pliant and disposed to make concessions—regardless of whether he receives any concession in return. However this may be, one must insist, as it is generally agreed, that compromise necessarily involves mutual concessions. 'It is simply a bad joke to speak of someone's being a party to a compromise when she has got nothing out of it'. \cite{Braybrooke1982}
---

\cite{Braybrooke1982}
Golding, Benditt, and Kuflik, especially the first two, somewhat confuse the
issues by attaching too much weight to a supposed distinction between "com-
promise" in the sense of a result and "compromise" in the sense of a process and
then, in regard to the latter, inclining to confine the term to morally acceptable
instances

Carens mentions empirical grounds for thinking that if the parties to a process of negoti
ation adopt a "problem-solving" orientation-in which, without abating the
pursuit of their own interests, they recognize a shared problem and offer coopera-
 tion in solving it-the chances of reaching a result satisfactory to all parties
increase

 The noun "compromise" covers every solution in which more than one party
gets something, but something less than she aimed at and was available as a
possible share (other claimants aside) of the goods sought. Some of them may be
very unfair solutions, in that one party gets almost all that she aimed at, while the
others get almost nothing-still something, but far less than they are entitled to.
Only, however, if one party gets nothing would we deny that there has been a
compromise; and we say so without implying that morally she should have got
something. It is simply a bad joke to speak of someone's being a party to a
compromise when she has got nothing out of it

 Young says, the theory of bargaining, in the branch occupied by the theory of
games, is a theory of outcomes, not of processes.



On ethic of compromise: When one can compromise without being compromised?
