%!TeX root= ../thesis.tex

\section{Compromise}

After having described voting rules as procedures designed to aggregate a set of preferences into a common choice, we could say that each one of these mechanism reflects some sort of compromise. In fact, reaching a collective choice by unanimous accord is a fairly rare situation, especially in elections with a large number of voters. In all other cases, someone has to give up their top preference in favour of a second, or third etc., choice. Yet, can we call this a compromise? What do we mean when we talk about this notion?

One way of approaching the analysis of a concept is to start from the etymology of the word itself. Compromise, from the Latin \textit{compromissus}, past participle of \textit{compromittere}: com \textit{together} and promittere \textit{to promise}.
The idea of compromise can indeed be found in texts dating back to Roman times, where two disputing parties who wished to submit their contention to arbitration, would appoint a third party (an arbitrator) and make the mutual promise to abide by the latter's unquestionable judgment. If either party broke this promise, it would have had to pay a penalty.
In a fragment of a letter written by Proculus, which is reported by \citet[p.529]{Zimmermann1996}, it is clear that the two disputants agreed to give the arbitrator unlimited powers in his decision. No appeal was possible against the final award, which was binding no matter how unjust and unequal it was. This idea of compromise as a mere arbitration process seems very different from the notion we have of it today. Let us consider the definitions of the two verbs in a modern dictionary: to arbitrate "to settle a dispute between two people or groups after hearing the arguments and opinions of both" \citep{Arbitration}; to compromise "to come to agreement by mutual concession" \citep{Compromise}.
The mutual concession is the element that immediately makes us think of a compromise and it is precisely the missing factor in the description of the Roman compromise and modern arbitration.

However, Roman society is not the first one to use mechanisms of compromise; we find references to use of arbitration and selection of arbitrators in Aristotle’s Constitution of the Athenians [53.2–4]. %considered to be written between 328 BC and 322 BC; official arbiters were appointed from among men older than 59
Moreover, Aristotle is the first to associate the concept of fairness with that of arbitration:
%
\begin{displayquote}
\textit{\say{Fairness, for example, seems to be just; but fairness is justice that goes beyond the written law.}}\citet{OnRhetoric}[1.13.13]
\textit{\say{And it is fair to want to go into arbitration rather than to court; for the arbitrator sees what is fair, but the jury looks to the law, and for this reason arbitrators have been invented, that fairness may prevail.}}\citet{OnRhetoric}[1.13.19]
\end{displayquote}
%
Throughout history, however, compromise has been intended in its Roman meaning: the resolution of a dispute by a third party whose judgment individuals agree to abide by.
The use of arbitration continued until the end of the Middle Ages, only to be overshadowed by the advent of absolute monarchies, like the one of Luis XIV, that tended to centralise all powers in a single strong figure.\citep{Zappala2018}
However, this did not last long. In France, for example, the first constitution approved after the revolution, in 1791, defended the right of arbitration:
%
\begin{displayquote}
\textit{\say{Chapitre V. Article 5. - Le droit des citoyens, de terminer définitivement leurs contestations par la voie de l'arbitrage, ne peut recevoir aucune atteinte par les actes du Pouvoir législatif.}}\citep{Constitution1791}
\end{displayquote}
%
It would seem at this point that we are straying a little from the topics treated in this thesis. However, this excursus on how the notion of compromise has been understood for centuries can help us to understand the reasons why in the history of social choice theory a compromise rule that prioritises fairness has never been investigated.

Consider an example used by \cite{Luban1985}, Rich and Poor are offered $1000$ dollars which they can obtain provided they reach a compromise on how to divide this sum. 
Rich who knows to have leverage over Poor offers him $100$ dollars threatening to walk from the deal if he does not accept. Poor, who is in desperate need of money, accepts.
Luban points out that this is considered a compromise because $1)$ it produces an agreement $2)$ the criterion of "equal satisfaction" is met if we imagine that $100$ dollars give Poor the same level of satisfaction that $900$ give Rich. Most of us, however, would not consider this outcome as a compromise but, instead, as an outrageous bullying. We can paraphrase \citet{Braybrooke1982} when he says \say{It is simply a bad joke to speak of someone's being a party to a compromise when she has got nothing out of it.}.

But then what is the best compromise possible between Rich and Poor? It is clear that there is no clear answer.
If equality is an important social value, we would intuitively think that the only acceptable result is a $500/500$ split. That is, the result where, in the interpretation of compromise as "mutual concession", both parties concede equally.
This consideration is the basis for our version of compromise in social choice theory presented in \Cref{ch:compromise}.
It is important to point out that the concepts of equality and fairness have very specific meanings depending on the considered literature. 
If we were to read a Marx's essay on the redistribution of wealth then a fair solution would be for Rich to get $100$ dollars and Poor $900$.

In this manuscript we propose a notion of compromise in which equality lies in the losses from the initial position of all parties. For us, a compromise is a situation in which all parties are not only willing to concede, but that at the end of the process have actually given up all equally something.



--------------

compromise as a noun vs compromise as a verb (outcome or process)

compromise, a general technique for eliciting mutual consent  COMPROMISE AND LITIGATION MARTIN SHAPIRO \cite{Shapiro1979}

King Solomon’s story
Machiavelli N (1532) (1961) The Prince

\cite{Luban1985}
In a simple and well-known example, Rich and Poor are offered $1000$ to divide as they will, provided that they agree on a division. Rich offers Poor $100$, threatening to walk if Poor doesn't accept; since Rich can afford to make good on the threat, and Poor needs the money, Poor accepts. On the first, minimal, criterion listed above, this is a successful negotiation because it produced an agreement. Indeed, it may be successful on the "equal satisfaction" criterion as well, if $100$ gives Poor the same level of satisfaction that $900$ gives Rich. On others, it is not. If simple equality is an important social value, only a $500/500$ split would be satisfactory; if public policy is strongly redistributive, or if we are (or should be) committed to the Biblical and Marxist principle "to each according to his need," a $50/950$ split may be a successful outcome. Depending on the notion of success at work, different empirical conditions would likely have to be met for negotiation to succeed. These differences, moreover, concern only the outcome of the negotiation. When we look at the process as well, we may be even less inclined to consider the $900/1000$ split satisfactory: after all, it resulted from sheer economic bullying on the part of Rich. Most of us, indeed, are likely
to consider the $900/1000$ outcome to this negotiation outrageous. Nevertheless, we often find social scientific work presupposing the least ambitious criterion of success: A negotiation is successful when it produces an agreement (even if the agreement is lopsided)


Nash describes his solution as "the amount of satisfaction each individual should expect to get from the situation" (p. 155), and thus as a
"fair bargain" (p. I58).
For criticisms of the Nash solution, see Amartya K. Sen, Collective Choice and Social
Welfare (San Francisco, CA: Holden Day, I970), pp. 121-22;

----
\cite{Day1989}
An uncompromising person is one who is not disposed to make any concessions, so (it is erroneously inferred) a compromising person is one who is pliant and disposed to make concessions—regardless of whether he receives any concession in return. However this may be, one must insist, as it is generally agreed, that compromise necessarily involves mutual concessions. 'It is simply a bad joke to speak of someone's being a party to a compromise when she has got nothing out of it'. \cite{Braybrooke1982}
---

\cite{Braybrooke1982}
Golding, Benditt, and Kuflik, especially the first two, somewhat confuse the
issues by attaching too much weight to a supposed distinction between "com-
promise" in the sense of a result and "compromise" in the sense of a process and
then, in regard to the latter, inclining to confine the term to morally acceptable
instances

Carens mentions empirical grounds for thinking that if the parties to a process of negoti
ation adopt a "problem-solving" orientation-in which, without abating the
pursuit of their own interests, they recognize a shared problem and offer coopera-
 tion in solving it-the chances of reaching a result satisfactory to all parties
increase

 The noun "compromise" covers every solution in which more than one party
gets something, but something less than she aimed at and was available as a
possible share (other claimants aside) of the goods sought. Some of them may be
very unfair solutions, in that one party gets almost all that she aimed at, while the
others get almost nothing-still something, but far less than they are entitled to.
Only, however, if one party gets nothing would we deny that there has been a
compromise; and we say so without implying that morally she should have got
something. It is simply a bad joke to speak of someone's being a party to a
compromise when she has got nothing out of it

Young says, the theory of bargaining, in the branch occupied by the theory of
games, is a theory of outcomes, not of processes.



On ethic of compromise: When one can compromise without being compromised?
