
\section{Main concepts and definitions}



\subsection{Axioms}

	\subsubsection{Anonymity}
	An SCF $f$ is anonymous if each pair of voters play interchangeable roles: $f(P)=f(P^{\ast})$ holds whenever $P^{\ast}$
	is obtained from $P$ by swapping the ballots cast by two voters.
	
	\subsubsection{Neutrality}
	An SCF $f$ is neutral if each pair of alternatives are
	interchangeable: when $P^+$ is obtained from $P$ by swapping the positions of two alternatives in every ballot and $f(P^+)$ is obtained from $f(P)$ via a similar swap.
	
	\subsubsection{Nondictatoriality}
	There does not exist a voter such that the rule simply always copies that voter’s ranking.
	
	\subsubsection{Nonimposition}
	There does not exist an alternative x that cannot be elected (for which it does not exist a profile P such that $f(P)={x}$).
	
	\subsubsection{Unanimity}
	If all votes rank x first, then x should win.
	
	\subsubsection{Pareto Property}
	In P alternative x Pareto dominates alternative y if every voter ranks x over y; y is Pareto dominated if such an x exists. Then an SCF $f$ is \textbf{Pareto} (optimal) if $f(P)$ never contains a Pareto dominated alternative.
	
	\begin{theorem}[Moulin, 1983]
		Let $m \geq 2$ be the number of alternatives and n be the number of voters. If n is divisible by any integer r with $1 < r \leq m$, then no neutral, anonymous and Pareto SCF is resolute (single valued).
	\end{theorem}
	
	\subsubsection{Condorcet consistency}
	A candidate is the Condorcet winner if it wins all of its pairwise elections. Does not always exist but the Condorcet criterion says that if it does exist, it should win.
	
	An SCF is a Condorcet extension (or Condorcet consistent) if it selects the Condorcet winner alone, when it exists.

	\begin{example} \textbf{Plurality rule} does not satisfy it.
		\begin{itemize}
			\item[] b $>$ a $>$ c $>$ d
			\item[] c $>$ a $>$ b $>$ d 
			\item[] d $>$ a $>$ b $>$ c
		\end{itemize}
		a is the Condorcet winner, but it does not win under plurality.
	\end{example}
	
	\textbf{Borda rule} does not satisfy it.
	
	\textbf{STV} does not satisfy it.
	
	The \textbf{Copeland} score of a Condorcet winner is $m-1$ and uniquely highest, thus is a Condorcet extension.
	
	\begin{theorem}[Condorcet’s voting paradox]
		Problems are caused by majority cycles (e.g. a $>$ b, b $>$ c and c $>$ a). These are known as Condorcet’s voting paradox and relate to the intransitivity of $>$.
	\end{theorem}
	
	\subsubsection{Majority criterion}
	If a candidate is ranked first by a majority of the votes, that candidate should win.
	
	\begin{example} \textbf{Borda rule} does not satisfy it.
		\begin{itemize}
			\item[] a $>$ b $>$ c $>$ d $>$ e
			\item[] a $>$ b $>$ c $>$ d $>$ e
			\item[] c $>$ b $>$ d $>$ e $>$ a
		\end{itemize}
		a is the majority winner, but it does not win under Borda.
	\end{example}

	\subsubsection{Manipulability}

	\subsubsection{Reinforcement}
	
	\subsubsection{Monotonicity}
	\begin{itemize}
		\item[] \textbf{Weak Monotonicity}
		If a candidate w wins for the current votes and we improve the position of w in some of the votes leaving everything else the same, then w should still win.
		
		\begin{example} \textbf{STV} does not satisfy it.
			\begin{itemize}
				\item[] 7 votes b $>$ c $>$ a
				\item[] 7 votes a $>$ b $>$ c
				\item[] 6 votes c $>$ a $>$ b
			\end{itemize}
			c drops out first, its votes transfer to a, a wins.
			
			But if 2 votes b $>$ c $>$ a change to a $>$ b $>$ c, b drops out first, its 5 votes transfer to c, and c wins.
		\end{example}
		\item[] \textbf{Strong Monotonicity}
		If a candidate w wins for the current votes and we then change the votes in such a way that for each vote, if a
		candidate c was ranked below w originally, c is still ranked below w in the new vote, then w should still win.
		
		None of our rules satisfy this.
	\end{itemize}

	\subsubsection{Strategy-Proofness}
	
	\subsubsection{Independence of Irrelevant Alternatives}
	If the rule ranks a above b for the current votes and we change the votes but do not change which is ahead between a and b in each vote then a should still be ranked ahead of b.
	
	None of our rules satisfy this.


\section{Voting rules}

%\subsection{Majority}
%
%\subsection{Pairwise Majority}
%Pairwise Majority Rule (PMR) declares the winning alternative to be the Condorcet winner and is undefined when a profile has no such winner.
%
%\subsection{Plurality}
%
%\subsection{Borda}
%
%\subsection{Copeland}
















