%!TeX root= ../thesis.tex
In the introductory chapter we have mentioned different concepts like voting, preference aggregation processes, voting rules. All these terms are rather vague if we do not provide precise definitions.
In this chapter we will present the main notions we will use throughout the course of the dissertation and assign them precise meanings.\commentBN{Precise that we focus on rankings as ballot but be aware that more exist and we'll briefly speak about in sec 2.}

\section{Voting with ranked ballots}

A social choice setting consists of a set $\voters$ of $n$ individuals\textemdash or voters\textemdash who have preferences over a set $\alts$ of $m$ alternatives.
A social choice problem deals with aggregating those preferences, through the use of a social choice rule, to obtain a common choice for the group.
The preferences of individuals are represented by linear orders over the set of alternatives $A$.
In general, a linear order $\mathbin{\pref}$ on a set $X$ is a relation that is:
\vspace{-0.3em}
\begin{itemize}
	\itemsep-0.5em 
	\item irreflexive: $a \not\pref a, \forall a \in X$
	\item transitive: $a \pref b \land b \pref c \Rightarrow a \pref c, \forall a,b,c \in X$
	\item complete: $a \pref b \lor b \pref a, \forall a \neq b \in X$
\end{itemize} 
\vspace{-0.3em}
We can also define a non-strict linear order $\mathbin{\prefeq}$ on a set $X$ as a relation that is transitive, complete, reflexive ($a \prefeq a, \forall a \in X$) and antisymmetric ($a \prefeq b \land b \prefeq a \Rightarrow a = b, \forall a,b \in X$).
For each non-strict linear order $\prefeq$ there exists an associated linear order $\pref$ obtained by removing all pairs $a \prefeq a, \forall a \in X$. In other words, we have $a \pref b \iff a \prefeq b \land a \neq b, \forall a,b \in X$.

$\linors$ denotes the set of all linear orders over $A$. A generic element $\mathbin{\prefi} \in \linors$ stands for the \textit{preference ranking} of the individual $i\in N$.
A \emph{profile} $\prof: N → \linors$ associates with each individual $i \in N$ a preference $\prof(i) =\mathbin{\prefi}$. We will use $\mathbin{\prefi}$ instead of $P(i)$ when the profile is clear from the context. $\linors^{N}$ denotes the set of all functions from $N$ to $\linors$, thus all possible profiles given $N$. 
%We write $r_{\prefi}(a)=\#\{b\in A \suchthat b \prefi a\}+1$ for the \emph{rank} of $a\in A$ at ${\prefi} \in \linors$. 
\begin{example}
	\label{ex:namedprofile}
	Given a set of alternatives $A=\{a,b,c\}$ and a set $N=\{v_1,v_2,v_3,v_4\}$ of voters, we represent a profile $P$ as:
	\begin{center}
		$
		\begin{array}{cccc}
			v_1 & v_2 & v_3 & v_4\\
			a &	b & c & c \\
			b &	c & a & a \\
			c &	a & b & b \\
		\end{array} \quad, 
		$
	\end{center}
	where each column is the preference ranking associated to a certain voter. In this example the voter $v_1$ prefers the alternative a over b over c, $v_2$ prefers b over c over a, and both $v_3$ and $v_4$ prefer c over a over b.
\end{example}

A \emph{\acl{SCR}} (\acs{SCR}) is a mapping $f:\linors^{N}\rightarrow 2^{A} \setminus \{\emptyset \}$ that associates to each profile a set of (tied) winners. 
As we saw in \Cref{ch:intro}, scholars have realised that the most efficient way to define aggregation functions is through the formulation of desirable properties.
These properties are called \textit{axioms} and they are helpful in dividing rules into classes whose elements satisfy the same set of axioms.
Or, in the same way, one can prove that a given class is empty, i.e. that it is impossible to satisfy a set of axioms simultaneously.
As we shall see, axioms that independently seem to be reasonable requirements can result, when combined, in undesirable effects that are often difficult to detect without an systematical analysis.

In what follows we will describe the axioms and the voting rules that we will consider through the rest of this dissertation. In particular we will divide the ones based on preference rankings from the ones using judgments as ballots, and we will see which axioms each rule satisfies.

\subsection{Axioms}
As already mentioned, an axiom formalises an appealing property of a voting rule.
A first intuitive desideratum is that the rule should not favour specific voters or candidates.
In many real cases, when a committee has to make a decision, in case of a tie a member, usually the chair, has the power to break it at his discretion.
Similarly, in many legislatures in case of a tie between approving a law or not, the status quo wins, highlighting an inequality between the alternatives.
Ideally we would not want either of these behaviours \commentOC{I find this a bit too normative.}. In what follows we present two axioms that promote equality among voters, \textit{anonymity}, and among candidates, \textit{neutrality}. 

	\begin{genthm}{Anonymity}
	An \acs{SCR} $f$ is anonymous iff for any $i,j \in N$ and $P,P'\in\linors^{N}$ such that $\mathbin{\pref'_k}=\mathbin{\pref_k}, \forall k \in {N \setminus\{i,j\}}$ and $\mathbin{\pref'_i}=\mathbin{\pref_j}, \mathbin{\pref'_j}=\mathbin{\pref_i}$, we have that $f(P)=f(P')$. In other words, if the identity of voters is irrelevant.
	%$P'$ obtained from $P$ by swapping the preference rankings of two voters $f(P)=f(P')$. 
	\end{genthm}

	\begin{genthm}{Neutrality}
	An \acs{SCR} $f$ is neutral iff for any $P,P'\in\linors^{N}$ such that $P'$ is obtained from $P$ by swapping the position of two alternatives in every preference rankings, we can obtain $f(P')$ from $f(P)$ via the same swap. In other words, if the identity of alternatives is irrelevant.	
	\end{genthm}

	\begin{example}
		\label{ex:anonymousprofile}
		With this in mind we can rewrite \Cref{ex:namedprofile} by considering anonymity. In fact, if we are not interested in the identity of the voters, $P$ can be expressed as:
		\begin{center}
			$
			\begin{array}{ccc}
				1 & 1 & 2 \\
				a &	b & c \\
				b &	c & a \\
				c &	a & b \\
			\end{array} \quad, 
			$
		\end{center}
		where each column represents the preference ranking associated to a certain number of voters. Here, one voter prefers the alternative a over b over c, one voter prefers b over c over a, and two voters prefer c over a over b. 
		This way of denoting a profile will be used often throughout the dissertation.
	\end{example}

	We could also consider a weaker form of anonymity were the voters are not necessarily equal but which does not allow for a dictator\textemdash a specific voter whose preferred choice is always the outcome of the voting rule. 

	\begin{genthm}{Nondictatoriality}
	An \acs{SCR} $f$ is nondictatorial iff $\forall i \in N$ there exists a profile $P \in\linors^{N}$ such that $\mathbin{\prefi}(1) \not\in f(P)$; where $\mathbin{\prefi}(1)$ is the first element of the linear order representing $i$ preferences. In other words, if no voter acts as a dictator.
	\end{genthm}

%	\begin{genthm}{Nonimposition}
%	As a weak version of neutrality, an \acs{SCR} $f$ is non-imposed iff given an alternative $a \in A$ there exists a profile $P\in\linors^{N}$ such that $f(P)=\{a\}$. In other words, there does not exist an alternative that cannot be elected.	
%	\end{genthm}

	Another fairly obvious property that might come to mind is that if \textit{enough} voters agree on an alternative then it must be selected as the winner. To better define the concept of \textit{enough}, given a profile $P\in\linors^{N}$ and an alternative $a\in A$, we consider the number of preference rankings for which $a$ is ranked first: $\eta(a)=|\{\mathbin{\prefi}, i \in N \suchthat \mathbin{\prefi}(1)=a\}|$.
	
	\begin{genthm}{Unanimity}
	An \acs{SCR} $f$ is unanimous iff for any $P\in\linors^{N}$ and $a \in A$ such that $\eta(a)=n$ then $f(P)=\{a\}$. In other words, if all voters rank the same alternative first, then that alternative should be the sole winner.	
	\end{genthm}

	\begin{genthm}{Majority}
	An \acs{SCR} $f$ is majoritarian iff for any $P\in\linors^{N}$ and $a \in A$ such that $\eta(a)> \frac{n}{2}$ then $f(P)=\{a\}$. In other words, if the majority of voters rank the same alternative first, then that alternative should be the sole winner.
	\end{genthm}

	This idea leads us to two more general considerations. First, if all voters prefer an alternative \textit{a} to another one \textit{b} then this preference must be translated into the result: \textit{b} cannot be part of the winners. 
	Second, if a majority of voters prefer an alternative \textit{a} to any other, then \textit{a} should be the winner. 
	To better define this last concept, given a profile $P\in\linors^{N}$ and two alternatives $a,b\in A$, we denote with $\mu(a,b)=|\{\mathbin{\prefi}, i \in N \suchthat a\prefi b\}|$ the number of preference rankings for which $a$ is preferred to $b$.

	\begin{genthm}{Pareto Property}
	Given a profile $P\in\linors^{N}$ and any distinct $a,b\in A$, $a$ Pareto dominates $b$ at $P$ (or equivalently $b$ is Pareto dominated by $a$ at $P$) iff $a \prefi b,\forall i\in N$.
	
	We denote by $\paretopt(P)= \set{a \in A \suchthat \forall b \in A\setminus\{a\}, \exists i \in N, a \pref_i b}$ the set of Pareto optimal alternatives at $P$.
	
	An \acs{SCR} $f$ is Paretian iff $f(P)\subseteq\paretopt(P)$ $\forall P\in\linors^{N}$. In other words, $f$ is Paretian (or Pareto optimal) if $f(P)$ never contains Pareto dominated alternatives.	
	\end{genthm} 
	
	\begin{genthm}{Condorcet criterion}
	Given a profile $P\in\linors^{N}$, an alternative $a\in A$ is the Condorcet winner iff $\mu(a,b)> \mu(b,a), \forall b\in{A\setminus{\{a\}}}$. In other words, if $a$ beats every other alternatives in pairwise comparisons.
	
	An \acs{SCR} $f$ is a Condorcet procedure if it elects the Condorcet winner whenever it exists.	
	\end{genthm}

	Note that for an \acs{SCR} $f$ being majoritarian implies that there is an alternative $a \in A$ that is preferred to any other by a majority of the voters: $\mu(a,b)> \frac{n}{2}, \forall b\in{A\setminus{\{a\}}}$. This also means that $\mu(a,b)> \mu(b,a), \forall b\in{A\setminus{\{a\}}}$ so that $a$ is the Condorcet winner. Thus, if $f$ is majoritarian $f$ is also a Condorcet procedure.
	The converse is not always true, if $a$ is the Condorcet winner then it is not necessarily the majority choice.
	\begin{example}
		\label{ex:condorcetMaj}
		Consider the following profile $P$:
		\begin{center}
			$
			\begin{array}{ccc}
				1 & 1 & 1 \\
				a &	c & d \\
				b &	b & b \\
				c &	a & b \\
				d & d & c \\
			\end{array} \quad, 
			$
		\end{center}
		\begin{alignat*}{5}
			\mu(a,b)&=1, \quad && \mu(a,c)&&=2  \quad && \mu(a,d)&&=2 \\ 
			\mu(b,a)&=2, \quad && \mu(b,c)&&=2  \quad && \mu(b,d)&&=2  \\
			\mu(c,a)&=1, \quad && \mu(c,b)&&=1  \quad && \mu(c,d)&&=2  \quad.\\
			\mu(d,a)&=1, \quad && \mu(d,b)&&=1  \quad && \mu(d,c)&&=1 
		\end{alignat*}
		$a$ is the Condorcet winner but it is not the majority choice.
	\end{example}
	
	It is also worth to mention that the Condorcet winner does not always exist and that, with three or more alternatives, \textit{majority cycles} may occur.
	\begin{example}
		\label{ex:condorcetParadox}
		Consider the following profile $P$:
		\begin{center}
			$
			\begin{array}{ccc}
				1 & 1 & 1 \\
				a &	b & c \\
				b &	c & a \\
				c &	a & b \\
			\end{array} \quad, 
			$
		\end{center}
		\begin{alignat*}{3}
			\mu(a,b)&=2, \quad && \mu(b,a)&&=1  \\ 
			\mu(b,c)&=2, &&\mu(c,b)&&=1  \quad.\\
			\mu(c,a)&=2, &&\mu(a,c)&&=1  
		\end{alignat*}
		Thus, $a$ is preferred to $b$, $b$ is preferred to $c$, but $c$ is preferred to $a$. Such cycle of preferences is also known as \textbf{Condorcet’s paradox}.
	\end{example}

	In all properties considered so far we have assumed that voters report their sincere preferences. But what if a voter can achieve better results by casting an insincere ballot rather than expressing her true preferences?

	\begin{genthm}{Manipulability}
	An \acs{SCR} $f$ is manipulable iff there exist two profiles $P, P' \in\linors^{N}$ and a voter $j \in N$ such that $\mathbin{\pref'_i}=\mathbin{\prefi}, \forall i \in {N \setminus\{j\}}$ and $f(P') \pref_j f(P)$.
	In other words, if there exists a voter who by changing her preference ranking can change the outcome to something more appealing to her.
	\end{genthm}

	\begin{genthm}{Strategy-Proofness}
		An \acs{SCR} $f$ is strategyproof iff it is not manipulable.
	\end{genthm}



%	\begin{genthm}{Weak Monotonicity}
%		An \acs{SCR} $f$ is weakly monotone iff for each $P, P' \in\linors^{N}$ and $j \in N$ such that $\mathbin{\pref'_i}=\mathbin{\prefi}, \forall i \in {N \setminus\{j\}}$, if the voter $j$ ranks the alternative $f(P)$ in a higher position in $\mathbin{\pref'_j}$ than in $\mathbin{\pref}_j$, then $f(P)=f(P')$.
%		In other words, if a voter lifts the ranking of the winning alternative then the winner should remain unchanged. 
%	\end{genthm}
%
%	\begin{genthm}{Strong Monotonicity (Maskin)}
%	An \acs{SCR} $f$ is strongly monotone iff for each $P, P' \in\linors^{N}$, $j \in N$ and $b \in A$ such that $\mathbin{\pref'_i}=\mathbin{\prefi}, \forall i \in {N \setminus\{j\}}$, if both $f(P)\pref_j b$ and $f(P)\pref'_j b$ hold, then $f(P)=f(P')$.
%	In other words, if a voter lowers the ranking of an alternative originally already ranked below the winning alternative then the winner should remain unchanged. 
%	\end{genthm}
%
%	\begin{genthm}{\acl{IIA} (\acs{IIA})}
%	An \acs{SCR} $f$ is independent of irrelevant alternatives iff for each $P, P' \in\linors^{N}$, and $a,b \in A$ such that $a \prefi b$ iff $a \pref^{'}_{i} b$, $\forall i \in N$ then $f(P)$ and $f(P')$ should order $a,b$ in the exact same way. 
%	In other words, the order between two alternatives should not depend on the addition of an irrelevant third alternative.
%	\end{genthm}
	
	

%Condorcet
%	\begin{example} \textbf{Plurality rule} does not satisfy.
%		\begin{itemize}
%			\item[] b $>$ a $>$ c $>$ d
%			\item[] c $>$ a $>$ b $>$ d 
%			\item[] d $>$ a $>$ b $>$ c
%		\end{itemize}
%		a is the Condorcet winner, but it does not win under plurality.
%	\end{example}
%	
%	\textbf{Borda rule} does not satisfy it.
%	
%	\textbf{STV} does not satisfy it.
%	
%	The \textbf{Copeland} score of a Condorcet winner is $m-1$ and uniquely highest, thus is a Condorcet extension.

%	\subsubsection{Majority criterion}
%	
%	\begin{example} \textbf{Borda rule} does not satisfy it.
%		\begin{itemize}
%			\item[] a $>$ b $>$ c $>$ d $>$ e
%			\item[] a $>$ b $>$ c $>$ d $>$ e
%			\item[] c $>$ b $>$ d $>$ e $>$ a
%		\end{itemize}
%		a is the majority winner, but it does not win under Borda.
%	\end{example}
%	
%Weak Monotonicity}
%		
%		\begin{example} \textbf{STV} does not satisfy it.
%			\begin{itemize}
%				\item[] 7 votes b $>$ c $>$ a
%				\item[] 7 votes a $>$ b $>$ c
%				\item[] 6 votes c $>$ a $>$ b
%			\end{itemize}
%			c drops out first, its votes transfer to a, a wins.
%			
%			But if 2 votes b $>$ c $>$ a change to a $>$ b $>$ c, b drops out first, its 5 votes transfer to c, and c wins.
%		\end{example}
%
%Strong Monotonicity}
%		None of our rules satisfy this.
	
%	\subsubsection{Independence of Irrelevant Alternatives}
%	
%	None of our rules satisfy this.

	
%\begin{theorem}[Moulin, 1983]
%	Let $m \geq 2$ be the number of alternatives and n be the number of voters. If n is divisible by any integer r with $1 < r \leq m$, then no neutral, anonymous and Pareto SCF is resolute (single valued).
%\end{theorem}
%
%\begin{theorem}[Condorcet’s voting paradox]
%	Problems are caused by majority cycles (e.g. a $>$ b, b $>$ c and c $>$ a). These are known as Condorcet’s voting paradox and relate to the intransitivity of $>$.
%\end{theorem}

\subsection{Voting rules}

\subsubsection*{Majority}
Majority rule $f^m$ is a voting process that selects as winner the candidate $a\in A$ who is ranked first by the majority of voters, i.e. $f^m(P)=\{a\}$ iff $\eta(a)>\frac{n}{2}$.

\subsubsection*{\acl{PSR} (\acs{PSR})}
The \acs{PSR} $f^{\w}$ is the voting rule defined by the \emph{scoring vector} $\w=(w_1, \dots, w_m)$ which associate weights $w_r \in \R$ to positions, with $w_1 ≥ w_2 ≥ … ≥ w_m$.
Let $\alpha^{a}_r$ be the number of times that alternative $a$ was ranked in the $r$-th position, then the score of $a$ is defined by
\begin{align}
	\label{eq:srule}
	s(a) = \sum_{i\in N} w_{\mathbin{\prefi}(x)}
	= \sum_{r=1}^{m} \alpha^{a}_r w_r\ .
\end{align}
The winners $f^{\w}(P)$ are the alternatives with highest score. 
Different scoring vector $\w$ define different rules, here we describe three of them.

\begin{indented}[Plurality]
	The vector defining plurality rule is $\w=(1, 0, \dots, 0)$. This means that the score of each candidate corresponds to the number of times she was ranked first $s(a)=\eta(a)$.
	The winners are the candidates who are ranked first by the largest number of voters.
\end{indented}

\begin{indented}[Anti-Plurality]
	The vector defining anti-plurality rule is $\w=(1, \dots,1, 0)$.
	The winners are the candidates who are ranked last by the smallest number of voters.
\end{indented}

\begin{indented}[k-Approval]
	Plurality and Anti-Plurality can be seen as particular cases of a more general approval voting where each voter approves only one candidate and $m-1$ candidates respectively. For a generic $k \in \intvl{1,m-1}$, the vector defining k-approval rule is $\w=(\underbrace{1, \dots,1}_{k}, \underbrace{0,\dots, 0}_{m-k})$.
\end{indented}

\begin{indented}[Borda]
	The vector defining Borda rule is $\w=(m-1, m-2, \dots, 0)$. This means that, for each preference ranking, each candidate gets points inversely proportional to her ranking. The winners are the candidates with the highest score.
	As we mentioned in \Cref{ch:intro}, the Borda count is also known as Lull's method.
\end{indented}

\noindent These are only few examples of scoring rules, but more can be defined by changing the vector: 
\begin{itemize}
	\item the Eurovision Song Contest uses a \acs{PSR} with scoring vector \\ $\w=(12,10,8,7,6,5,4,3,2,1,0, \dots,0)$;
	\item the Formula One racing uses a \acs{PSR} with scoring vector \\ $\w=(25,18,15,12,10,8,6,4,2,1,0,\dots,0)$. \commentBN{talk about convexity}
\end{itemize}




\subsubsection*{Condorcet Procedures}

\cite{Fishburn1977}
%Equivalent to Lull-Copeland
\begin{indented}[Dodgson]
\end{indented}

\begin{indented}[Copeland]
\end{indented}

\subsubsection*{BK compromises}
\begin{indented}[Fall-back bargaining (Bucklin)]
\end{indented}



\commentBN{This is the table with the results but probably some are worth proving.}
\commentOC{Nice overview. I wouldn’t spend much space about proofs, as these are known results (and proving everything that is non obvious would take much space, I suppose). Perhaps a few easy proofs could be useful to give the feeling to the reader of how this works. Anyway, references are certainly needed. The set “Condorcet” is a bit odd here, being (I suppose) the set of rules which, by definition, do satisfy the axiom, this is not informative. One or two specific Condorcet rules would be more welcome, I’d say.}
\scalebox{0.8}{
	\begin{tabular}{cccccccccc}
		&	Anon.	&	Neut.	&	Nondict.	&	Unan.	&	Pareto	&	Condorcet	&	Majority &	Manipul.	&	SP	\\
		Majority	&	$\checkmark$	&	$\checkmark$	&	$\checkmark$	&	$\checkmark$	&	$\checkmark$	&	$\bm{\times}$	&	$\checkmark$	&	$\checkmark$	&	$\bm{\times}$	\\
		Plurality	&	$\checkmark$	&	$\checkmark$	&	$\checkmark$	&	$\checkmark$	&	$\checkmark$	&	$\bm{\times}$	&	$\checkmark$	&	$\checkmark$	&	$\bm{\times}$	\\
		Anti-Plurality	&	$\checkmark$	&	$\checkmark$	&	$\checkmark$	&	$\checkmark$	&	$\bm{\times}$	&	$\bm{\times}$	&	$\bm{\times}$	&	$\checkmark$	&	$\bm{\times}$	\\
		Borda	&	$\checkmark$	&	$\checkmark$	&	$\checkmark$	&	$\checkmark$	&	$\checkmark$	&	$\bm{\times}$	&	$\bm{\times}$	&	$\checkmark$	&	$\bm{\times}$	\\
		Condorcet (TODO)	&	$\checkmark$	&	$\checkmark$	&	$\checkmark$	&	$\checkmark$	&	*	&	$\checkmark$	&	$\checkmark$	&	$\checkmark$	&	$\bm{\times}$	\\
		BK Compromises	&	$\checkmark$	&	$\checkmark$	&	$\checkmark$	&	$\checkmark$	&	$\checkmark$	&	$\bm{\times}$	&	$\checkmark$	&	$\checkmark$	&	$\bm{\times}$	\\
		
	\end{tabular}
}

*depends– Dogson does not

% CONSISTENCY
%	\begin{genthm}{Consistency}
%		An \acs{SCR} $f$ is consistent iff $P \in\linors^{N}, P' \in\linors^{N'}$ such that $N \cap N' \neq \emptyset$ and $f(P)\cap f(P') \neq \emptyset$, then we have that $f(P \cup P')=f(P)\cap f(P')$. In other words, if the common winning alternatives for the two disjoint sets of voters are equivalent to the ones selected for the union of these sets.
%		The consistency axiom is also referred to as \say{reinforcement}.
%	\end{genthm}
%	\commentOC{You didn’t define $f$ as a rule operating on variable populations. 
%	In other words, $f$ is supposed to apply only on $N$-sized profiles.
%	So, there is a type problem about $f(P)$ and $f(P')$.
%	I suggest to drop this axiom if you don’t need it very much; or you’d have to introduce such kind of rules.}
%** Fallback yes, q<n no
%
%	\begin{tabular}{ccccccccccc}
%	&	Anon.	&	Neut.	&	Nondict.	&	Unan.	&	Pareto	&	Condorcet	&	Majority &	Manipul.	&	SP	&	Consist.	\\
%	Majority	&	$\checkmark$	&	$\checkmark$	&	$\checkmark$	&	$\checkmark$	&	$\checkmark$	&	$\bm{\times}$	&	$\checkmark$	&	$\checkmark$	&	$\bm{\times}$	&	$\checkmark$	\\
%	Plurality	&	$\checkmark$	&	$\checkmark$	&	$\checkmark$	&	$\checkmark$	&	$\checkmark$	&	$\bm{\times}$	&	$\checkmark$	&	$\checkmark$	&	$\bm{\times}$	&	$\checkmark$	\\
%	Anti-Plurality	&	$\checkmark$	&	$\checkmark$	&	$\checkmark$	&	$\checkmark$	&	$\bm{\times}$	&	$\bm{\times}$	&	$\bm{\times}$	&	$\checkmark$	&	$\bm{\times}$	&	$\checkmark$	\\
%	Borda	&	$\checkmark$	&	$\checkmark$	&	$\checkmark$	&	$\checkmark$	&	$\checkmark$	&	$\bm{\times}$	&	$\bm{\times}$	&	$\checkmark$	&	$\bm{\times}$	&	$\checkmark$	\\
%	Condorcet	&	$\checkmark$	&	$\checkmark$	&	$\checkmark$	&	$\checkmark$	&	*	&	$\checkmark$	&	$\checkmark$	&	$\checkmark$	&	$\bm{\times}$	&	$\bm{\times}$	\\
%	BK Compromises	&	$\checkmark$	&	$\checkmark$	&	$\checkmark$	&	$\checkmark$	&	$\checkmark$	&	$\bm{\times}$	&	$\checkmark$	&	$\checkmark$	&	$\bm{\times}$	&	**	\\
%	PSR	&	$\checkmark$	&	$\checkmark$	&	$\checkmark$	&	$\checkmark$	&	$\checkmark$	&	$\bm{\times}$	&	$\bm{\times}$	&	$\checkmark$	&	$\bm{\times}$	&	$\checkmark$	\\
%	
%\end{tabular}
%\begin{center}
%	$ P1:
%	\begin{array}{ccc}
%		4 & 4 & 2 \\
%		a &	b & c\\
%		b &	a & d\\
%		c &	c & a\\
%		d & d & b 
%	\end{array} \quad, 
%	$
%	$ P2:
%	\begin{array}{ccc}
%		2 & 1 \\
%		a &	c\\
%		b &	b\\
%		c &	a\\
%		d & d 
%	\end{array} , 
%	$
%\end{center}
%with $q= \ceil{\frac{n}{2}}$ then $f(P1)=\{a,b\}, f(P2)=\{a\}, f(P1 \cup P2)=\{a,b\} \neq f(P1)\cap f(P2)=\{a\}$
\subsection{Incompleteness}

\section{Voting with other}
\commentBN{we can have different different input or different output and because they're different mathematical object they also satisfy different axioms}
\begin{genthm}{Majority judgment}
	
\end{genthm}













