%!TeX root= ../thesis.tex
In the introductory chapter we have mentioned different concepts like voting, preference aggregation processes, voting rules. All these terms are rather vague if we do not provide precise definitions.
In this chapter we will present the main notions we will use throughout the course of the dissertation and assign them precise meanings.

It is worth specifying that when we talk about voting we imply an aggregation of votes expressed through a ballot. The term ballot comes from the ancient Italian \textit{ballotta}, a small ball used in medieval elections to cast a vote.
We can already find references to golden or silver balls used by the members of the various committees in the long procedure of electing the Doge after Venice's ordinance \citep{Doglioni1666}.
Over time, the term has assumed the generic meaning of a method to cast a vote (often secret).
In this dissertation we mainly use linear orders of preferences as ballots, i.e. each voter specifies which candidate is her best choice, her second choice and so on. However, this representation is often blamed as the reason for the impossibility results described by \citet{Arrow1950}. 
Other types of ballots have been proposed, for example \citet{Balinski2011} consider voters' judgments on candidates rather than preferences. 
To give an idea of the difference between these two representations we can imagine that in the case of ordered ballots a voter would say \say{I prefer alternative \textit{a} to alternative \textit{b}} while using judgments he would say \say{I consider alternative \textit{a} excellent and alternative \textit{b} mediocre}.
The latter notion will be explored in \Cref{sec:judgmentballots} and further on in \Cref{ch:MJ}.

A set of ballots constitutes the input of the aggregation function, since there are different representations of this object, this means that we can define different functions on different inputs. But what is the output?
Here again, there is no single answer. One can consider \textit{resolute} functions, i.e. functions that return a unique winner. In such cases, whenever there are ties between candidates, mechanisms are designed to break them.
However, this is not the assumption we will make in the course of this manuscript. The functions we will describe return the set of winning candidates.

It is important to emphasize that since these functions can be defined on different inputs and outputs, they are different mathematical objects. If we define a desirable property of a function that uses ranked ballots it might not be possible to convert it into a property of a function that uses judgments, and vice-versa.

%\commentBN{we can have different different input or different output and because they're different mathematical object they also satisfy different axioms}

\section{Voting with ranked ballots}
%\commentBN{ranked is the most used (Arrow, Handbook of COMSOC)}
A social choice setting consists of a set $\voters$ of $n$ individuals\textemdash or voters\textemdash who have preferences over a set $\alts$ of $m$ alternatives.
A social choice problem deals with aggregating those preferences, through the use of a social choice rule, to obtain a common choice for the group.
The preferences of individuals are represented by linear orders over the set of alternatives $A$.
In general, a linear order $\mathop{\pref}$ on a set $X$ is a relation that is:
\vspace{-0.3em}
\begin{itemize}
	\itemsep-0.5em 
	\item irreflexive: $a \not\pref a, \forall a \in X$
	\item transitive: $a \pref b \land b \pref c \Rightarrow a \pref c, \forall a,b,c \in X$
	\item complete: $a \pref b \lor b \pref a, \forall a \neq b \in X$
\end{itemize} 
\vspace{-0.3em}
We can also define a non-strict linear order $\mathop{\prefeq}$ on a set $X$ as a relation that is transitive, complete, reflexive ($a \prefeq a, \forall a \in X$) and antisymmetric ($a \prefeq b \land b \prefeq a \Rightarrow a = b, \forall a,b \in X$).
For each non-strict linear order $\prefeq$ there exists an associated linear order $\pref$ obtained by removing all pairs $a \prefeq a, \forall a \in X$. In other words, we have $a \pref b \iff a \prefeq b \land a \neq b, \forall a,b \in X$.

$\linors$ denotes the set of all linear orders over $A$. A generic element $\mathop{\prefi} \in \linors$ stands for the \textit{preference ranking} of the individual $i\in N$.
A \emph{profile} $\prof: N → \linors$ associates with each individual $i \in N$ a preference $\prof(i) =\mathop{\prefi}$. We will use $\mathop{\prefi}$ instead of $P(i)$ when the profile is clear from the context. $\linors^{N}$ denotes the set of all functions from $N$ to $\linors$, thus all possible profiles given $N$. 
%We write $r_{\prefi}(a)=\#\{b\in A \suchthat b \prefi a\}+1$ for the \emph{rank} of $a\in A$ at ${\prefi} \in \linors$. 
\begin{example}
	\label{ex:namedprofile}
	Given a set of alternatives $A=\{a,b,c\}$ and a set $N=\{v_1,v_2,v_3,v_4\}$ of voters, we represent a profile $P$ as:
	\begin{center}
		$
		\begin{array}{cccc}
			v_1 & v_2 & v_3 & v_4\\
			a &	b & c & c \\
			b &	c & a & a \\
			c &	a & b & b \\
		\end{array} \quad, 
		$
	\end{center}
	where each column is the preference ranking associated to a certain voter. In this example the voter $v_1$ prefers the alternative a over b over c, $v_2$ prefers b over c over a, and both $v_3$ and $v_4$ prefer c over a over b.
\end{example}

A \emph{\acl{SCR}} (\acs{SCR}) is a mapping $f:\linors^{N}\rightarrow 2^{A} \setminus \{\emptyset \}$ that associates to each profile a set of (tied) winners. 
As we saw in \Cref{ch:intro}, scholars have realized that the most efficient way to define aggregation functions is through the formulation of desirable properties.
These properties are called \textit{axioms} and they are helpful in dividing rules into classes whose elements satisfy the same set of axioms.
Or, in the same way, one can prove that a given class is empty, i.e. that it is impossible to satisfy a set of axioms simultaneously.
As we shall see, axioms that independently seem to be reasonable requirements can result, when combined, in undesirable effects that are often difficult to detect without an systematical analysis.

In what follows we will describe the axioms and the voting rules that we will consider through the rest of this dissertation. In particular we will divide the ones based on preference rankings from the ones using judgments as ballots, and we will see which axioms each rule satisfies.

\subsection{Axioms}
As already mentioned, an axiom formalizes an appealing property of a voting rule.
A first intuitive desideratum is that the rule should not favor specific voters or candidates.
In many real cases, when a committee has to make a decision, in case of a tie a member, usually the chair, has the power to break it at his discretion.
Similarly, in many legislatures in case of a tie between approving a law or not, the status quo wins, highlighting an inequality between the alternatives.
However, these behaviors are not always desired or accepted. In what follows we present two axioms that promote equality among voters, \textit{anonymity}, and among candidates, \textit{neutrality}. 

	\begin{genthm}{Anonymity}
	An \acs{SCR} $f$ is anonymous iff for any $i,j \in N$ and $P,P'\in\linors^{N}$ such that $\mathop{\pref'_k}=\mathop{\pref_k}, \forall k \in {N \setminus\{i,j\}}$ and $\mathop{\pref'_i}=\mathop{\pref_j}, \mathop{\pref'_j}=\mathop{\pref_i}$, we have that $f(P)=f(P')$. In other words, if the identity of voters is irrelevant.
	%$P'$ obtained from $P$ by swapping the preference rankings of two voters $f(P)=f(P')$. 
	\end{genthm}

	\begin{genthm}{Neutrality}
	An \acs{SCR} $f$ is neutral iff for any $P,P'\in\linors^{N}$ such that $P'$ is obtained from $P$ by swapping the position of two alternatives in every preference rankings, we can obtain $f(P')$ from $f(P)$ via the same swap. In other words, if the identity of alternatives is irrelevant.	
	\end{genthm}

	\begin{example}
		\label{ex:anonymousprofile}
		With this in mind we can rewrite \Cref{ex:namedprofile} by considering anonymity. In fact, if we are not interested in the identity of the voters, $P$ can be expressed as:
		\begin{center}
			$
			\begin{array}{ccc}
				1 & 1 & 2 \\
				a &	b & c \\
				b &	c & a \\
				c &	a & b \\
			\end{array} \quad, 
			$
		\end{center}
		where each column represents the preference ranking associated to a certain number of voters. Here, one voter prefers the alternative a over b over c, one voter prefers b over c over a, and two voters prefer c over a over b. 
		This way of denoting a profile will be used often throughout the dissertation.
	\end{example}

	We could also consider a weaker form of anonymity were the voters are not necessarily equal but which does not allow for a dictator\textemdash a specific voter whose preferred choice is always the outcome of the voting rule. 

	\begin{genthm}{Nondictatoriality}
	An \acs{SCR} $f$ is nondictatorial iff $\forall i \in N$ there exists a profile $P \in\linors^{N}$ such that $\mathop{\prefi}(1) \not\in f(P)$; where $\mathop{\prefi}(1)$ is the first element of the linear order representing $i$ preferences. In other words, if no voter acts as a dictator.
	\end{genthm}

%	\begin{genthm}{Nonimposition}
%	As a weak version of neutrality, an \acs{SCR} $f$ is non-imposed iff given an alternative $a \in A$ there exists a profile $P\in\linors^{N}$ such that $f(P)=\{a\}$. In other words, there does not exist an alternative that cannot be elected.	
%	\end{genthm}

	Another fairly obvious property that might come to mind is that if \textit{enough} voters agree on an alternative then it must be selected as the winner. To better define the concept of \textit{enough}, given a profile $P\in\linors^{N}$ and an alternative $a\in A$, we consider the number of preference rankings for which $a$ is ranked first: $\eta(a)=|\{\mathop{\prefi}, i \in N \suchthat \mathop{\prefi}(1)=a\}|$.

%	\begin{genthm}{Unanimity}
%	An \acs{SCR} $f$ is unanimous iff for any $P\in\linors^{N}$ and $a \in A$ such that $\eta(a)=n$ then $f(P)=\{a\}$. In other words, if all voters rank the same alternative first, then that alternative should be the sole winner.	
%	\end{genthm}

	\begin{genthm}{Majority}
	An \acs{SCR} $f$ is majoritarian iff for any $P\in\linors^{N}$ and $a \in A$ such that $\eta(a)> \frac{n}{2}$ then $f(P)=\{a\}$. In other words, if the majority of voters rank the same alternative first, then that alternative should be the sole winner.
	\end{genthm}

	This idea leads us to two more general considerations. First, if all voters prefer an alternative \textit{a} to another one \textit{b} then this preference must be translated into the result: \textit{b} cannot be part of the winners. 
	Second, if a majority of voters prefer an alternative \textit{a} to any other, then \textit{a} should be the winner. 
	To better define this last concept, given a profile $P\in\linors^{N}$ and two alternatives $a,b\in A$, we denote with $\mu(a,b)=|\{\mathop{\prefi}, i \in N \suchthat a\prefi b\}|$ the number of preference rankings for which $a$ is preferred to $b$.

	\begin{genthm}{Pareto Property}
	Given a profile $P\in\linors^{N}$ and any distinct $a,b\in A$, $a$ Pareto dominates $b$ at $P$ (or equivalently $b$ is Pareto dominated by $a$ at $P$) iff $a \prefi b,\forall i\in N$.
	
	We denote by $\paretopt(P)= \set{a \in A \suchthat \forall b \in A\setminus\{a\}, \exists i \in N, a \pref_i b}$ the set of Pareto optimal alternatives at $P$.
	
	An \acs{SCR} $f$ is Paretian iff $f(P)\subseteq\paretopt(P)$ $\forall P\in\linors^{N}$. In other words, $f$ is Paretian if $f(P)$ never contains Pareto dominated alternatives.	
	\end{genthm} 
	
	\begin{genthm}{Condorcet criterion}
	Given a profile $P\in\linors^{N}$, an alternative $a\in A$ is the Condorcet winner iff $\mu(a,b)> \mu(b,a), \forall b\in{A\setminus{\{a\}}}$. In other words, if $a$ beats every other alternatives in pairwise comparisons.
	
	An \acs{SCR} $f$ is a Condorcet procedure if it elects the Condorcet winner whenever it exists.	
	\end{genthm}

	Note that for an \acs{SCR} $f$ being majoritarian implies that there is an alternative $a \in A$ that is preferred to any other by a majority of the voters: $\mu(a,b)> \frac{n}{2}, \forall b\in{A\setminus{\{a\}}}$. This also means that $\mu(a,b)> \mu(b,a), \forall b\in{A\setminus{\{a\}}}$ so that $a$ is the Condorcet winner.
	The converse is not always true, if an alternative is the Condorcet winner then it is not necessarily the majority choice.
	\begin{example}
		\label{ex:condorcetMaj}
		Consider the following profile $P$:
		\begin{center}
			$
			\begin{array}{ccc}
				1 & 1 & 1 \\
				a &	c & d \\
				b &	b & b \\
				c &	a & a \\
				d & d & c \\
			\end{array} \quad, 
			$
		\end{center}
		\begin{alignat*}{5}
			\mu(a,b)&=1, \quad && \mu(a,c)&&=2  \quad && \mu(a,d)&&=2 \\ 
			\mu(b,a)&=2, \quad && \mu(b,c)&&=2  \quad && \mu(b,d)&&=2  \\
			\mu(c,a)&=1, \quad && \mu(c,b)&&=1  \quad && \mu(c,d)&&=2  \quad.\\
			\mu(d,a)&=1, \quad && \mu(d,b)&&=1  \quad && \mu(d,c)&&=1 
		\end{alignat*}
		$b$ is the Condorcet winner but it is not the majority choice.
	\end{example}
	We conclude that the Condorcet criterion is a stronger axiom and its satisfaction implies that of the majority criterion: if $f$ is a Condorcet procedure then $f$ is also majoritarian. 
	
	It is also worth to mention that the Condorcet winner does not always exist and that, with three or more alternatives, \textit{majority cycles} may occur.
	\begin{example}
		\label{ex:condorcetParadox}
		Consider the following profile $P$:
		\begin{center}
			$
			\begin{array}{ccc}
				1 & 1 & 1 \\
				a &	b & c \\
				b &	c & a \\
				c &	a & b \\
			\end{array} \quad, 
			$
		\end{center}
		\begin{alignat*}{3}
			\mu(a,b)&=2, \quad && \mu(b,a)&&=1  \\ 
			\mu(b,c)&=2, &&\mu(c,b)&&=1  \quad.\\
			\mu(c,a)&=2, &&\mu(a,c)&&=1  
		\end{alignat*}
		Thus, $a$ is preferred to $b$, $b$ is preferred to $c$, but $c$ is preferred to $a$, so there is no Condorcet winner. Such cycle of preferences is also known as \textbf{Condorcet’s paradox}.
	\end{example}

	In all properties considered so far we have assumed that voters report their sincere preferences. But what if a voter can achieve better results by casting an insincere ballot rather than expressing her true preferences?

	\begin{genthm}{Manipulability}
	An \acs{SCR} $f$ is manipulable iff there exist two profiles $P, P' \in\linors^{N}$ and a voter $j \in N$ such that $\mathop{\pref'_i}=\mathop{\prefi}, \forall i \in {N \setminus\{j\}}$ and $f(P') \pref_j f(P)$.
	In other words, if there exists a voter who by changing her preference ranking can change the outcome to something more appealing to her.
	\end{genthm}

	\begin{genthm}{Strategy-Proofness}
		An \acs{SCR} $f$ is strategyproof iff it is not manipulable.
	\end{genthm}



%	\begin{genthm}{Weak Monotonicity}
%		An \acs{SCR} $f$ is weakly monotone iff for each $P, P' \in\linors^{N}$ and $j \in N$ such that $\mathop{\pref'_i}=\mathop{\prefi}, \forall i \in {N \setminus\{j\}}$, if the voter $j$ ranks the alternative $f(P)$ in a higher position in $\mathop{\pref'_j}$ than in $\mathop{\pref}_j$, then $f(P)=f(P')$.
%		In other words, if a voter lifts the ranking of the winning alternative then the winner should remain unchanged. 
%	\end{genthm}
%
%	\begin{genthm}{Strong Monotonicity (Maskin)}
%	An \acs{SCR} $f$ is strongly monotone iff for each $P, P' \in\linors^{N}$, $j \in N$ and $b \in A$ such that $\mathop{\pref'_i}=\mathop{\prefi}, \forall i \in {N \setminus\{j\}}$, if both $f(P)\pref_j b$ and $f(P)\pref'_j b$ hold, then $f(P)=f(P')$.
%	In other words, if a voter lowers the ranking of an alternative originally already ranked below the winning alternative then the winner should remain unchanged. 
%	\end{genthm}
%
%	\begin{genthm}{\acl{IIA} (\acs{IIA})}
%	An \acs{SCR} $f$ is independent of irrelevant alternatives iff for each $P, P' \in\linors^{N}$, and $a,b \in A$ such that $a \prefi b$ iff $a \pref^{'}_{i} b$, $\forall i \in N$ then $f(P)$ and $f(P')$ should order $a,b$ in the exact same way. 
%	In other words, the order between two alternatives should not depend on the addition of an irrelevant third alternative.
%	\end{genthm}
	
	

%Condorcet
%	\begin{example} \textbf{Plurality rule} does not satisfy.
%		\begin{itemize}
%			\item[] b $>$ a $>$ c $>$ d
%			\item[] c $>$ a $>$ b $>$ d 
%			\item[] d $>$ a $>$ b $>$ c
%		\end{itemize}
%		a is the Condorcet winner, but it does not win under plurality.
%	\end{example}
%	
%	\textbf{Borda rule} does not satisfy it.
%	
%	\textbf{STV} does not satisfy it.
%	
%	The \textbf{Copeland} score of a Condorcet winner is $m-1$ and uniquely highest, thus is a Condorcet extension.

%	\subsubsection{Majority criterion}
%	
%	\begin{example} \textbf{Borda rule} does not satisfy it.
%		\begin{itemize}
%			\item[] a $>$ b $>$ c $>$ d $>$ e
%			\item[] a $>$ b $>$ c $>$ d $>$ e
%			\item[] c $>$ b $>$ d $>$ e $>$ a
%		\end{itemize}
%		a is the majority winner, but it does not win under Borda.
%	\end{example}
%	
%Weak Monotonicity}
%		
%		\begin{example} \textbf{STV} does not satisfy it.
%			\begin{itemize}
%				\item[] 7 votes b $>$ c $>$ a
%				\item[] 7 votes a $>$ b $>$ c
%				\item[] 6 votes c $>$ a $>$ b
%			\end{itemize}
%			c drops out first, its votes transfer to a, a wins.
%			
%			But if 2 votes b $>$ c $>$ a change to a $>$ b $>$ c, b drops out first, its 5 votes transfer to c, and c wins.
%		\end{example}
%
%Strong Monotonicity}
%		None of our rules satisfy this.
	
%	\subsubsection{Independence of Irrelevant Alternatives}
%	
%	None of our rules satisfy this.

	
%\begin{theorem}[Moulin, 1983]
%	Let $m \geq 2$ be the number of alternatives and n be the number of voters. If n is divisible by any integer r with $1 < r \leq m$, then no neutral, anonymous and Pareto SCF is resolute (single valued).
%\end{theorem}
%
%\begin{theorem}[Condorcet’s voting paradox]
%	Problems are caused by majority cycles (e.g. a $>$ b, b $>$ c and c $>$ a). These are known as Condorcet’s voting paradox and relate to the intransitivity of $>$.
%\end{theorem}

\subsection{Voting rules}

Having established desirable properties for aggregating preferences, we can now describe some methods designed for this purpose.
Probably, one of the most intuitive approaches when a group of people get together and wish to find a common choice, is to elect the alternative supported by the majority of voters. The first thing we note is that this rule, as formulated, would not allow for the existence of multiple winners. Thus, while it might produce acceptable results with only two alternatives and an odd number of voters, in all other cases it would not always yield a winner.

Nevertheless, let us step back from these considerations for a moment and assume that there is always a majority that agrees on the choice of the winner. What would the decision of the majority imply?
In his rules for a open society, \citet[vol. 2, ch. 19]{Popper1945} writes:
\textit{\say{Democracy cannot be the majority, although fully characterized as the rule of the institution of general elections is most important. For a majority might rule in a tyrannical way. (The majority of those who are less than 6 ft. high may decide that the minority of those over 6 ft. shall pay all taxes.)}}.
Also thanks to his example, it is easy to imagine what Popper meant when he said that the majority can rule tyrannically.
But one thing that is also very important to point out is that a sort of tyranny could also lie in the election itself, in the sense that the majority choice could be hated by the rest of the population.
Imagine a situation where in a group of $100$ voters, $51$ of them have the following preference ranking $a\pref b \pref \dots \pref z$ and $49$ instead $z \pref y \pref \dots \pref a$. A majority of $51$ individuals would elect the candidate $a$ who is detested by the remaining $49$.

This makes us realize that rather intuitive methods, which seem fair at a first glance, can actually have unintended implications.To mitigate this problem and analyze the properties of voting rules, we can make use of the axioms just described.

The literature on voting rules is extensive and many have been proposed over the years, here we will present just a few examples of three particular categories: \textit{\acl{PSR}}, that are methods assigning points to positions and electing the candidate with the greatest cumulative score; \textit{Condorcet Procedures}, which are methods satisfying the Condorcet criterion, i.e. always electing the Condorcet winner whenever it exists; \textit{BK Compromises}, which is the term we use to refer to the generalization of bargaining rules explored by \citet{Brams2001}, where, starting from the top choice of each voters, we fall back until one or more candidates have \textit{enough} support to be considered winner.

Let us analyze these procedures in more detail.

%\subsubsection*{Majority}
%Majority rule $f^m$ is a voting process that selects as winner the candidate $a\in A$ who is ranked first by the majority of voters, i.e. $f^m(P)=\{a\}$ iff $\eta(a)>\frac{n}{2}$.

\subsubsection*{\acl{PSR} (\acs{PSR})}
The \acs{PSR} $f_{\w}$ is the voting rule defined by the \emph{scoring vector} $\w=(w_1, \dots, w_m)$ which associate weights $w_r \in \R$ to positions, with $w_1 ≥ w_2 ≥ … ≥ w_m$.
Let $\alpha^{a}_r$ be the number of times that alternative $a$ was ranked in the $r$-th position, then the score of $a$ is defined by
\begin{align}
	s(a) = \sum_{i\in N} w_{\mathop{\prefi}(x)}
	= \sum_{r=1}^{m} \alpha^{a}_r w_r\ .
\end{align}
The winners $f_{\w}(P)=\{a\in A \suchthat s(a)\geq s(b), \forall b \in A \setminus\{a\} \}$ are the alternatives with highest score, we will write this as $f_{\w}(P)=\argmax_{a\in A}s(a)$. 
Different scoring vector $\w$ define different rules, here we describe three of them.

\begin{indented}[Plurality]
	The vector defining plurality rule is $\w=(1, 0, \dots, 0)$. This means that the score of each candidate corresponds to the number of times she was ranked first $s(a)=\eta(a)$.
	The winners are the candidates who are ranked first by the largest number of voters.
\end{indented}

\begin{indented}[Anti-Plurality]
	The vector defining anti-plurality rule is $\w=(1, \dots,1, 0)$.
	The winners are the candidates who are ranked last by the smallest number of voters.
\end{indented}

\begin{indented}[k-Approval]
	Plurality and Anti-Plurality can be seen as particular cases of a more general approval voting where each voter assigns a score of 1 to, respectively, only one candidate and $m-1$ candidates. For a generic $k \in \intvl{1,m-1}$, the vector defining k-approval rule is $\w=(\underbrace{1, \dots,1}_{k}, \underbrace{0,\dots, 0}_{m-k})$.
\end{indented}
\vspace{-1.2em}
\begin{indented}[Borda]
	The vector defining Borda rule is $\w=(m-1, m-2, \dots, 0)$. This means that, for each preference ranking, each candidate gets points inversely proportional to her ranking. The winners are the candidates with the highest score.
	As we mentioned in \Cref{ch:intro}, the Borda count is also known as Lull's method.
\end{indented}

\noindent These are only few examples of scoring rules, but more can be defined by changing the vector. In some instances of real applications: 
\begin{itemize}
	\item the Eurovision Song Contest uses a \acs{PSR} with scoring vector $\w=(12,10,8,7,6,5,4,3,2,$ \\$1,0, \dots,0)$;
	\item the Formula One racing uses a \acs{PSR} with scoring vector
	$\w=(25,18,15,12,10,8,6,4,2,$\\$1,0,\dots,0)$
\end{itemize}




\subsubsection*{Condorcet Procedures}
Any method satisfying the Condorcet criterion is a Condorcet procedure. As we saw in the previous section, however, the Condorcet winner does not always exist. In these cases different ways to solve this ambiguity have been used, resulting in different procedures. These rules proposed over the years are many and go beyond the scope of this thesis, here we will describe only two of them but, for more details, \citet{Fishburn1977} describes and compares nine famous Condorcet procedures.

We would like to warn the reader, that these two specific rules will not be mentioned again in the course of the thesis.
The following material could be skipped (until the next set of BK rules) without compromising the understanding of the rest of the manuscript.
However, it seems worthwhile to us to give an example of Condorcet procedures and, in particular, two methods which, as we shall see, do not satisfy the same axioms.

Given a profile $P\in\linors^{N}$ and two alternatives $a,b\in A$, recall that $\mu(a,b)=|\{\mathop{\prefi}, i \in N \suchthat a\prefi b\}|$ represents the number of preference rankings for which $a$ is preferred to $b$.
We denote with $\rhd$ a simple majority win, and we say that $a \rhd b$ iff $\mu(a,b)>\mu(b,a)$.

\begin{indented}[Copeland]
	Copeland's method \textemdash which was also proposed by Lull as we saw in \Cref{ch:intro} \textemdash is often considered to be an extension of Condorcet's method. 
	In fact, the idea of this procedure is based on the fact that if a candidate whose pairwise wins are greater than anyone else is deserving of winning, then one whose number of pairwise wins minus pairwise losses is greater than everyone else is even more deserving.
	Let us formalize this concept, we define with $\gamma(a)=|\{b \in A\setminus \{a\} \suchthat a \rhd b\}|- |\{b \in A\setminus \{a\} \suchthat b \rhd a\}|$ the number of simple majority wins of $a$ minus the number of simple majority losses of $a$.
	The winners under the Copeland rule are the alternatives for which this value is the greatest $f_{C}(P)=\argmax_{a\in A}\gamma(a)$.
	%$f_{C}(P)=\{a\in A \suchthat \gamma(a)\geq \gamma(b), \forall b \in A \setminus\{a\} \}$.
\end{indented}

\begin{indented}[Schwartz]
	The idea behind the procedure proposed by \citet{Schwartz1972} is to identify the sets of alternatives such that every member in the set wins in simple majority all other alternatives outside the set. The union of all these sets represents the Schwartz set.
	We say that $a$ Schwartz dominates $b$, $a \vdash b$, iff $\exists x_1, \dots, x_k \in A \suchthat a \rhd x_1 \land x_1 \rhd x_2 \land \dots \land x_k\rhd b$ and  $\nexists y_1, \dots, y_l \in A \suchthat b \rhd y_1 \land y_1 \rhd y_2 \land \dots \land y_l\rhd a$. In other words, if there exists a path from $a$ to $b$ in $\rhd$ but not vice-versa.
%	\textemdash note that this is equivalent to check if a and be are connected in the asymmetric transitive closure of $\rhd$.
	The members of the Schwartz set are all candidates who are not Schwartz dominated.
	The winners are all the candidates in the Schwartz set, $f_{S}(P)=\{a\in A \suchthat  \nexists b \in A \setminus\{a\} \ s.t. \ b \bot a \}$.
	
\end{indented}


\begin{theorem}[\citet{Fishburn1977}]
	$f_{S}(P)$ is not Paretian.
\end{theorem}
\begin{proof}
	\label{ex:schwartzPareto}
	Consider the following profile $P$:
	\begin{center}
		$
		\begin{array}{ccc}
			1 & 1 & 1 \\
			a &	d & c \\
			b &	a & d \\
			c &	b & a \\
			d & c & b \\
		\end{array} \quad, 
		$
	\end{center}
	Consider, for each pair of alternatives, the number of profiles that prefer one alternative to the other.
	\begin{alignat*}{5}
		\mu(a,b)&=3, \quad && \mu(a,c)&&=2  \quad && \mu(a,d)&&=1 \\ 
		\mu(b,a)&=0, \quad && \mu(b,c)&&=2  \quad && \mu(b,d)&&=1  \\
		\mu(c,a)&=1, \quad && \mu(c,b)&&=1  \quad && \mu(c,d)&&=2  \quad.\\
		\mu(d,a)&=2, \quad && \mu(d,b)&&=2  \quad && \mu(d,c)&&=1 
	\end{alignat*}
	From this we deduce that $a$ Pareto dominates $b$ and, also, that the simple majority relations are $a \rhd b$, $a \rhd c$, $b \rhd c$, $c \rhd d$, $d \rhd a$, $d \rhd b$.
	
	For an alternative $a$ to be in the Schwartz set, there must not be an alternative $b$ for which there is a path from $b$ to $a$ in $\rhd$ but no path from $a$ to $b$.
	If we represent the preferences with a graph in which each node is an alternative and the arcs correspond to the $\rhd$ relations, it is easy to see that from whichever node we start we can always reach any other node.

	\begin{minipage}{.45\textwidth}
	\raggedright
		\scalebox{0.5}{
			\begin{tikzpicture}
				\node[shape=circle,draw=black,scale=2] (a) at (0,4) {a};
				\node[shape=circle,draw=black,scale=2] (b) at (4,0) {b};
				\node[shape=circle,draw=black,scale=2] (c) at (0,-4) {c};
				\node[shape=circle,draw=black,scale=2] (d) at (-4,0) {d};
				
				\path [->](a) edge node[right] {} (b);
				\path [->](b) edge node[right] {} (c);
				\path [->](c) edge node[right] {} (d);
				\path [->](d) edge node[right] {} (a);
				\path [->](a) edge node[right] {} (c);
				\path [->](d) edge node[right] {} (b);
				%		\path [->](E) edge[bend right=60]  node[right] {$8$} (F);   
			\end{tikzpicture}
		}
	\end{minipage}
	\begin{minipage}{.45\textwidth}
		\raggedleft
	\begin{alignat*}{5}
		&a \rhd b && \land \quad &&b \rhd c \land c \rhd d \land d \rhd a\\
		&a \rhd c &&\land \ && c \rhd d \land d \rhd a\\
		&a \rhd c \land c \rhd d \quad &&\land \ &&d \rhd a \\		
		&b \rhd c &&\land \ &&c \rhd d \land d \rhd b \\
		&b \rhd c \land c \rhd d &&\land \ &&d \rhd b \\	
		&c \rhd d &&\land \ &&d \rhd a \land a \rhd c
	\end{alignat*}
\end{minipage}
	
	\vspace{2em}
	\noindent
	This means that the Schwartz set is equal to the set of all alternatives, thus that $f_S(P)=\{a,b,c,d\}$. But $b$ is a Pareto dominated alternative, thus the Schwartz rule does not satisfy the Pareto criterion.
\end{proof}

This example shows us that even "basic" properties like Pareto are worth verifying, because rules that seem reasonable may not satisfy them.
And, to be fair, deviating a little, this may lead us to another consideration that will be useful later on: is Pareto really that basic and essential? \citet{Sen1970,Sen2004} thinks that this is not always the case. 
In fact, he introduced the concept of \textit{Minimal Liberty}, according to which there should be at least a pair of alternatives $a,b$ such that if an individual prefers $a$ to $b$ then society prefers $a \pref b$.
Considering all preference orders to be possible (\textit{Unrestricted Domain}), he showed that there is no way to aggregate preferences into a single common choice that simultaneously satisfies the Pareto criterion and Minimal Liberty.
Sen suggests that one way to escape this paradox is to give up Paretian efficiency:
\say{While the Pareto criterion has been thought to be an expression of individual liberty, it appears that in choices involving more than two alternatives it can have consequences that are, in fact, deeply illiberal.}\citep{Sen1970}
In our quest to identify rules that could reflect a true compromise between voters (\Cref{ch:compromise}), we will come across a class of rules that we find reasonable in some contexts but which do not satisfy the Pareto criterion.

\subsubsection*{BK compromises}
\label{sec:BK}
In order to describe the next set of rules it is necessary to make a premise. We have decided to group under the umbrella name of Brams and Kilgour (BK) compromises, all those rules which, starting from everyone’s ideal alternative, fall back to the voters’ second, third and more generally $k$-\emph{th} best, until a certain quota of voters $q$ support one of the alternatives considered so far.

This method, in its main idea, has been rediscovered several times over the years. Its proposal can be traced back to \citet{Condorcet1789, Condorcet1793}, whose procedure, translated by \citet[pp. 249-250]{McLean1994}, reads: \say{If one candidate has the absolute majority of first votes, he will be elected. If one candidate has the absolute majority of first votes and second votes together, he will be elected. If several candidates obtain this majority, the one with the most votes will be preferred. If one candidate has the absolute majority of the three votes together, he will be elected, and if several candidates obtain this majority, the one with the most votes will be preferred.} 

As \cite{Camps2014} point out, the same method was later analyzed by \cite{Lhuilier1793} and adopted in Geneva. Sometime after that, in the early 20th century, this procedure was promoted by the politician James W. Bucklin and adopted first in the municipality of Grand Junction (Colorado, USA) and then in many other cities of the United States \cite[p. 167]{Barber2000}. In 1986 it was proposed again by \citet{Sertel1986} and \citet{Sertel1999} under the name of Majoritarian Compromise. Later, \citet{Brams2001} generalize this concept and introduce a class of \acs{SCR} called $q$\emph{-approval fallback bargaining}, where $q$ is the required quantity of support that can vary from a single voter up to unanimity. Different choices of $q$ lead to different \acp{SCR}. Considering $n$ voters, the choice of $q=1$ corresponds to the plurality rule, $q=\ceil{\frac{n}{2}}$ is the quota used by the Majoritarian Compromise and $q=n$ represents a bargaining procedure called \emph{Fallback bargaining}, which has been further analyzed by \citet{Kibris2007} and \citet{Congar2012}.

To describe these BK compromises, recall that we are considering a finite set $N$ of $n$ individuals and a finite set $A$ of $m$ alternatives. A \emph{profile} $P: N → \linors$ associates with each individual $i \in N$ a preference $P(i) = {\prefi}$.
We write $r_{\prefi}(a)=\#\{b\in A \suchthat b \prefi a\}+1$ for the \emph{rank} of $a\in A$ at ${\prefi} \in \linors$.

Given any $ 1 \leq q \leq n$ and $1 \leq d \leq m$, consider the compromise set, i.e. the set of alternatives that are ranked among the top d by at least q voters $\text{CS}^{q}_{d}=\{a \in A \suchthat \#\{r_{\prefi}(a) \leq d, \forall i \in N\} \geq q \}$.
The q-approval fallback bargaining depth, $d_{q}^{*}$, is the smallest value that makes this set nonempty: $d_{q}^{*}=\min\{d \suchthat \text{CS}^{q}_{d} \neq \emptyset\}$.

The \acs{SCR} \textbf{q-approval fallback bargaining} is defined as the set of all alternatives that are ranked $d_{q}^{*}$ for at least q bargainers: $f_{\text{BK}}^{q}=\text{CS}^{q}_{d}$ where $d=d_{q}^{*}$.

Consider the maximum support at depth $d$ as the biggest value that makes the compromise set nonempty: $q_{d}^{*}=\max\{q \suchthat \text{CS}^{q}_{d} \neq \emptyset\}$.
We can define a revised version of the BK rule by adding an extra step. If multiple alternatives are ranked $d_{q}^{*}$ for at least q bargainers, the \textbf{revised q-approval fallback bargaining} returns the ones with the greatest support: $f_{\text{rBK}}^{q}= \text{CS}^{q}_{d}$ where $d=d_{q}^{*}$ and $q=q_{d}^{*}$. 
Note that when CS is a singleton, or when all the alternatives in CS are supported by the same amount of voters then $f_{\text{BK}}^{q}=f_{\text{rBK}}^{q}$
\Cref{ex:BK} shows how different quotas lead to different results.

\begin{example}[\cite{Brams2001}]
	\label{ex:BK}
	Consider the following profile $P$:
	\begin{center}
		$
		\begin{array}{cccc}
			1 & 1 & 1 & 1 \\
			a &	a & a & d \\
			b &	c & d & b \\
			c &	b & b & c \\
			d &	d & c & a \\
		\end{array} \quad, 
		$
	\end{center}
	\begin{itemize}
		\itemsep0em
		\item [] when $q=1$ then $d_{1}^{*}=1$ and $f_{\text{BK}}^{1}=\{a,d\}$ and $f_{\text{rBK}}^{1}=\{a\}$;
		\item [] when $q=2$ then $d_{2}^{*}=1$ and $f_{\text{BK}}^{2}=\{a\}$ and $f_{\text{rBK}}^{2}=\{a\}$;
		\item [] when $q=3$ then $d_{3}^{*}=1$ and $f_{\text{BK}}^{3}=\{a\}$ and $f_{\text{rBK}}^{3}=\{a\}$;
		\item [] when $q=4$ then $d_{4}^{*}=3$ and $f_{\text{BK}}^{4}=\{b\}$ and $f_{\text{rBK}}^{4}=\{b\}$.
	\end{itemize}
\end{example}

When $q=n$ the BK rule is referred to as \textbf{\acl{FB}} (\acs{FB}): $f_{\text{BK}}^{n}= f_{\text{FB}}$.

When $q=\ceil{\frac{n}{2}}$ the revised BK rule is denoted as \textbf{\acl{MC}} (\acs{MC}): 
\\$f_{\text{rBK}}^{\ceil{\frac{n}{2}}}= f_{\text{MC}}$. 
\Cref{ex:BKvsMC} shows a profile where the two sets of outcomes $f_{\text{MC}}$ and $f_{\text{BK}}^{\ceil{\frac{n}{2}}}$ do not coincide.

\begin{example}[\cite{Sertel1999} and \cite{Brams2001}]
	\label{ex:BKvsMC}
	Consider the following profile $P$:
	\begin{center}
		$
		\begin{array}{ccc}
			1 & 1 & 1 \\
			a &	c & b \\
			c &	a & c \\
			b &	b & a \\
		\end{array} \quad, 
		$
	\end{center}
	and $q=\ceil{\frac{n}{2}}=2$. Therefore, $d_{2}^{*}=2$ and the two \acp{SCR} return $f_{\text{BK}}^{2}=\{a,c\}$ and $f_{\text{MC}}=\{c\}$.
\end{example}

%It is generally more important to be sure of electing men who are worthy of holding office than to have a small probability of electing the worthiest man. - Condorcet

To conclude this section we show in \Cref{tab:overview} an overview of which of the axioms described is satisfied by the rules discussed so far. Note that we say that an axiom is satisfied by the class \acs{BK} Compromises if it is satisfied by every element in the class, and the converse if there is at least one element in the class for which it is not satisfied.

\begin{table}[h]
\scalebox{0.8}{
	\begin{tabular}{ccccccccc}
		&	Anon.	&	Neut.	&	Nondict.	&	Pareto	&	Condorcet	&	Majority &	Manipul.	&	SP	\\
		Plurality	&	$\checkmark$	&	$\checkmark$	&	$\checkmark$	&	$\checkmark$	&	$\bm{\times}$	&	$\checkmark$	&	$\checkmark$	&	$\bm{\times}$	\\
		Anti-Plurality	&	$\checkmark$	&	$\checkmark$	&	$\checkmark$	&	$\bm{\times}$	&	$\bm{\times}$	&	$\bm{\times}$	&	$\checkmark$	&	$\bm{\times}$	\\
		Borda	&	$\checkmark$	&	$\checkmark$	&	$\checkmark$	&	$\checkmark$	&	$\bm{\times}$	&	$\bm{\times}$	&	$\checkmark$	&	$\bm{\times}$	\\
		Copeland	&	$\checkmark$	&	$\checkmark$	&	$\checkmark$	&	$\checkmark$	&	$\checkmark$	&	$\checkmark$	&	$\checkmark$	&	$\bm{\times}$	\\
		Schwartz	&	$\checkmark$	&	$\checkmark$	&	$\checkmark$	&	$\bm{\times}$	&	$\checkmark$	&	$\checkmark$	&	$\checkmark$	&	$\bm{\times}$	\\
		BK Compromises	&	$\checkmark$	&	$\checkmark$	&	$\checkmark$ &	$\checkmark$	&	$\bm{\times}$	&	$\checkmark$	&	$\checkmark$	&	$\bm{\times}$	\\
		
	\end{tabular}

}
\caption{Axioms satisfied by the voting rules described.}
\label{tab:overview}
\end{table}


% CONSISTENCY
%	\begin{genthm}{Consistency}
%		An \acs{SCR} $f$ is consistent iff $P \in\linors^{N}, P' \in\linors^{N'}$ such that $N \cap N' \neq \emptyset$ and $f(P)\cap f(P') \neq \emptyset$, then we have that $f(P \cup P')=f(P)\cap f(P')$. In other words, if the common winning alternatives for the two disjoint sets of voters are equivalent to the ones selected for the union of these sets.
%		The consistency axiom is also referred to as \say{reinforcement}.
%	\end{genthm}
%	\commentOC{You didn’t define $f$ as a rule operating on variable populations. 
%	In other words, $f$ is supposed to apply only on $N$-sized profiles.
%	So, there is a type problem about $f(P)$ and $f(P')$.
%	I suggest to drop this axiom if you don’t need it very much; or you’d have to introduce such kind of rules.}
%** Fallback yes, q<n no
%
%	\begin{tabular}{ccccccccccc}
%	&	Anon.	&	Neut.	&	Nondict.	&	Unan.	&	Pareto	&	Condorcet	&	Majority &	Manipul.	&	SP	&	Consist.	\\
%	Majority	&	$\checkmark$	&	$\checkmark$	&	$\checkmark$	&	$\checkmark$	&	$\checkmark$	&	$\bm{\times}$	&	$\checkmark$	&	$\checkmark$	&	$\bm{\times}$	&	$\checkmark$	\\
%	Plurality	&	$\checkmark$	&	$\checkmark$	&	$\checkmark$	&	$\checkmark$	&	$\checkmark$	&	$\bm{\times}$	&	$\checkmark$	&	$\checkmark$	&	$\bm{\times}$	&	$\checkmark$	\\
%	Anti-Plurality	&	$\checkmark$	&	$\checkmark$	&	$\checkmark$	&	$\checkmark$	&	$\bm{\times}$	&	$\bm{\times}$	&	$\bm{\times}$	&	$\checkmark$	&	$\bm{\times}$	&	$\checkmark$	\\
%	Borda	&	$\checkmark$	&	$\checkmark$	&	$\checkmark$	&	$\checkmark$	&	$\checkmark$	&	$\bm{\times}$	&	$\bm{\times}$	&	$\checkmark$	&	$\bm{\times}$	&	$\checkmark$	\\
%	Condorcet	&	$\checkmark$	&	$\checkmark$	&	$\checkmark$	&	$\checkmark$	&	*	&	$\checkmark$	&	$\checkmark$	&	$\checkmark$	&	$\bm{\times}$	&	$\bm{\times}$	\\
%	BK Compromises	&	$\checkmark$	&	$\checkmark$	&	$\checkmark$	&	$\checkmark$	&	$\checkmark$	&	$\bm{\times}$	&	$\checkmark$	&	$\checkmark$	&	$\bm{\times}$	&	**	\\
%	PSR	&	$\checkmark$	&	$\checkmark$	&	$\checkmark$	&	$\checkmark$	&	$\checkmark$	&	$\bm{\times}$	&	$\bm{\times}$	&	$\checkmark$	&	$\bm{\times}$	&	$\checkmark$	\\
%	
%\end{tabular}
%\begin{center}
%	$ P1:
%	\begin{array}{ccc}
%		4 & 4 & 2 \\
%		a &	b & c\\
%		b &	a & d\\
%		c &	c & a\\
%		d & d & b 
%	\end{array} \quad, 
%	$
%	$ P2:
%	\begin{array}{ccc}
%		2 & 1 \\
%		a &	c\\
%		b &	b\\
%		c &	a\\
%		d & d 
%	\end{array} , 
%	$
%\end{center}
%with $q= \ceil{\frac{n}{2}}$ then $f(P1)=\{a,b\}, f(P2)=\{a\}, f(P1 \cup P2)=\{a,b\} \neq f(P1)\cap f(P2)=\{a\}$


%\subsection{Incompleteness}
%\commentBN{Add later if needed and time permitting.}


\section{Voting with rated ballots}
\label{sec:judgmentballots}

In the classical Arrovian framework for social choice a set of voters is assumed to give a complete preferences order over a set of alternatives \citep{Arrow1950}.
When we think about it this is not so obvious to achieve, for example, often there are too many alternatives and the cost for the voters to provide a complete order is simply too high.
In this situation, one could opt for an approval system, in which voters approve only the desired alternatives \citep{Brams2007}. Yet this does not resolve other criticisms raised against the system of ranked ballots: that of preference intensity. 
Suppose that a slight majority of voters slightly prefer alternative $a$ to alternative $b$, these would appear in the preference rankings as $a \pref b$. Now consider that the remaining slight minority despises $a$, and strongly prefers $b$ to $a$. There is no way for any rule in the framework of ranked preferences to distinguish the difference in intensity between $a \pref b$ in the first group and $b \pref a$ in the second. How to decide what is the most desirable alternative for society?

An alternative is to evaluate candidates according to a common language shared by society. Note that from the evaluation of candidates we can deduce a ranking, but the converse is not true. Thus, an evaluation is more informative than a ranking.
However, we immediately see that we should define lower and upper bounds on the score a voter could assign to avoid problems of exaggeration.
This describes the framework introduced by \citet{Smith2000}, and later analyzed by \citet{Pivato2014,Gaertner2012,Zahid2015}, when proposing his \textit{Range Voting} rule.
Here the common language consists of a range of numerical scores, for example from 0 to 100, and the rule works as follows: each voter assigns a score to all or some alternatives, the one with the highest average score is elected. If more than one alternative has the same highest score than the tie is broken randomly.

But a whole bunch of new questions arise. What is the right size for this scale? If it is too large it creates manipulation problems because voters may exaggerate their preferences; if it is not large enough it reduces expressiveness. Is it better to use a numerical scale or a more immediate language? Are there really benefits to using ratings instead of rankings? 
Luckily for us, there is a vast literature of economists and psychologists who have addressed these questions, even sometimes arriving at very different answers. Unfortunately, to discuss this would go beyond the scope of this thesis, but for a (non-exhaustive) list of work on the subject one can read \citet{Cox1980,Sparling2011,Churchill1984,Preston2000,Maio1996}.

\citet{Balinski2007} settle for a scale of seven expressions when defining their \textit{\acl{MJ}} rule: \textit{Excellent, Very Good, Good, Passable, Inadequate, Mediocre, Bad}. Each voters judges all the candidates and the one with the highest median is elected. We describe this rule in more detail in the next section.
For now, we would like to conclude our introduction with a small anecdote, a piece from a 2012 interview with Arrow in which, when asked how he thought voting systems should be evaluated in the future, he replied: \say{I’m a little inclined to think that score systems where you categorize in maybe three or four classes probably (in spite of what I said about manipulation) is probably the best} \citep{PodcastArrow}.

%https://voteupapp.com/

\subsection*{Majority judgment}
	\label{sec:MJ}
	\acrlong{MJ} (\ac{MJ}) is a voting method proposed by \citet{Balinski2007,Balinski2011} to elect one out of $m$ candidates based on the judgments of $n$ voters. The latter express their preferences by assigning to each candidate one of the following adjectives: Excellent, Very good, Good, Average, Mediocre, Inadequate, To be rejected. Those adjectives represent a common language whose semantic is assumed to be a shared knowledge among the voters carrying thus an absolute meaning. For each candidate the median of the grades she received is computed, this is called \textit{majority-grade}. The candidate with the highest majority-grade is elected. Ties are broken by considering the majority-grade of first order: one vote associated with the majority-grade of each tied candidates is removed and their medians are recomputed. The candidate with the highest new median is elected. If there is still a tie the process is repeated until a unique winner is found.
	
	In order to formalize this description, consider a finite set $N$ of voters (or judges) with $\#N=n$ and a finite set $A$ of alternatives (or competitors) with $\#A=m$. 
	A \textit{common language} $\triangle = \{ \delta_1, \delta_2, \dots , \delta_7\}$ is a set of strictly ordered grades. The notation $\delta_1 \geq \delta_2$ indicates that $\delta_1$ is a better or equivalent grade than $\delta_2$. Here $\delta_1$ corresponds to Excellent, $\delta_2$ to Very good etc.
	
	A profile $P : A\times N \rightarrow \triangle$ is a $m$ by $n$ matrix of grades $P \in \triangle^{A \times N}$.
	
	The operator $\rho: \triangle^{N} \rightarrow \triangle^{\card{N}}$ defines an ordering function that given a vector of grades $P_i$ returns the vector ordered by decreasing grades.
	
	We define $f: \triangle^{N} \rightarrow \triangle$ a function that assigns to any vector of grades a final grade. A grading function $f^A: \triangle^{A \times N} \rightarrow \triangle^A$ returns a vector of final grades by applying $f$ to every alternative in $A$.
	
	The \emph{majority-grade}, $f_\text{maj}$, is the function that associates to a vector of grades $q \in \triangle^{N}$ its median grade value: $f_\text{maj}(q) = \rho(q)_{\floor{\frac{\card{q}}{2}} + 1}$. Note that in case $\card{q}$ is even two medians could be used, but in \citet{Balinski2011} definition the lower grade is picked. By applying $f_\text{maj}$ to all vectors of grades associated to the alternatives in $A$ we obtain the corresponding grading function $f^A_\text{maj}$. Formally, $f^A_\text{maj}(P)_i = f^A_\text{maj}(P_i)$. The $i$-\emph{th} element of the resulting vector is the median of the ordered vector of grades associated to the $i$-\emph{th} alternative. 
	
	The \ac{MJ} function $F_\text{maj}:\triangle^{A \times N} \rightarrow A$	is a function that selects the alternative with the highest median grade as winner. We can define it as $F_\text{maj}(P) = \argmax_{i\in A}\rho(P_i)_{\floor{\frac{n}{2}}+ 1}$ assuming it is a singleton. Consider the case when it is not a singleton, this means that two or more alternatives are associated with the same highest median grade $h=\max_{i\in A}\rho(P_i)_{\floor{\frac{n}{2}} + 1}$. In this case ties are broken by removing one $h$ grade from the vectors of grades of each tied alternative, recomputing the new median grade and repeating the process until one unique winner is found or there are no more grades to remove. When $n$ is odd, this is equivalent to take the next element after the median, i.e. the one at index $(\floor{\frac{n}{2}} + 1) +1$, if there is still a tie we then look at the previous element before the median, i.e. the one at index $(\floor{\frac{n}{2}} + 1) -1$, and keep alternating until the tie is broken or there are no more elements in the vector. When $n$ is even the process is similar but we alternate starting from the element before the median, $\floor{\frac{n}{2}}$, then the one after, $\floor{\frac{n}{2}} + 2$, and so on. If after applying the mechanism there are still ties we break them using an arbitrary ordering defined on all alternatives, e.g. lexicographical order.
	
	Considering the axioms aforementioned, \ac{MJ} satisfies anonymity, neutrality, nondictatoriality but it fails manipulability and it does not satisfy the majority criterion.
	\begin{example}[\citet{Laslier2018}]
		\label{ex:MJfailsMajority}
		To show that \ac{MJ} is not majoritarian, consider the following profile formed by three voters $i_1, i_2, i_3$ and two candidates $a, b$. Suppose the voters assign to candidates a grade between 0 and 20.
		\begin{center}
			$
			\begin{array}{cccc}
				&\mathbf{i_1} & \mathbf{i_2} & \mathbf{i_3} \\
				a \quad &20&9&9\\
				b \quad &11&0&10\\
			\end{array}\quad .
			$
		\end{center}
		The median of $a$ is $9$ and the median of $b$ is $10$, thus $b$ is elected although only one voter slightly prefers $b$ to $a$. This example can be reproduced with \ac{MJ} grades and with a larger number of voters:
		\begin{center}
			$
			\begin{array}{cccc}
				&\mathbf{50} & \mathbf{50} & \mathbf{1} \\
				a \quad &\text{Excellent}&\text{Mediocre}&\text{Mediocre}\\
				b \quad &\text{Good}&\text{To be rejected}&\text{Average}\\
			\end{array}\quad .
			$
		\end{center}
	\end{example}

	Is it worth mentioning that the same example is also studied by \citet[p. 281]{Balinski2011}, where the authors stress the fact that is very unlikely that the last voter would associate a different grade to 9 and 10. In a large electorate the distinction would be too fine to make a significant difference between 20 and 19 or 10 and 9, so they affirm that the example would more realistically translate to: 
	\begin{center}
		$
		\begin{array}{cccc}
			&\mathbf{50} & \mathbf{50} & \mathbf{1} \\
			a \quad &\text{Excellent}&\text{Mediocre}&\text{Mediocre}\\
			b \quad &\text{Average}&\text{To be rejected}&\text{Mediocre}\\
		\end{array}\quad .
		$
	\end{center}
	And $a$ would be preferred.

	Regarding manipulability, \ac{MJ} is manipulable to the extent that voters can exaggerate by excess the judgments of their top choices and by defect those of their worst choices. It must be mentioned that \citet{Bassett1999} proved that the high breakdown property of the median\textemdash which is the sensitivity to outlying observations \textemdash makes difficult for a minority to manipulate the ranking. However, \citet{Gehrlein2003} studied that the probability of being subject to manipulation is just slightly smaller than other methods like Borda and Copeland.










