%!TeX root= ../thesis.tex

\begin{abstract}
A social choice rule aggregates the preferences of a group of individuals over a set of alternatives into a collective choice. The literature admits several social choice rules whose recommendations are supposed to reflect a compromise among individuals. We observe that all these compromise rules can be better described as \emph{procedural compromises}, i.e., they impose over individuals a willingness to compromise but they do not ensure an outcome where everyone has effectively compromised. We revisit the concept of a compromise in a collective choice environment with at least three individuals having strict preferences over a finite set of alternatives. Referring to a large class of spread measures, we view the concept of compromise from an \emph{equal loss} perspective, favoring an outcome where every voter concedes as equally as possible. As such, being a compromise may fail Pareto efficiency, which we ensure by asking voters to concede as equally as possible among the Pareto efficient alternatives. We show that Condorcet consistent rules, scoring rules (except antiplurality) and Brams-Kilgour compromises (except fallback bargaining) all fail to ascertain an outcome which is a compromise. A slight restriction on acceptable spread measures suffices to extend the negative result to antiplurality and fallback bargaining.
This failure also prevails for social choice problems with two individuals: all well-known two-person social choice rules of the literature, namely, fallback bargaining, Pareto and veto rules, short listing and veto rank, fail to pick ex-post compromises. We conclude that there is a need to propose and study rules that satisfy this equal loss, or outcome oriented, notion of a compromise.
\end{abstract}

\section{Introduction}
\label{sec:introduction}
In a typical social choice problem, several individuals express their preferences over a set of alternatives and one shall be picked as the collective outcome. Although the literature admits several \acp{SCR} with different properties, there is a common understanding that collective choices must reflect “compromises”. One of the first to explicitly refer to a \ac{SCR} as a compromise is \citet{Sertel1986} introducing the \emph{majoritarian compromise}. This \ac{SCR}, further analysed by \citet{Sertel1999}, is a rediscovery of a method suggested by James W.\ Bucklin at the beginning of the \nth{20} century \citep{Erdelyi2015}. Starting from everyone’s ideal alternative, it falls back to the voters’ second, third and more generally $k$-\emph{th} best, until one of the alternatives considered appears among the first $k$ best for a majority. \citet{Brams2001} generalise this concept and introduce a class of \acp{SCR} called $q$\emph{-approval fall-back bargaining}, where $q$ is the required quantity of support that can vary from a single voter up to unanimity. Different choices of $q$ lead to different \acp{SCR}. Considering $n$ voters, the choice of $q=1$ corresponds to the plurality rule, $q=\ceil{\frac{n}{2}}$ is closely related, but not identical, to the majoritarian compromise and $q=n$ represents a bargaining procedure called \emph{fall-back bargaining}, which has been further analysed by \citet{Kibris2007} and \citet{Congar2012}. We will refer to these rules as \ac{BK} compromises with threshold q.

As \citet{OezkalSanver2004} discuss, the concept of compromising is mostly understood as the trade off between the number of voters supporting an alternative (i.e., the quantity of support) and the distance of that alternative from the supporters’ ideal alternative (i.e., the quality of support). This trade off, which is explicit for $q$-approval fall-back bargaining, is also the basis for several other \acp{SCR} such as the \emph{median voting rule} proposed by \citet{Bassett1999} and further analysed by \citet{Gehrlein2003} or the \emph{Condorcet practical method} described by \citet{Nurmi1999}. \Citet{Merlin2019} identify and analyse a large class of \emph{compromise rules} that explore this trade off.

One can argue that a collective choice \emph{per se} implies a compromise. After all, except extreme cases such as dictatorships,
a \acp{SCR} operates on the principle that all voters could fall back from their ideal position. Whether all voters effectively do fall back and whether this fall is “equal among voters” is the subject of our analysis. In what follows, we will present examples where they do not, which could be considered counter to the spirit of compromising. \footnote{This objection to the compromise nomenclature was raised by Jean-François Laslier during a CNRS workshop on compromising hosted by Istanbul Bilgi University in 2018.}

Consider the following example.
\begin{example}
	\label{ex:ex1}
	Let $N$ be a set of $n ≥ 3$ voters and $A$ a set of alternatives. $\linors$ represents the set of linear orders over $A$. Consider the following preference profile $P\in \linors^{N}$,
	\begin{center}
		$
		\begin{array}{cccc}
			\mathbf{1} \quad &c&b&a\\
			\mathbf{n-1} \quad &a&b&c\\
		\end{array}\quad ,
		$
	\end{center}
	which represents one individual who prefers $c$ to $b$, $b$ to $a$, hence $c$ to $a$; and $n-1$ individuals who prefer $a$ to $b$, $b$ to $c$, hence $a$ to $c$. At $P$, all \acs{BK} compromises, except when $q=n$ (i.e. fall-back bargaining) and $q=1$, will ignore the single voter and will pick $a$ as the collective outcome.
\end{example}

As a matter of fact, almost every \ac{SCR} will ignore this “marginal minority” and choose $a$ in this situation. While this choice is defensible on the grounds of qualified majoritarianism, it is questionable whether $a$ can be qualified as a compromise. Observe that $b$ receives unanimous support when each voter falls back one step from his ideal point. The question becomes more compelling when $a$ remains the collective choice even if the ignored group is much larger.

\commentBN{This example does not hold anymore because BK would select $\{a,c\}$ so I slightly modified the text.}
\begin{example}
	\label{ex:ex2}
	Consider the following preference profile with $n=100$:
	\begin{center}
		$
		\begin{array}{cccc}
			\mathbf{49} \quad &c&b&a\\
			\mathbf{51} \quad &a&b&c\\
		\end{array} \quad.
		$
	\end{center}
	When $q\in \intvl{1,\frac{n}{2}+1} $, all \acs{BK} compromises pick $\{a,c\}$, and, again, it does not appear as a compromise as almost half of voters reach their best alternative while the remaining half have to be contented with their worst one.
\end{example}

Observe that all these \acp{SCR} impose to voters a willingness to compromise, but do not effectively ensure a compromise as the collective choice. In fact, the term “compromise” in this literature refers to procedural compromises that differ from outcome oriented compromises, a conceptual distinction that seems to be overlooked in the literature.
This can also be viewed as a distinction between ex-ante and ex-post compromises, where the profile is the source of uncertainty.

To define an ex-post compromise, we adapt a concept of equal losses that considers allocation of continuous utilities. This principle is used for bargaining \citep{Chun1988, Chun1991} and bankruptcy problems \citep{Herrero2001}. 
We introduce two definitions of compromise. In both of them, we pick a spread measure that determines how equally a given vector of numbers is distributed and we propose to make a collective choice where voters give up from their ideal points “as equally as possible”. The difference between the two is that one, called \emph{egalitarian compromise}, insists on equality at the expense of Pareto efficiency while the other, called \emph{Paretian compromise}, is constrained to pick among the Pareto efficient alternatives. We show that Pareto efficient \acp{SCR} cannot ensure egalitarian compromises under any spread measure. We prove that several well-known \acp{SCR} such as Condorcet extensions, scoring rules, $q$-approval fall-back bargaining, all fail to be Paretian compromises under any spread measure. We provide examples for which being a Paretian compromise would necessitate to pick an alternative that is, although Pareto optimal, ranked very low by all voters. Such alternative would never be picked by any of the above-mentioned \acp{SCR}. 

We conclude that the equal-loss principle appears adequate for collective choice problems with at least three individuals, when egalitarianism, in the sense of conceding equally, is a major concern. Imagine a situation where the head of a laboratory needs to decide which project to fund and she asks for the preferences of the laboratory members. The workplace harmony is extremely important and, in order to avoid conflicts, the winning project must be equally supported by all members. 

Consider now a situation with only two voters. As the vast literature on the ultimatum game \citep{Werner2014} suggests, mutual consent is hard to obtain when one individual sees injustice at the levels of mutual losses. The equal-loss principle seems to be crucial in this new scenario.

Collective choice models with two individuals can be interpreted as bargaining or arbitration problems. While the bargaining interpretation necessitates an explicitly defined disagreement outcome \citep{Kibris2007}, the arbitration interpretation \citep{Sprumont1993} remains within the classical collective choice environment with no explicit disagreement outcome. In this paper, we consider the latter interpretation. 

Arbitration rules are thoroughly discussed by \citet{Barbera2020}. As prominent examples, we have fallback bargaining proposed by \citet{Brams2001}, the veto-rank and short listing procedures analysed by \citet{Clippel2014} and the Pareto-and-veto rules analysed by \citet{Laslier2020}. These models consider discrete alternatives which are not contained by the classical \citet{Nash1950} bargaining environment with convex utilities. We make the same assumption. However, as \citet{Mariotti1998} and \citet{Nagahisa2002} illustrate, the two worlds can be interconnected, as we do for the equal-loss principle of \citet{Chun1988} and \citet{Chun1991}. The arbitration environment presents an instance where the equal-loss principle could matter and it is rather surprising to discover that most interesting \acp{SCR} used as arbitration solutions fail to be Paretian compromises. 

The rest of the paper is organised as follows. \Cref{sec:notation} presents the basic notions and notation. \Cref{sec:compromise} introduces egalitarian compromises and Paretian compromises, two concepts that turn out to be logically incompatible. \Cref{sec:more2voters} shows that with at least three individuals, many \acp{SCR} fail to pick a compromise. \Cref{sec:2voters} considers the two-individual case, showing that most \acp{SCR} of the literature fail to pick compromises. \Cref{sec:conclusion} makes some concluding remarks. 

\section{Basic notions and notation}
\label{sec:notation}
Consider a finite set $N$ of individuals with $\#N=n\geq 2$ and a finite set $A$ of alternatives with $\#A=m\geq 3$. We write $\linors$ for the set of linear orders over $A$.
A generic element $\prefi$ of $\linors$ stands for a preference of $i\in N$. This implies that, given any $x ≠ y\in A$, precisely one of $x \prefi y$ and $y\prefi x$ holds while $x \prefi x$ holds for no $x\in A.$ Moreover, $x\prefi y$ and $y\prefi z$ implies $x\prefi z$ $\forall x,y,z\in A$.

A \emph{profile} $P: N → \linors$ associates with each individual $i \in N$ a preference $P(i) = {\prefi}$. A \emph{\acl{SCR}} (\acs{SCR}) is a mapping $f:\linors^{N} \rightarrow 2^{A} \setminus \{\emptyset \}$. 

We write $r_{\prefi}(x)=\#\{y\in A \suchthat y \prefi x\}+1$ for the \emph{rank} of $x\in A$ at ${\prefi} \in \linors$. We denote by $\lambda_{\prefi}(x)=r_{\prefi}(x)-1$ the loss in terms of ranks for $i\in N$ with preference $\prefi$, when $x$ is elected instead of the best alternative
for $i$. The mapping $\lambda_P: A → \alllosses$ assigns to each $x\in A$ the loss vector $\lambda_{P}(x)=(\lambda_{\prefi}(x))_{i\in N}$ induced by the election of $x$. The double brackets denote intervals in the integers.

We are interested in measuring the spread of loss vectors. To this end, we define a \emph{spread measure} $\sigma: \alllosses → \R_{+}$ as a function that associates a spread value to every possible loss
vector. We write $\Sigma$ for the set of spread measures $\sigma$ that satisfy, for every $l\in\alllosses$, $\sigma(l)=0 ⇔ l_{i}=l_{j}$ $\forall i,j\in N$. Thus, the spread of $l$ gets its lowest value $0$ in case of perfect equality and only in this case. Note that this condition incorporates the minimal requirement to identify a spread measure and leads to the largest set of spread measures one could define. As discussed in \Cref{sec:RestrictionOnSigma}, this flexibility has the advantage of making our results more general.

Given any distinct $x,y\in A$, we say that $x$ \emph{Pareto dominates} $y$ at $P \in\linors^{N}$ (or equivalently $y$ is \emph{Pareto dominated} by $x $ at $P$) iff $x\prefi y,\forall i\in N$. We denote by
$\paretopt(P)= \set{x \in A \suchthat \forall y \in A\setminus\{x\}, \exists i \in N \suchthat x \pref_i y}$ the set of \emph{Pareto optimal} alternatives at $P$.
A \ac{SCR} $f$ is \emph{Paretian} iff $f(P)\subseteq\paretopt(P)$ $\forall P\in\linors^{N}$.

\section{Egalitarian versus Paretian compromises}
\label{sec:compromise}
\subsection{Egalitarian compromises}
\label{sec:EgCompromise}
We let $\argmin_{X}(\sigma \circ \lambda_P) = \set{x \in X \suchthat \allowbreak{}\forall y \in X: \sigma(\lambda_P(x)) ≤ \sigma(\lambda_P(y))}$ denote the minimal elements of $X \subseteq A$ according to $(\sigma \circ \lambda_{P})$. In other words, $\argmin_{X}(\sigma\circ\lambda_{P})$ denotes the alternatives in X whose loss vectors are the most equally distributed according to the spread measure $\sigma$.

In what follows, we define some classes of \acp{SCR} that we are interested in analysing. 


\begin{definition} A \ac{SCR} $f$ is an \emph{Egalitarian Compromise} (EC) iff \[\exists \sigma \in \Sigma \suchthat \forall P \in \linors^N \text{ we have }f(P) \subseteq \musigma.\]
\end{definition}

\begin{definition} A \ac{SCR} $f$ is \emph{Egalitarian Compromise Compatible} (ECC) iff \[\exists \sigma \in \Sigma \suchthat \forall P \in \linors^N \text{ we have } f(P) \cap \musigma \neq \emptyset.\]
\end{definition}

Under a \ac{SCR} that is EC (resp., ECC), \emph{all} (resp., \emph{some}) winners are among the alternatives with most equally distributed losses. Clearly, EC is a subclass of ECC. Perhaps less obviously, being ECC (or EC) is incompatible with being Paretian. This will be deduced from the following proposition, which will also be useful to prove other theorems.% \cref{th:incompatibility}.

\begin{proposition} \label{prop:muSigmaLast}
	For $n ≥ 2, m ≥ 3$, there exists a profile $P \in \linors^N$ and an alternative $a_m$ such that $\forall i \in N$: $r_{\prefi}(a_m)=m$, and such that $\forall \sigma \in \Sigma: \musigma = \set{a_m}$; hence, $\musigma \cap \paretopt(P) = \emptyset$.
\end{proposition}
\begin{proof}
	Consider the following profile $P$:
	\begin{center}
		$
		\begin{array}{cccccc}
			\mathbf{1} \quad &a_1&a_2&\dots&a_{m-1}&a_m\\
			\mathbf{n-1} \quad &a_{\pi_{(1)}}&a_{\pi_{(2)}}&\dots&a_{\pi_{(m-1)}}&a_m\\
		\end{array}
		$ \quad,
	\end{center}
	where $\pi$ is the following permutation over $\intvl{1, m-1}$:
	\[
	\pi(i) = 
	\begin{cases}
		i+1 & \text{if } i \in \intvl{1, m-2} \\
		1 & \text{if } i = m-1
	\end{cases} \quad .
	\]
	In $P$, $a_m$ is the only alternative such that $r_{\prefi}(a_m)$ is independent of $i$; hence, $\sigma(\lambda_P(a)) > 0$, $\forall a \in A\setminus \{a_m\}$, $\forall \sigma \in \Sigma$. Thus, the set $\musigma$ consists of the sole element $a_m$, and, because $a_m$ is Pareto dominated, $\musigma \cap \paretopt(P) = \emptyset$.
\end{proof}

Our main result for \cref{sec:EgCompromise} follows easily.
\begin{theorem} \label{th:nonParetian}
	For $n\geq 2, \ m\geq3$, no Paretian \ac{SCR} is ECC.
\end{theorem}
\begin{proof}
	Proving this amounts to show that $\forall \sigma \in \Sigma, \exists P \in \linors^N \suchthat \paretopt(P) \cap \musigma = \emptyset$. Suffices to use \cref{prop:muSigmaLast}, which asserts that there exists a profile $P$ such that $\forall \sigma \in \Sigma: \musigma \cap \paretopt(P) = \emptyset$.
\end{proof}

\subsection{Paretian compromises}
Having seen the tension for a \ac{SCR} being Paretian and ECC, we investigate the consequences of inverting the order of priorities by insisting that at least some of the winning alternatives are Pareto optimal, and considering the most equally distributed loss vectors among those.

We consider two classes of \acp{SCR}. 
Observe that $\mustar$ denotes the set of Pareto optimal alternatives whose loss vectors are the most equally distributed according to the spread measure $\sigma$.

\begin{definition} A \ac{SCR} $f$ is a \emph{Paretian Compromise} PC iff \[\exists \sigma \in \Sigma \suchthat \forall P \in \linors^N \text{ we have } f(P) \subseteq \mustar.\]
\end{definition}

\begin{definition} A \ac{SCR} $f$ is \emph{Paretian Compromise Compatible} PCC iff \[\exists \sigma \in \Sigma \suchthat \forall P \in \linors^N \text{ we have } f(P) \cap \mustar \neq \emptyset.\]
\end{definition}

Again, it is clear that PC is a subclass of PCC. It will also probably come with no surprise that for a \ac{SCR}, being PC is incompatible with being ECC, as being PC requires to be Paretian, which permits to use \cref{th:nonParetian}. On the other hand, it is less immediate that being EC is incompatible with
being PCC, because being PCC does not require to be Paretian. This is however true.

\begin{theorem} \label{th:incompatibility} 
	For $n ≥ 2, m ≥ 3$, no \ac{SCR} is both EC and PCC.
\end{theorem}
\begin{proof}	
	Considering the profile $P$ of \cref{prop:muSigmaLast}, with $a_m$ the alternative mentioned there, any EC $f$ and any $\sigma \in \Sigma$, suffices to prove that $f(P) \cap {\mustar[\sigma][P]} = \emptyset$.
	
	First, from \cref{prop:muSigmaLast} we have that
	$\set{a_m} \cap \paretopt(P) = \emptyset$, hence $\set{a_m} \cap$ \break $ {\mustar[\sigma][P]} = \emptyset$. 
	
	Second, because $f$ is EC, for some $\sigmatop$, $f(P) \subseteq {\musigma[\sigmatop][P]}$. Using \cref{prop:muSigmaLast} again, we see that ${\musigma[\sigmatop][P]} = \set{a_m}$, hence $f(P) = \set{a_m}$.
	
	That $f(P) \cap {\mustar[\sigma][P]} = \emptyset$ follows from these two facts.
\end{proof}

It is interesting to note that the incompatibility is not complete, however.

\begin{remark}
	For $n ≥ 2$, $m ≥ 3$, there exist \acp{SCR} that are both ECC and PCC, such as the \ac{SCR} that selects the whole set of alternatives at every profile. However, this \ac{SCR} fails to be Paretian, as is any \ac{SCR} that is ECC.
\end{remark}


\section{Which \acp{SCR} are compromises?}
\label{sec:more2voters}
In this section we assume $n\geq 3$ and leave the analysis of $n=2$ to the
next section.

\subsection{Condorcet consistent rules}

An alternative $x\in A$ is a \emph{Condorcet winner} at $P\in \linors^N$ iff for all $y\in A \setminus \set{x} $, $\#\set{i \in N \suchthat x \prefi y} >\#\set{i \in N \suchthat y \prefi x}$. So each profile admits
either no or a unique Condorcet winner. An \ac{SCR} $f$ is \emph{Condorcet
	consistent} iff $f(P)=$ $\left\{ x\right\} $ at each $P\in \linors^N$ that
admits $x$ as the (unique) Condorcet winner.

\begin{theorem} \label{th:condorcet}
	Let $n\geq 3$ and $m\geq 3$. A Condorcet consistent \ac{SCR} $f$ is neither ECC nor PCC.
\end{theorem}
\begin{proof}
	Consider the following profile $P$, where the dots represent the sequence $a_4$ to $a_m$:
	\begin{center}
		$
		\begin{array}{cccccc}
			\mathbf{n-1} \quad &a_1&a_2&a_3&\dots\\
			\mathbf{1} \quad &a_3&a_2&\dots&a_1\\
		\end{array}
		$ \quad.
	\end{center}
	
	Consider any Condorcet consistent \ac{SCR} $f$. Thus, $f(P)=\{a_1\}$. However, $\musigma=\mustar=\{a_2\}$ $\forall \sigma \in \Sigma$, so there exists a profile $P$ such that both $f(P)\cap \musigma$ and $f(P)\cap \mustar$ are empty.
\end{proof}

Note that Condorcet consistent rules need not be Paretian so the fact that they all fail ECC does not follow from \cref{th:nonParetian}. 

\subsection{Scoring rules}
\label{sec:scoringrules}
A \emph{score vector} is an $m-$tuple $w=(w_{1},\dots,$ $w_{m})\in \intvl{0, 1}^{m}$ with $w_{1}=1$, $w_{m}=0$ and $w_{i}\geq w_{i+1}$ $\forall i\in \intvl{1, m-1}$. Given a score vector $w$, we write $s^{w}(x,P)=\sum_{i\in N}w_{r_{\prefi}(x)}$ for the score of $x\in A$ at $P \in \linors^N$. Each vector $w$ identifies a \emph{scoring rule} $f^w_n$ defined as $f^w_n(P)=\left\{ x\in A:s^{w}(x,P)\geq s^{w}(y,P) \ \forall y\in A\right\}$ for every $P \in \linors^N$.

We first show that no scoring rule is ECC, for any value of $n$ and $m$ at least 3.

\begin{theorem}\label{th:srECC}
	Let $n\geq 3$ and $m\geq 3.$ No score vector $w$ induces a scoring rule $f^w_n$ that is ECC.
\end{theorem}
\begin{proof}
	Take any score vector $w$. Consider the profile $P$ of \cref{prop:muSigmaLast}. Observe that $\musigma=\{a_m\} \ \forall \sigma \in \Sigma $. However, as $w_{1}>w_{m}$, we have $s^{w}(a_{1},P)>s^{w}(a_{m},P)$ which implies $a_{m}\notin f^{w}(P)$.
\end{proof}

We call antiplurality score vector the vector $w$ such that $w_{i} = 1, \forall i \in \intvl{1, m-1}$, and $w_{m}=0$.

\begin{theorem}
	\label{th:AntSatsPCC}
	Let $m\geq 3$ and let $w$ be the antiplurality score vector. The \ac{SCR} $f_{n}^{w}$ satisfies PCC for all $n\geq 3$.
\end{theorem}
\begin{proof}
	Define $\bar\sigma \in \Sigma$ as, $\forall l \in \intvl{0,m-1}^N$: $\bar\sigma(l) = 1$ iff $\exists i, j \in N \suchthat l_i ≠ l_j$; $\bar\sigma(l) = 0$ otherwise.
	We show the non-emptyness of $f^w_n(P) \cap \mustar[\bar\sigma]$ for any profile $P$.
	
	Let $k = \min_{\paretopt(P)} \set{(\bar\sigma \circ \lambda_P)(x)}$ be the minimal value attained by $\bar\sigma \circ \lambda_P$ over $\paretopt(P)$. By construction of $\bar\sigma$, $k$ equals either $0$ or $1$.
	
	For $k = 1$, take any $x \in f^w_n(P) \cap \paretopt(P)$. This intersection is non-empty because whenever the antiplurality rule picks a Pareto dominated alternative $z$, it also picks all alternatives which Pareto dominate $z$.
	By definition of $\bar\sigma$, $\bar\sigma(x) ≤ 1$, hence, $x \in \mustar[\bar\sigma]$.
	%	We then have that $x \in \mustar[\bar\sigma]$ as by definition of $\bar\sigma$, $\bar\sigma(x) ≤ 1$.
	
	For $k = 0$, take any $x \in \mustar[\bar\sigma]$. As $\bar\sigma (\lambda _{P}(x))=0$, we have, $\forall i, j \in N$: $\lambda_i^P(x) = \lambda_j^P(x)$, hence, $\forall i, j \in N$: $r_{\succ_i}(x) = r_{\succ_j}(x)$. 
	The case $r_{\succ_i}(x) = m, \forall i \in N$ is ruled out by $x \in \paretopt(P)$. Hence, $r_{\succ_i}(x) ≤ m - 1, \forall i \in N$, hence, $x \in f^w_n(P)$.
\end{proof}

It is worth noting that the antiplurality rule $f_{n}^{w}$ is not Paretian, hence fails PC  for all $n\geq 3$. This can be seen by picking a unanimous profile $P \in \linors^{N}$ with $a_{1}\prefi a_{2}\prefi \dots \prefi a_{m}$ $\forall i\in N$, where $\mustar=\left\{ a_{1}\right\} \forall \sigma \in \Sigma $ while $f_{n}^{w}(P)=A \setminus \left\{ a_{m}\right\}$.

\begin{theorem}
	\label{th:srPCC}
	Let $m\geq 3.$ Take any score vector $w$ which is not the antiplurality score vector. For some $n ≥ 3$, the \ac{SCR} $f_{n}^{w}$ fails PCC.
\end{theorem}

\begin{proof}
	Take any $m\geq 3$ and any score vector $w$ that is not the antiplurality score vector; therefore, $w_{m-1}<1$. Pick any $n$ such that $n ≥ m - 1$ and $n > \frac{1}{1 - w_{m - 1}}$. Consider a profile $P \in \linors^N$ conforming to
	
	\begin{center}
		$
		\begin{array}{cccccc}
			i = 1 \quad & a_2 & … & a_m & a_1\\
			2 ≤ i ≤ m - 2 \quad & a_1 & … & a_m & a_i\\
			m - 1 ≤ i ≤ n \quad & a_1 & … & a_m & a_{m-1}\\
		\end{array}
		$\quad,
	\end{center}
	where all alternatives except $a_m$ appear at least once in the last rank.
	Thus, for every $\sigma \in \Sigma$, we have 
	$\sigma (\lambda _{P}(x))>0$ $\forall x\in A \setminus \left\{ a_{m}\right\}$
	while
	$\sigma (\lambda_{P}(a_{m}))=0$. 
	Moreover, $a_{m}\in \paretopt(P)$. Thus, $\mustar=\left\{ a_{m}\right\} $ $\forall \sigma \in \Sigma $. On the other hand, $s^{w}(a_{1}; P)=n-1$, $s^{w}(a_{m}; P)=n\cdot w_{m-1}$ and
	as $n > \frac{1}{1 - w_{m - 1}}$ (or, equivalently, $n - 1 > n w_{m - 1}$), we have $s^{w}(a_{1}; P)>s^{w}(a_{m};$ $P)$,
	establishing $a_{m}\notin f^{w}(P)$, thus $f^{w}(P)\cap \mustar=\emptyset $ $\forall \sigma \in \Sigma $.
\end{proof}

\subsection{\acs{BK} compromises}
\label{sec:BKn3}
Given any $k\in \intvl{1, m}$, we write $n_{k}(x,P)=\#\{i\in
N\mid r_{\prefi}(x)\leq k\}$ for the \emph{$k$-support} that $x$ gets at $P$, that is, the number of individuals for whom the rank of alternative $x\in A$ is lower than or equal to $k$ in the profile $P \in \linors^N$.
Note that $n_{k}(x,P)\in \intvl{1, n}$ is non-decreasing on $k$ and $n_{m}(x,P)=n.$ For each $q\in \intvl{1,n}$, we define $\rho_{q}(x,P)=\min \{k\in \intvl{1,m} \suchthat n_{k}(x,P)\geq q\}$ as the minimal rank $k$ at which the $k$-support that $x$ gets at $P$ is at least $q$. We
write $\rho _{q}(P) = \min_{x \in A} \set{\rho_{q}(x, P)}$ for the minimal rank $k$ at which the $k$-support that some alternative gets at $P$ is at least $q$. A \emph{\ac{BK} compromise with threshold }$q$ is the
\ac{SCR} $f_{q}$ defined for each $P\in \linors^N$ as $f_{q}(P)=\{x\in A \suchthat n_{\rho _{q}(P)}(x,P)\geq q\}$. \commentBN{Changed definition from this $f_{q}(P)=\{x\in A \suchthat n_{\rho _{q}(P)}(x,P)\geq n_{\rho _{q}(P)}(y,P)$ $\forall y\in A\}$ which instead will be the revised version.}
We can also define a revised version of the \acs{BK} compromise where among the winners only the alternatives with the greatest support are selected: $f'_{q}(P)=\{x\in A \suchthat n_{\rho _{q}(P)}(x,P)\geq n_{\rho _{q}(P)}(y,P), \forall y\in A\}$

We first consider the \acs{BK} compromise with threshold $q=n$, $f_n$, which corresponds to the rule also known as \textit{fallback bargaining} \citep{Brams2001}.

\begin{theorem}
	\label{th:FBsatsPC}
	Let $n\geq 3$ and $m\geq 3.$ The \acs{BK} compromise $f_{n}$ satisfies PC.
\end{theorem}

\begin{proof}
	Define $\bar{\sigma } \in \Sigma$ as, $\forall l \in \intvl{0,m-1}^N$: $\bar\sigma(l) = 1$ iff $\exists i, j \in N \suchthat l_i ≠ l_j$; $\bar\sigma(l) = 0$ otherwise.
	Considering any $x \in f_n(P)$, let us show that $x \in \mustar[\bar{\sigma}]$. Because $x \in f_n(P)$, $x \in \paretopt(P)$, and therefore, suffices to show that $\forall y \in \paretopt(P)$, $\bar{\sigma}(\lambda_P(y)) ≥ \bar{\sigma}(\lambda_P(x))$. Given the choice of $\bar{\sigma}$, picking any $y \in \paretopt(P)$ with $y≠x$, suffices to show that $\bar{\sigma}(\lambda_P(y)) = 1$, equivalently, that $\exists i, j \in N \suchthat r_{\prefi}(y) ≠ r_{\pref_j}(y)$. 
	Because $x \in f_n(P)$, $\rho_n(P) = \rho_n(x, P) = \max_{N} r_{\prefi}(x)$.
	It follows from $\rho_n(P) = \min_{z \in A} \set{\rho_n(z, P)}$ that $\rho_n(y, P) ≥ \rho_n(x, P)$, thus, $\exists i \in N \suchthat r_{\prefi}(y) ≥ \rho_n(P)$. 
	Also, $y \in \paretopt(P)$ implies that $\exists j \in N \suchthat r_{\pref_j}(y) < r_{\pref_j}(x)$, thus $\exists j \in N \suchthat r_{\pref_j}(y) < \rho_n(P)$. 
	Therefore, $r_{\prefi}(y) ≠ r_{\pref_j}(y)$.
\end{proof}

\begin{theorem}
	\label{th:FBfailsECC}
	Let $n\geq 3$ and $m\geq 3.$ The BK compromise $f_{n}$ fails ECC. 
\end{theorem}
\begin{proof}
	As $f_{n}$ is Paretian, the proof comes straightforward from \cref{th:nonParetian}.
\end{proof}

\begin{theorem}
	\label{th:BKthreshold}
	Let $n\geq 3$ and $m\geq 3.$ A BK compromise $f_{q}$ with threshold $q \in \intvl{1, n-1}$ is neither ECC nor PCC.
\end{theorem}
\begin{proof}
	%Take any $n\geq 3$ and $m\geq 3.$ Let $A=\left\{ a_{1},\text{ }a_{2,}...
	%\text{ }a_{m}\right\} $. Pick some $q\in \left\{ 1,...,n\right\} $ and
	%consider the BK compromise $f_{q}$. 
	Consider the following profile $P$ (also used in the proof of \cref{th:condorcet}), where the dots represent the sequence $a_4$ to $a_m$:
	\begin{center}
		$
		\begin{array}{cccccc}
			\mathbf{n-1} \quad &a_1&a_2&a_3&\dots\\
			\mathbf{1} \quad &a_3&a_2&\dots&a_1\\
		\end{array}
		$\quad.
	\end{center}
	\commentBN{Changed the proof after the new definition.}
	When $q=1$ we have that $f_{1}(P)=\{a_1,a_3\}$ and when $q \in \intvl{2, n-1}$ we have that $f_{q}(P)=\{a_1\}$. Because $\sigma(\lambda_P(a_2)) = 0$ and $\sigma(\lambda_P(a_1)) > 0$ and $\sigma(\lambda_P(a_3)) > 0$, neither $\musigma$ nor $\mustar$ contain $a_1$ nor $a_3$ for any $\sigma \in \Sigma$. 

	Note that for $q \in \intvl{1, n-1}$ we have that $f'_{q}(P)=\{a_1\}$ since the revised version of the \acs{BK} compromise selects the alternatives with the greatest support. This version of the rule was used in the proof for \Cref{th:BKthreshold} published by \citet{Cailloux2022}.
	%Remzi’s proof
	%Take any $n\geq 3$ and $m\geq 3.$ Let $A=\left\{ a_{1},\text{ }a_{2,}...%
	%\text{ }a_{m}\right\} $. Pick some $q\in \left\{ 1,...,n\right\} $ and
	%consider the BK compromise $f_{q}$. Consider the profile $P\in \linors^{N}$ such that 
	%$a_{1}\succ _{i}a_{2}\succ _{i}...\succ _{i}a_{m}$ $\forall i\in N\diagdown
	%\left\{ n\right\} $ and $a_{\pi (1)}\succ _{n}a_{\pi (2)}\succ _{n}...\succ
	%_{n}a_{\pi (m)}$ where $\pi $ is a bijection on $\left\{ 1,\text{ }2,...,%
	%\text{ }m\right\} $ with $\pi (1)=3$, $\pi (2)=2,\pi (3)=1$, $\pi (i)=i+1$ $%
	%\forall i\in \left\{ 4,...,\text{ }m-1\right\} $ and $\pi (m)=4 $, we have $%
	%f_{q}(P)=\left\{ a_{1}\right\} $ while $\mu _{\sigma }(P)=\mu _{\sigma
	%}^{\ast }(P)=\left\{ a_{2}\right\} $ $\forall \sigma \in \Sigma $.
\end{proof}

\subsection{Restrictions on sigma}
\label{sec:RestrictionOnSigma}
The perfect equality recognition condition we adopt for spread measures, i.e., that the spread gets its lowest value $0$ in case of perfect equality and only in this case, is very basic. Unless this condition is violated, $\Sigma$ is the largest set of spread measures we could conceive. On the other hand, it is possible to let $\Sigma$ shrink by imposing additional conditions over spread measures. Nevertheless, as the satisfaction of PC, PCC, EC, or ECC requires the existence of a spread measure, all of our negative results, namely, those expressed by Theorems \ref{th:nonParetian}, \ref{th:incompatibility}, \ref{th:condorcet}, \ref{th:srECC}, \ref{th:srPCC}, \ref{th:FBfailsECC} and \ref{th:BKthreshold} prevail when $\Sigma$ is restricted. In a similar vein, the positive results in Theorems \ref{th:AntSatsPCC} and \ref{th:FBsatsPC} risk to be lost with additional conditions over spread measures.
Indeed, this section shows that a mild restriction removes the positive results concerning the only two rules that we found to be compatible with any of our compromise concepts.

\begin{definition}
	\label{def:conditionC}
	Given any $m\geq4$ and $n\geq \max\{4,m-1\}$, we say that a spread measure $\sigma$ satisfies condition $C_{m,n}$ iff we have $\sigma(m-3, m-1, m-2, \dots, m-2) < \sigma(m-2, m-3, \dots, 1, 0, \dots, 0)$.
\end{definition}

As both vectors are $n$ dimensional, the term $m-2$ repeats $n-2$ times in the first vector and the term $0$ repeats $n-m+2$ times in the second vector.

This condition imposes a very reasonable requirement on spread measures for large values of $m$ and $n$. Asking for $\sigma(1,3,2,2)$ to be smaller than $\sigma(2,1,0,0)$ is demanding while asking for $\sigma(5,7,6,6,6,6,6)$ to be smaller than $\sigma(6,5,4,3,2,1,0)$ reflects a mild assumption. In any case, as we state below, several well-known spread measures of the literature (see \citet{Allison1978} for a comprehensive account) satisfy \cref{def:conditionC}. Letting $\bar{l} = \sum_{i=1}^{n} l_i / n$ denote the arithmetic mean of the values of $l = (l_1, …, l_n)$, we consider the following measures:

\begin{itemize}
	\item the mean absolute difference $\sigma_{mad}(l)= \frac{1}{n^2} \sum_{i=1}^{n}\sum_{j=1}^{n}|l_i-l_j|$;
	\item the average absolute deviation $\sigma_{ad}(l)= \frac{\sum_{i=1}^{n}|l_i-\bar{l}|}{n}$;
	\item the standard deviation $\sigma_{sd}(l)= \sqrt{\frac{\sum_{i=1}^{n}(l_i-\bar{l})^2}{n}}$;
	\item the Gini coefficient $\sigma_{G}(l)= \frac{\sum_{i=1}^{n}\sum_{j=1}^{n}|l_i-l_j|}{2 \cdot n \cdot \sum_{i=1}^{n} l_i}$.
\end{itemize} 

\begin{remark}
	\label{prop:spreadMeas}
	We checked experimentally that $\sigma_{mad}$, $\sigma_{ad}$, $\sigma_{sd}$ and $\sigma_{G}$ all satisfy condition $C_{m,n}$, for $m \in \intvl{4,1000}$ and $n \in \intvl{b, 1000}$ where $b = \max\{4,m-1\}$.
\end{remark}

%A \emph{spread measure} $\sigma: \alllosses → \R_{+}$ satisfies condition gamma iff  $\sigma (m-3,$ $m-1,m-2,...,$ $m-2)$ <$\sigma(m-2,$ $m-1,...1,$0, $\ 0)$.
%			(\lambda_{P}(y))$ that associates a spread value to every possible loss
%vector. We write On the other hand, $\lambda
%			^{P}(x)=(m-3,$ $m-1,m-2,...,$ $m-2)$ and $\lambda_{P}(y)=(m-2,$ $m-1,,...1,$
%			$0,$ $\ 0)$.

We write $\Sigma^{C_{m,n}} \subseteq \Sigma$ for the set of spread measures that satisfy condition $C_{m,n}$. 
\begin{theorem} 
	\label{th:3votRestriction}
	For all $m\geq 4$, $n\geq \max\{4,m-1\}$, under $\Sigma^{C_{m,n}}$,
	\begin{itemize}
		\item [1)] $f_n^{w}$ fails PCC when $w$ is the antiplurality score vector;
		\item [2)] the \acs{BK} compromise  $f_n$ fails PCC.
	\end{itemize}
\end{theorem}
\begin{proof}
	Given any $\sigma \in \Sigma^{C_{m,n}}$, let us show that there exists a profile $P$ such that $f_n^{w} \cap \mustar = \emptyset$ and $f_n \cap \mustar = \emptyset$. To that aim, consider some $x,y\in A$ and some $P\in \linors^{N}$ with $r_{\prefi[1]}(x)=m-2$, $r_{\prefi[2]}(x)=m,$ $r_{\prefi}(x)=m-1$ $\forall i\in N \setminus \left\{ 1, 2\right\}$, and $r_{\prefi}(y)=m-i$ $\forall i\in \intvl{1,m-1}$, $r_{\prefi[j]}(y)=1$ $\forall j\in \intvl{m,n}$. Moreover, for each $z\in A \setminus \left\{ x,y\right\} $, let $r_{\prefi[i]}(z)=m$ for some $i\in N$. 
	
	Note that $f_n^{w}(P) = f_{n}(P) = \set{y}$. On the other hand, $\lambda_{P}(x)=(m-3, m-1,m-2,\dots,m-2)$ and $\lambda_{P}(y)=(m-2, m-3,\dots,1,0, \dots, 0)$. As $\sigma(\lambda_{P}(x)) < \sigma(\lambda_{P}(y))$ (because $\sigma \in \Sigma^{C_{m,n}}$), we see that $y\notin \mustar$.
\end{proof}

\section{Two-voters case}
\label{sec:2voters}
In addition to \emph{fallback bargaining (FB)} \citep{Brams2001} (defined in \cref{sec:BKn3}), we consider three prominent solutions of the literature.

\emph{Pareto-and-Veto rules (PV)} \citep{Moulin1983, Abreu1991, Laslier2020} distribute a veto power of $v_1$ and $v_2$ alternatives to voters 1 and 2, respectively, with $v_1+v_2=m-1$. So, every voter $i=1,2$ (simultaneously) vetoes his worst $v_i$ alternatives. The \ac{SCR} picks all non-vetoed and Pareto optimal alternatives.

The \emph{Veto-Rank mechanism (VR)} is commonly used in the selection of arbitrators \citep{Clippel2014}. Given a list of $m$ (odd) alternatives (that are candidates to be arbitrators), each of the two voters (that are the two parties that must agree on an arbitrator) simultaneously vetoes his worst $\frac{m-1}{2}$ alternatives. The selected alternatives are the ones with the highest Borda score among the non-vetoed alternatives.

Again within the context of selecting arbitrators, \citet{Clippel2014} propose and analyse \emph{Shortlisting (SL)} where one of the two parties starts by vetoing her worst $\frac{m-1}{2}$ alternatives ($m$ being odd), and then the second party chooses her best alternative out of the remaining ones. As the outcome of the procedure depends on the party that starts, symmetry among players is ensured by defining the solution as the union of the two outcomes where one and the other party starts.


\begin{definition}
	Given any $m \geq 7$, a spread measure $\sigma \in \Sigma$ satisfies condition $D_m$ iff 
	$\sigma(\ceil{\frac{m}{2}}, \ceil{\frac{m}{2}} - 2) < \sigma(0, \ceil{\frac{m}{2}} - 1)$ and 
	$\sigma(\ceil{\frac{m}{2}} - 2, \ceil{\frac{m}{2}}) < \sigma(\ceil{\frac{m}{2}} - 1, 0)$.
\end{definition}

For $m=7$ the condition requires $\sigma(4, 2) < \sigma(0, 3)$ and $\sigma(2, 4) < \sigma(3, 0)$ which is reasonable in our context. When the value of $m$ is larger, the condition appears even more convincing. As $m$ grows, the distance between $0$ and $\ceil{\frac{m}{2}} - 1$ grows, while the distance between $\ceil{\frac{m}{2}}$ and $\ceil{\frac{m}{2}} - 2$ remains constant. Requiring, for example, the spread of $(15, 13)$ to be smaller than the spread of $(0, 14)$ is very reasonable.

We write $\Sigma^{D_{m}} \subseteq \Sigma$ for the set of spread measures that satisfy condition $D_{m}$. 

\begin{theorem} \label{th:2votPCC}
	Let $m \geq 7$. Under $\Sigma^{D_{m}}$, FB and PV fail PCC. Furthermore, when $m$ is odd, VR and SL also fail PCC.
\end{theorem}
\begin{proof}
	Take any $m \geq 7$ and any $\sigma \in \Sigma^{D_m}$. Define $\alpha = \ceil{\frac{m}{2}} - 1$ and $\beta = \ceil{\frac{m}{2}} - 2$. It follows from $\sigma \in \Sigma^{D_{m}}$, that $\sigma(\alpha + 1, \beta) < \sigma(0, \beta + 1)$ and $\sigma(\beta, \alpha + 1) < \sigma(\beta + 1, 0)$.
	For $m$ odd, note that $\alpha + \beta + 2 = m$ and consider the profile $P$ where voter $i_1$ has the preference $x \succ a_1 \succ … \succ a_\alpha \succ y \succ b_1 \succ … \succ b_\beta$ and voter $i_2$ has the preference $b_1 \succ … \succ b_\beta \succ y \succ x \succ a_1 \succ … \succ a_\alpha$. For $m$ even, note that $\alpha + \beta + 3= m$, and define the profile $P$ in the same way, except that a supplementary alternative $z$ is added at the bottom of both rankings.
	
	Note that $\sigma(\lambda_{P}(y)) = \sigma(\alpha + 1, \beta)$ and that $\sigma(\lambda_{P}(x)) = \sigma(0, \beta + 1)$. 
	Therefore, $\sigma(\lambda_{P}(y)) < \sigma(\lambda_{P}(x))$. As $y$ is not Pareto-dominated, an \ac{SCR} that uniquely picks $x$ at $P$ cannot be PCC. In a similar vein, at the profile $P'$ which is obtained by the inversion of the preferences of $i_1$ and $i_2$ at $P$, an \ac{SCR} that is PCC cannot pick $x$ uniquely.	
	
	The proof will be concluded by showing that FB, PV, and (when $m$ is odd) VR and SL all pick only $x$ at $P$ or at $P'$.
	
	We readily see that FB picks only $x$ at $P$ (and at $P'$) since $x$ is the first alternative which reaches the unanimous consent.
	For PV, let $v_{i_1} ≥ v_{i_2}$ (thus $v_{i_1} ≥ \ceil{\frac{m-1}{2}} ≥ \ceil{\frac{m-2}{2}} = \beta + 1$ and $v_{i_2} ≤ \floor{\frac{m - 1}{2}} = \ceil{\frac{m - 2}{2}} = \alpha$), and consider the profile $P$. Observe that the first voter vetoes at least $y$ and every $b_j$ ($1 ≤ j ≤ \beta$) while no voter vetoes $x$. As $x$ Pareto-dominates every $a_j$ ($1 ≤ j ≤ \alpha$), PV picks only $x$ at $P$. When $v_{i_2} ≥ v_{i_1}$, a similar reasoning yields that PV picks only $x$ at $P'$.
	
	Now let $m$ be odd.
	
	For VR, a reasoning similar to the one applied to PV yields $x$ as the unique choice at $P$: each voter vetoes her worst $\frac{m-1}{2}$ alternatives, thus $i_1$ vetoes $y$ and every $b_j$ ($1 ≤ j ≤ \beta$) and $i_2$ vetoes every $a_j$ ($1 ≤ j ≤ \alpha$). The alternative $x$ is the only non-vetoed alternative, so it is selected as the sole winner.
	
	Finally, SL also picks $x$, as it is the unique winner no matter which voter starts the veto phase. If $i_1$ starts, $y$ and every $b_j$ ($1 ≤ j ≤ \beta$) get vetoed, then $i_2$ chooses her best alternative out of the remaining ones which is $x$. If $i_2$ starts, every $a_j$ ($1 ≤ j ≤ \alpha$) get vetoed, then $i_1$ chooses her best alternative which is $x$. 
\end{proof}



\section{Concluding remarks}
\label{sec:conclusion}
We define an ex-post compromise as an outcome where individuals give up as equally as possible from their ideal points. 
With three or more individuals, several well known \acp{SCR} fail to pick ex-post compromises, under any reasonable meaning attributed to “giving up equally”. 
Our findings cover Condorcet extensions and scoring rules but also \acs{BK} compromises, which impose a willingness to compromise without ensuring a compromised outcome.
In particular we find that: no Condorcet procedure is ECC or PCC (\Cref{th:condorcet}), no scoring rule is ECC (\Cref{th:srECC}) or PCC except for the antiplurality rule (\Cref{th:AntSatsPCC,th:srPCC}), \acs{BK} compromises are neither ECC or PCC (\Cref{th:BKthreshold,th:FBfailsECC}) except for fallback bargaining that is PC (\Cref{th:FBsatsPC}).

Our impossibility results are stated for the set of spread measures $\Sigma$, but they prevail for any subset of $\Sigma$. As $\Sigma$ is the largest set of sensible spread measures, they are valid for any specific concept of equity one might pick. On the other hand, our possibility results on fallback bargaining and antiplurality are not propagated to subsets of $\Sigma$. In fact, as soon as a reasonably mild restriction on $\Sigma$ is imposed, both rules are no longer PCC (\Cref{th:3votRestriction}). With two individuals and a similar restriction on $\Sigma$, all well-known two-person SCRs of the literature, namely, fallback bargaining, Pareto and veto rules, short listing and veto rank, fail to pick ex-post compromises (\Cref{th:2votPCC}).

The exclusion of the equal-loss principle by almost all \acp{SCR} of the literature leads to ask whether the principle is uninteresting in a discrete social choice context. This seems to be the case for voting situations where the number of voters exceeds the number of candidates and usually every candidate is ranked last by at least one voter. In these cases, the main concern is about the support of alternatives rather than equality. On the other hand, two-person collective choice problems are typically interpreted as arbitration or bargaining situations where mutual consent is a critical element in reaching a solution. Thus, the equal-loss principle appears to be valid for two-person collective choice problems and our analysis raises the question of designing new discrete arbitration rules compatible with the equal-loss principle. 

We also want to mention that different notions of compromise can be conceived. \citet{Borgers1991} defines as compromises all Pareto-optimal alternatives that are not the top choice of any individual. In this context a compromise does not always exist. This is also a possible approach to adopt and we thank an anonymous reviewer for this remark.

We close by noting, as one anonymous reviewer to whom we are grateful remarked, that the tension between equity and efficiency is not new in economics. In our paper we try to cast this tension in a context where it does not seem to have been considered yet. We certainly hope that this is only the beginning of a discussion that may lead to further progress in the future. In particular, viewing a compromise through the equal loss principle can be especially interesting in richer informational settings with a status-quo point, cardinal individual preferences or a continuum of alternatives.

%\begin{acknowledgements}
%	This paper is a part of the ‘Polarization viewed from a social choice perspective’ (POSOP) research project that is carried on under the RDI program funded by Istanbul Bilgi University. We would like to thank POSOP  for the support. We also thank Jean-François Laslier who provided the inspiration and the basis for this article. Last but not least, we thank the associate editor and three anonymous reviewers for their comments and valuable suggestions.
%\end{acknowledgements}



